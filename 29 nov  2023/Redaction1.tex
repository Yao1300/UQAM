\documentclass[12pt]{memoireuqam1.3}
%\documentclass[12pt,twoside]{memoireuqam1.3}
% Si vous souhaitez en recto-verso
\usepackage{graphicx}% Pour les figures
\usepackage[french]{babel}
\usepackage[utf8]{inputenc} % Pour utiliser les caractères accentués
\usepackage[T1]{fontenc}
\usepackage{url}
\usepackage{cwpuzzle}
\usepackage{xcolor}
\usepackage{logicpuzzle}
\newtheorem{Prop}{Proposition}[subsubsection]
\newtheorem{Theo}{Th\'eor\`eme}[subsubsection]
\newtheorem{Prp}{Propri\'et\'es}[subsubsection]
\newtheorem{Lem}{Lemme}[subsubsection]
\newtheorem{Pre}{Preuve}[subsubsection]
\newtheorem{Rem}{Remarque}[subsubsection]
\newtheorem{Ex}{Exemple}[subsubsection]
\newtheorem{Cor}{Corollaire}[subsubsection]
\newtheorem{Def}{D\'efinition}[subsubsection]

%%%%
% Visitez http://www.labmath.uqam.ca/presentation-de-la-classe-memoireuqam1-3/ pour plus details
%%%%

\begin{document}

%%%%%%%%%%%%%%%%%%%%
% Pour la page titre
%%%%%%%%%%%%%%%%%%%%
\title{Énumération de familles de polyominos inscrits dans un rectangle de largeur fixée et de hauteur variable}
% Votre nom complet tel qu'il apparaît à votre dossier du registrariat de l'UQAM
\author{Akakpo Yao Ihébami}
% Année et mois courant sauf si spécifié autrement pas \degreemonth et \degreeyear
%\degreemonth{mois du dépôt}
%\degreeyear{année du dépôt}
\uqammemoire %% ou \uqamthese ou \uqamrapport
\matiere{mathématiques-Informatiques}


\thispagestyle{empty}        % La page titre n'est pas numérotée
\maketitle

%%%%%%%%%%%%%%%%%%%%
% Page préliminaires
%%%%%%%%%%%%%%%%%%%%
\section*{Résumé}
\newpage
\section*{Abstract}
\newpage
\section*{Remerciements}
\chapter{Introduction}
\section{Background sur les polyominos}
\section{Utilité des polyominos}
\section{Applications(domaines d'applications, problèmes ouverts)}
\section{(Classification des polyominos}
\section{Problématique}
\section{Annonce du plan}
\chapter{Préliminaires}
\section*{Introduction}
\section{Polyomino}
\subsection{Notations}
Tout au long de ce travaille, nous ut
\begin{Puzzle}{3}{3}%
|1 |2 |4 |. |3 |*|5|. |8 |7 |6 |.
\end{Puzzle}
\begin{logicpuzzle}[rows=3,columns=3,color=red!100]
%\valueH{1,2,3}
%\valueV{1,2,3}
\setrow{3}{,,}
\setrow{2}{,{},}
\setrow{1}{,,}
\fillcell{2}{2}
\framepuzzle[black!50]
\end{logicpuzzle}
\subsection{Définitions}
\subsubsection*{Cellule}
\subsection{Exemples}
\subsection{Propriétés et généralités}
\section{Polyomino inscrit dans un rectangle}
\section{Théorie des automates}
\section{Automate décrivant la génération des polyominos inscrits dans un rectangle}
\subsection{Notions d'états}
\subsection{Exemples d'automate décrivant les polyominos inscrits dans un rectangle de largeur 2}
\section*{Conclusion}
\chapter{Dénombrement des polyominos inscrits dans un rectangle de largeur 3 et de hauteur quelconque}
\section*{Introduction}
\section{Automate décrivant les polyominos inscrits dans un rectangle $b\times h$}
\subsection{Les états possibles}
\subsection{États initiaux et états finaux}
\subsection{Transitions entre états}
\subsection{Matrice de tranfert}
\subsection{Définition et génération de la matrice de tranfert}
\subsection{Utilisation de la matrice de transfert pour compter les polyominos}
\section{Quelques résultats}
\subsection{Résultats pour $h=2$}
\subsection{Résultats pour $h=3$}
\subsection{résultats pour $h=4$}
\subsection{Résultats pour $h$ quelconque: formule de récurrence }
\section*{Conclusion}
\chapter{Dénombrement des polyominos inscrits dans un rectangle de largeur 4 et de hauteur quelconque}
\section*{Introduction}
\section{Automate décrivant les polyominos inscrits dans un rectangle de largeur 4}
\subsection{Les états possibles}
\subsection{États initiaux et états finaux}
\subsection{Transitions entre états}
\subsection{Matrice de tranfert}
\subsection{Définition et génération de la matrice de tranfert}
\subsection{Utilisation de la matrice de transfert pour compter les polyominos}
\section{Quelques résultats}
\subsection{Résultats pour $h=4$}
\subsection{Résultats pour $h$ quelconque: formule de récurrence }
\section*{Conclusion}
\chapter{Conclusion et perspectives}
\renewcommand \bibname{R\'EF\'ERENCES}% FACULTATIF
%si vous voulez qu'apparaisse le titre RÉFÉRENCES plutôt que BIBLIOGRAPHIE

\renewcommand \listfigurename{LISTE DES FIGURES}
\renewcommand \appendixname{APPENDICE}
\renewcommand \figurename{Figure}
\renewcommand \tablename{Tableau}

\pagenumbering{roman} % numérotation des pages en chiffres romains
\addtocounter{page}{1} % Pour que les remerciements commencent à la page ii
\chapter*{Remerciements}
Je tiens tout d'abord à remercier mon directeur et mon co-directeur de mémoire, respectivement les professeurs  Alexandre Blondin Massé, du département d’informatique de l’UQAM et Alain  Goupil  du département d’informatique et mathématiques, de l’UQTR qui se sont investis ardemment dans le déroulement de mon projet de recherche. Au travers des interactions continues qui nous ont liés tout au long de nos travaux, ils m’ont mis dans les  meilleures conditions de travail du début jusqu’au bout.  

J'envoie mes sincères remerciements à toute l’équipe du LACIM pour le travail qu’ils font et continuent de faire pour l’avancée de la recherche dans les différents domaines qui y sont représentés.  

Je rends un grand hommage à tous mes enseignants de l’UQAM, ils se  sont beaucoup investis dans ma formation. Ils ont su me donner une formation d’une qualité exceptionnelle avec des programmes très innovants et constamment actualisés.
 
Je remercie  sincèrement madame Isabella Couture, secrétaire de direction et du programme d'études en mathématiques, pour toutes ses prestations au sein du département et, surtout, la rapidité avec laquelle elle nous fait parvenir la mise à jour des informations.

J'adresse mes remerciements aux corps professoral et administratif de l'UQAM, à tous les membres de l'ASEQ et à tous mes collègues de parcours avec qui on a passé des moments de peine, d'enthousiasme et de bonheur. 
\tableofcontents % Pour générer la table des matières
\listoftables % Pour générer la liste des tableaux
\listoffigures % Pour générer la liste des figures
\chapter*{Résumé}
Dans ce mémoire, nous nous intéressons à l'énumération des polyominos et forêts de polyominos inscrits dans un rectangle de largeur et de hauteur quelconques selon des paramètres tels que l'aire, le périmètre, le nombre de feuilles et le nombre de composantes connexes.

 Plus précisément, soit $B$ un entier naturel non nul et $R_{B}$ la famille des rectangles de largeur $B$ et de hauteur entière. Nous construisons des automates $\mathcal{A}_{B}$ et $\mathcal{A}_{FB}$ dont les chemins sont en bijection respectivement avec les polyominos et les forêts de polyominos contenus dans les rectangles de $R_{B}$. Nous construisons les matrices de transition relatives à chacun des automates et les théorèmes garantissant l'inscriptibilité  d'un polyomino dans un rectangle de $R_{B}$. Nous énumérons tous les polyominos et les forêts de polyominos inscrits dans ces types de rectangles.
 
  Pour arriver à nos fins, nous établissons plusieurs formules fondamentales et exactes relevant de la combinatoire, notamment les formules permettant les déterminations d'aire, de périmètre, du nombre de feuilles et du nombre de composantes connexes échangés lors d'une transition d'un état à un autre. Nous présentons de façon détaillée les cas $B=2$ et $B=3$ notamment la génération des matrices de transfert ainsi que les fonctions génératrices associées.
% Utilisez l'environnement  abstract pour rédiger votre résumé


%%%%%%%%%%%%%%%%%%%%
% Document principal
%%%%%%%%%%%%%%%%%%%%

\include{nom_de_fichier_de_l_introduction}
% Utilisez l'environnement  introduction pour rédiger votre introduction
\include{nom_de_fichier_du_premier_chapitre}
\include{nom_de_fichier_du_deuxieme_chapitre}
\include{nom_de_fichier_du_chapitre_suivant}

\include{nom_de_fichier_de_la_conclusion}
% Utilisez l'environnement  conclusion pour rédiger votre conclusion

%%%%%%%%%%%%%%%%%%%%
% Page liminaires
%%%%%%%%%%%%%%%%%%%%

\chapter*{Annexe}
\addcontentsline{toc}{chapter}{ANNEXE}
 \begin{spacing}{0.30}
\section*{Matrices de transitions des automates $\mathcal{A}_{F3}$ et $\mathcal{A}_{3}$}
 \end{spacing}
\addcontentsline{toc}{section}{Matrices de transitions des automates $\mathcal{A}_{F3}$ et $\mathcal{A}_{3}$}
Nous présentons dans cette section les matrices de transitions des automates $\mathcal{A}_{F3}$ et  $\mathcal{A}_{3}$.
\begin{spacing}{0.30} 
\subsection*{Matrice $\mathcal{M}_{F3}$}
 \end{spacing}
\addcontentsline{toc}{section}{Matrice $\mathcal{M}_{F3}$}
En se basant sur les états de $\mathcal{A}_{F3}$ présentés selon l'ordre donné par le tableau \ref{tab1}, nous obtenons la matrice $\mathcal{M}_{F3}$ dans le tableau \ref{tab2}.
\small
\begin{longtable}{|c|c|c|} 
\hline
$e_{1}=(100,000,\{\{0\}\})$&
$e_{2}=(100,100,\{\{0\}\})$&
$e_{3}=(010,000,\{\{0\}\})$\\
$e_{4}=(010,010)\{\{0\}\})$&
$e_{5}=(001,000,\{\{0\}\}$&
$e_{6}=(001,001,\{\{0\}\})$\\
$e_{7}=(110,110,\{\{0\}\})$&
$e_{8}=(110,210,\{\{0\}\})$&
$e_{9}=(110,120,\{\{0\}\})$\\
$e_{10}=(110,220,\{\{0\})$&
$e_{11}=(101,000,\{\{0\},\{1\}\})$& \\
$e_{12}=(101,101,\{\{0\},\{1\}\})$ &
$e_{13}=(101,100,\{\{0\},\{1\}\})$& \\
$e_{14}=(101,001,\{\{0\},\{1\})$&
$e_{15}=(101,101,\{\{0,1\}\})$ &\\
$e_{16}=(011,011,\{\{0\}\})$&
$e_{17}=(011,021,\{\{0\}\})$&
$e_{18}=(011,012,\{\{0\}\})$\\
$e_{19}=(011,022,\{\{0\}\})$&
$e_{20}=(111,121,\{\{0\}\})$&
$e_{21}=(111,221,\{\{0\}\})$\\
$e_{22}=(111,122,\{\{0\}\})$&
$e_{23}=(111,222,\{\{0\}\})$&
$e_{24}=(111,131,\{\{0\}\})$\\
$e_{25}=(111,231,\{\{0\}\})$&
$e_{26}=(111,132,\{\{0\}\})$&
$e_{27}=(111,232,\{\{0\}\})$\}\\
$e_{28}=(000,000,\{\{ \}\})$&
$e_{29}=(10,00,\{\{0\}\})$&
$e_{30}=(10,10,\{\{0\}\})$\\
$e_{31}=(01,00,\{\{0\}\})$&
$e_{32}=(01,01,\{\{0\}\})$&
$e_{33}=(11,11,\{\{0\}\})$\\
$e_{34}=(11,21,\{\{0\}\})$&
$e_{35}=(11,12,\{\{0\}\})$&
$e_{36}=(11,22,\{\{0\}\})$\\
$e_{37}=(00,00,\{\{ \}\})$&
$e_{38}=(1,0,\{\{0\}\})$ & 
$e_{39}=(1,1,\{\{0\}\})$\\
$e_{40}=(0,0,\{\{ \}\}$ & &\\
\hline
\caption{\label{tab1} Les états de $\mathcal{A}_{F3}$.}
\end{longtable}
\normalsize
\tiny
\begin{longtable}{|c|c|c|} 
\hline
$T_{1,1}= 0$&
$T_{1,2}= wx^2yz^2$&
$T_{1,3}= wx^4y\zeta$\\
$T_{1,4}= 0$&
$T_{1,5}= wx^4y\zeta$&
$T_{1,6}= 0$\\
$T_{1,7}= 0$&
$T_{1,8}= w^2x^4yz^2$&
$T_{1,9}= 0$\\
$T_{1,10}= 0$&
$T_{1,11}= 0$&
$T_{1,12}= 0$\\
$T_{1,13}= w^2x^6yz^2\zeta$&
$T_{1,14}= 0$&
$T_{1,15}= 0$\\
$T_{1,16}= w^2x^6yz^2\zeta$&
$T_{1,17}= 0$&
$T_{1,18}= 0$\\
$T_{1,19}= 0$&
$T_{1,20}= 0$&
$T_{1,21}= w^3x^6yz^2$\\
$T_{1,22}= 0$&
$T_{1,23}= 0$&
$T_{1,24}= 0$\\
$T_{1,25}= 0$&
$T_{1,26}= 0$&
$T_{1,27}= 0$\\
$T_{1,28}= y$&
$T_{1,29}= wx^4y$&
$T_{1,30}= 0$\\
$T_{1,31}= wx^4y\zeta$&
$T_{1,32}= 0$&
$T_{1,33}= w^2x^6yz^2$\\
$T_{1,34}= 0$&
$T_{1,35}= 0$&
$T_{1,36}= 0$\\
$T_{1,37}= y$&
$T_{1,38}= wx^4y\zeta$&
$T_{1,39}= 0$\\
$T_{1,40}= y$&
$T_{2,1}= 0$&
$T_{2,2}= wx^2y$\\
$T_{2,3}= wx^4y\zeta$&
$T_{2,4}= 0$&
$T_{2,5}= wx^4y\zeta$\\
$T_{2,6}= 0$&
$T_{2,7}= 0$&
$T_{2,8}= w^2x^4y$\\
$T_{2,9}= 0$&
$T_{2,10}= 0$&
$T_{2,11}= 0$\\
$T_{2,12}= 0$&
$T_{2,13}= w^2x^6y\zeta$&
$T_{2,14}= 0$\\
$T_{2,15}= 0$&
$T_{2,16}= w^2x^6yz^2\zeta$&
$T_{2,17}= 0$\\
$T_{2,18}= 0$&
$T_{2,19}= 0$&
$T_{2,20}= 0$\\
$T_{2,21}= w^3x^6y$&
$T_{2,22}= 0$&
$T_{2,23}= 0$\\
$T_{2,24}= 0$&
$T_{2,25}= 0$&
$T_{2,26}= 0$\\
$T_{2,27}= 0$&
$T_{2,28}= y$&
$T_{2,29}= wx^4y$\\
$T_{2,30}= 0$&
$T_{2,31}= wx^4y\zeta$&
$T_{2,32}= 0$\\
$T_{2,33}= w^2x^6yz^2$&
$T_{2,34}= 0$&
$T_{2,35}= 0$\\
$T_{2,36}= 0$&
$T_{2,37}= y$&
$T_{2,38}= wx^4y\zeta$\\
$T_{2,39}= 0$&
$T_{2,40}= y$&
$T_{3,1}= wx^4y\zeta$\\
$T_{3,2}= 0$&
$T_{3,3}= 0$&
$T_{3,4}= wx^2yz^2$\\
$T_{3,5}= wx^4y\zeta$&
$T_{3,6}= 0$&
$T_{3,7}= 0$\\
$T_{3,8}= 0$&
$T_{3,9}= w^2x^4yz^2$&
$T_{3,10}= 0$\\
$T_{3,11}= w^2x^8y\zeta^2$&
$T_{3,12}= 0$&
$T_{3,13}= 0$\\
$T_{3,14}= 0$&
$T_{3,15}= 0$&
$T_{3,16}= 0$\\
$T_{3,17}= w^2x^4yz^2$&
$T_{3,18}= 0$&
$T_{3,19}= 0$\\
$T_{3,20}= 0$&
$T_{3,21}= 0$&
$T_{3,22}= 0$\\
$T_{3,23}= 0$&
$T_{3,24}= w^3x^6yz^3$&
$T_{3,25}= 0$\\
$T_{3,26}= 0$&
$T_{3,27}= 0$&
$T_{3,28}= y$\\
$T_{3,29}= 0$&
$T_{3,30}= 0$&
$T_{3,31}= 0$\\
$T_{3,32}= 0$&
$T_{3,33}= 0$&
$T_{3,34}= 0$\\
$T_{3,35}= 0$&
$T_{3,36}= 0$&
$T_{3,37}= 0$\\
$T_{3,38}= 0$&
$T_{3,39}= 0$&
$T_{3,40}= 0$\\
$T_{4,1}= wx^4y\zeta$&
$T_{4,2}= 0$&
$T_{4,3}= 0$\\
$T_{4,4}= wx^2y$&
$T_{4,5}= wx^4y\zeta$&
$T_{4,6}= 0$\\
$T_{4,7}= 0$&
$T_{4,8}= 0$&
$T_{4,9}= w^2x^4y$\\
$T_{4,10}= 0$&
$T_{4,11}= w^2x^8y\zeta^2$&
$T_{4,12}= 0$\\
$T_{4,13}= 0$&
$T_{4,14}= 0$&
$T_{4,15}= 0$\\
$T_{4,16}= 0$&
$T_{4,17}= w^2x^4y$&
$T_{4,18}= 0$\\
$T_{4,19}= 0$&
$T_{4,20}= 0$&
$T_{4,21}= 0$\\
$T_{4,22}= 0$&
$T_{4,23}= 0$&
$T_{4,24}= w^3x^6yz$\\
$T_{4,25}= 0$&
$T_{4,26}= 0$&
$T_{4,27}= 0$\\
$T_{4,28}= y$&
$T_{4,29}= 0$&
$T_{4,30}= 0$\\
$T_{4,31}= 0$&
$T_{4,32}= 0$&
$T_{4,33}= 0$\\
$T_{4,34}= 0$&
$T_{4,35}= 0$&
$T_{4,36}= 0$\\
$T_{4,37}= 0$&
$T_{4,38}= 0$&
$T_{4,39}= 0$\\
$T_{4,40}= 0$&
$T_{5,1}= wx^4y\zeta$&
$T_{5,2}= 0$\\
$T_{5,3}= wx^4y\zeta$&
$T_{5,4}= 0$&
$T_{5,5}= 0$\\
$T_{5,6}= wx^2yz^2$&
$T_{5,7}= w^2x^6yz^2\zeta$&
$T_{5,8}= 0$\\
$T_{5,9}= 0$&
$T_{5,10}= 0$&
$T_{5,11}= 0$\\
$T_{5,12}= 0$&
$T_{5,13}= 0$&
$T_{5,14}= w^2x^6yz^2\zeta$\\
$T_{5,15}= 0$&
$T_{5,16}= 0$&
$T_{5,17}= 0$\\
$T_{5,18}= w^2x^4yz^2$&
$T_{5,19}= 0$&
$T_{5,20}= 0$\\
$T_{5,21}= 0$&
$T_{5,22}= w^3x^6yz^2$&
$T_{5,23}= 0$\\
$T_{5,24}= 0$&
$T_{5,25}= 0$&
$T_{5,26}= 0$\\
$T_{5,27}= 0$&
$T_{5,28}= y$&
$T_{5,29}= wx^4y\zeta$\\
$T_{5,30}= 0$&
$T_{5,31}= wx^4y$&
$T_{5,32}= $\\
$T_{5,33}= w^2x^6yz^2$&
$T_{5,34}= 0$&
$T_{5,35}= $\\
$T_{5,36}= 0$&
$T_{5,37}= y$&
$T_{5,38}= wx^4y\zeta$\\
$T_{5,39}= 0$&
$T_{5,40}= y$&
$T_{6,1}= wx^4y\zeta$\\
$T_{6,2}= 0$&
$T_{6,3}= wx^4y\zeta$&
$T_{6,4}= 0$\\
$T_{6,5}= 0$&
$T_{6,6}= wx^2y$&
$T_{6,7}= w^2x^6yz^2\zeta$\\
$T_{6,8}= 0$&
$T_{6,9}= 0$&
$T_{6,10}= 0$\\
$T_{6,11}= 0$&
$T_{6,12}= 0$&
$T_{6,13}= 0$\\
$T_{6,14}= w^2x^6y\zeta$&
$T_{6,15}= 0$&
$T_{6,16}= 0$\\
$T_{6,17}= 0$&
$T_{6,18}= w^2x^4y$&
$T_{6,19}= 0$\\
$T_{6,20}= 0$&
$T_{6,21}= 0$&
$T_{6,22}= w^3x^6y$\\
$T_{6,23}= 0$&
$T_{6,24}= 0$&
$T_{6,25}= 0$\\
$T_{6,26}= 0$&
$T_{6,27}= 0$&
$T_{6,28}= y$\\
$T_{6,29}= wx^4y\zeta$&
$T_{6,30}= 0$&
$T_{6,31}= wx^4y$\\
$T_{6,32}= $&
$T_{6,33}= w^2x^6yz^2$&
$T_{6,34}= 0$\\
$T_{6,35}= $&
$T_{6,36}= 0$&
$T_{6,37}= y$\\
$T_{6,38}= wx^4y\zeta$&
$T_{6,39}= 0$&
$T_{6,40}= y$\\
$T_{7,1}= 0$&
$T_{7,2}= wx^2y$&
$T_{7,3}= 0$\\
$T_{7,4}= wx^2y$&
$T_{7,5}= wx^4y\zeta$&
$T_{7,6}= 0$\\
$T_{7,7}= 0$&
$T_{7,8}= 0$&
$T_{7,9}= 0$\\
$T_{7,10}= w^2x^2yz^{-2}$&
$T_{7,11}= 0$&
$T_{7,12}= 0$\\
$T_{7,13}= w^2x^6y\zeta$&
$T_{7,14}= 0$&
$T_{7,15}= 0$\\
$T_{7,16}= 0$&
$T_{7,17}= w^2x^4y$&
$T_{7,18}= 0$\\
$T_{7,19}= 0$&
$T_{7,20}= 0$&
$T_{7,21}= 0$\\
$T_{7,22}= 0$&
$T_{7,23}= 0$&
$T_{7,24}= 0$\\
$T_{7,25}= w^3x^4yz^{-1}$&
$T_{7,26}= 0$&
$T_{7,27}= 0$\\
$T_{7,28}= y$&
$T_{7,29}= 0$&
$T_{7,30}= wx^2y$\\
$T_{7,31}= wx^4y\zeta$&
$T_{7,32}= 0$&
$T_{7,33}= 0$\\
$T_{7,34}= w^2x^4y$&
$T_{7,35}= 0$&
$T_{7,36}= 0$\\
$T_{7,37}= y$&
$T_{7,38}= wx^4y\zeta$&
$T_{7,39}= 0$\\
$T_{7,40}= y$&
$T_{8,1}= 0$&
$T_{8,2}= wx^2yz$\\
$T_{8,3}= 0$&
$T_{8,4}= wx^2y$&
$T_{8,5}= wx^4y\zeta$\\
$T_{8,6}= 0$&
$T_{8,7}= 0$&
$T_{8,8}= 0$\\
$T_{8,9}= 0$&
$T_{8,10}= w^2x^2yz^{-1}$&
$T_{8,11}= 0$\\
$T_{8,12}= 0$&
$T_{8,13}= w^2x^6yz\zeta$&
$T_{8,14}= 0$\\
$T_{8,15}= 0$&
$T_{8,16}= 0$&
$T_{8,17}= w^2x^4y$\\
$T_{8,18}= 0$&
$T_{8,19}= 0$&
$T_{8,20}= 0$\\
$T_{8,21}= 0$&
$T_{8,22}= 0$&
$T_{8,23}= 0$\\
$T_{8,24}= 0$&
$T_{8,25}= w^3x^4y$&
$T_{8,26}= 0$\\
$T_{8,27}= 0$&
$T_{8,28}= y$&
$T_{8,29}= 0$\\
$T_{8,30}= wx^2y$&
$T_{8,31}= wx^4y\zeta$&
$T_{8,32}= 0$\\
$T_{8,33}= 0$&
$T_{8,34}= w^2x^4y$&
$T_{8,35}= 0$\\
$T_{8,36}= 0$&
$T_{8,37}= y$&
$T_{8,38}= wx^4y\zeta$\\
$T_{8,39}= 0$&
$T_{8,40}= y$&
$T_{9,1}= 0$\\
$T_{9,2}= wx^2y$&
$T_{9,3}= 0$&
$T_{9,4}= wx^2yz$\\
$T_{9,5}= wx^4y\zeta$&
$T_{9,6}= 0$&
$T_{9,7}= 0$\\
$T_{9,8}= 0$&
$T_{9,9}= 0$&
$T_{9,10}= w^2x^2yz^{-1}$\\
$T_{9,11}= 0$&
$T_{9,12}= 0$&
$T_{9,13}= w^2x^6y\zeta$\\
$T_{9,14}= 0$&
$T_{9,15}= 0$&
$T_{9,16}= 0$\\
$T_{9,17}= w^2x^4yz$&
$T_{9,18}= 0$&
$T_{9,19}= 0$\\
$T_{9,20}= 0$&
$T_{9,21}= 0$&
$T_{9,22}= 0$\\
$T_{9,23}= 0$&
$T_{9,24}= 0$&
$T_{9,25}= w^3x^4y$\\
$T_{9,26}= 0$&
$T_{9,27}= 0$&
$T_{9,28}= y$\\
$T_{9,29}= 0$&
$T_{9,30}= wx^2yz$&
$T_{9,31}= wx^4y\zeta$\\
$T_{9,32}= 0$&
$T_{9,33}= 0$&
$T_{9,34}= w^2x^4yz$\\
$T_{9,35}= 0$&
$T_{9,36}= 0$&
$T_{9,37}= y$\\
$T_{9,38}= wx^4y\zeta$&
$T_{9,39}= 0$&
$T_{9,40}= y$\\
$T_{10,1}= 0$&
$T_{10,2}= wx^2yz$&
$T_{10,3}= 0$\\
$T_{10,4}= wx^2yz$&
$T_{10,5}= wx^4y\zeta$&
$T_{10,6}= 0$\\
$T_{10,7}= 0$&
$T_{10,8}= 0$&
$T_{10,9}= 0$\\
$T_{10,10}= w^2x^2y$&
$T_{10,11}= 0$&
$T_{10,12}= 0$\\
$T_{10,13}= w^2x^6yz\zeta$&
$T_{10,14}= 0$&
$T_{10,15}= 0$\\
$T_{10,16}= 0$&
$T_{10,17}= w^2x^4yz$&
$T_{10,18}= 0$\\
$T_{10,19}= 0$&
$T_{10,20}= 0$&
$T_{10,21}= 0$\\
$T_{10,22}= 0$&
$T_{10,23}= 0$&
$T_{10,24}= 0$\\
$T_{10,25}= w^3x^4yz$&
$T_{10,26}= 0$&
$T_{10,27}= 0$\\
$T_{10,28}= y$&
$T_{10,29}= 0$&
$T_{10,30}= wx^2yz$\\
$T_{10,31}= wx^4y\zeta$&
$T_{10,32}= 0$&
$T_{10,33}= 0$\\
$T_{10,34}= w^2x^4yz$&
$T_{10,35}= 0$&
$T_{10,36}= 0$\\
$T_{10,37}= y$&
$T_{10,38}= wx^4y\zeta$&
$T_{10,39}= 0$\\
$T_{10,40}= y$&
$T_{11,1}= 0$&
$T_{11,2}= wx^2yz^2$\\
$T_{11,3}= wx^4y$&
$T_{11,4}= 0$&
$T_{11,5}= 0$\\
$T_{11,6}= wx^2yz^2$&
$T_{11,7}= 0$&
$T_{11,8}= w^2x^4yz^2$\\
$T_{11,9}= 0$&
$T_{11,10}= 0$&
$T_{11,11}= 0$\\
$T_{11,12}= w^2x^4yz^4$&
$T_{11,13}= 0$&
$T_{11,14}= 0$\\
$T_{11,15}= 0$&
$T_{11,16}= 0$&
$T_{11,17}= 0$\\
$T_{11,18}= w^2x^4yz^2$&
$T_{11,19}= 0$&
$T_{11,20}= 0$\\
$T_{11,21}= 0$&
$T_{11,22}= 0$&
$T_{11,23}= w^3x^4yz^2$\\
$T_{11,24}= 0$&
$T_{11,25}= 0$&
$T_{11,26}= 0$\\
$T_{11,27}= 0$&
$T_{11,28}= y$&
$T_{11,29}= wx^4y\zeta$\\
$T_{11,30}= wx^2yz^2\zeta$&
$T_{11,31}= wx^4y\zeta$&
$T_{11,32}= wx^2yz^2\zeta$\\
$T_{11,33}= 0$&
$T_{11,34}= w^2x^4yz^2$&
$T_{11,35}= w^2x^4yz^2$\\
$T_{11,36}= 0$&
$T_{11,37}= 2y$&
$T_{11,38}= wx^4y\zeta$\\
$T_{11,39}= 2wx^2yz^2$&
$T_{11,40}= 3y$&
$T_{12,1}= 0$\\
$T_{12,2}= wx^2y$&
$T_{12,3}= wx^4y$&
$T_{12,4}= 0$\\
$T_{12,5}= 0$&
$T_{12,6}= wx^2y$&
$T_{12,7}= 0$\\
$T_{12,8}= w^2x^4y$&
$T_{12,9}= 0$&
$T_{12,10}= 0$\\
$T_{12,11}= 0$&
$T_{12,12}= w^2x^4y$&
$T_{12,13}= 0$\\
$T_{12,14}= 0$&
$T_{12,15}= 0$&
$T_{12,16}= 0$\\
$T_{12,17}= 0$&
$T_{12,18}= w^2x^4y$&
$T_{12,19}= 0$\\
$T_{12,20}= 0$&
$T_{12,21}= 0$&
$T_{12,22}= 0$\\
$T_{12,23}= w^3x^4yz^{-2}$&
$T_{12,24}= 0$&
$T_{12,25}= 0$\\
$T_{12,26}= 0$&
$T_{12,27}= 0$&
$T_{12,28}= y$\\
$T_{12,29}= wx^4y\zeta$&
$T_{12,30}= wx^2y\zeta$&
$T_{12,31}= wx^4y\zeta$\\
$T_{12,32}= wx^2y\zeta$&
$T_{12,33}= 0$&
$T_{12,34}= w^2x^4y$\\
$T_{12,35}= w^2x^4y$&
$T_{12,36}= 0$&
$T_{12,37}= 2y$\\
$T_{12,38}= wx^4y\zeta$&
$T_{12,39}= 2wx^2y$&
$T_{12,40}= 3y$\\
$T_{13,1}= 0$&
$T_{13,2}= wx^2y$&
$T_{13,3}= wx^4y$\\
$T_{13,4}= 0$&
$T_{13,5}= 0$&
$T_{13,6}= wx^2yz^2$\\
$T_{13,7}= 0$&
$T_{13,8}= w^2x^4y$&
$T_{13,9}= 0$\\
$T_{13,10}= 0$&
$T_{13,11}= 0$&
$T_{13,12}= w^2x^4yz^2$\\
$T_{13,13}= 0$&
$T_{13,14}= 0$&
$T_{13,15}= 0$\\
$T_{13,16}= 0$&
$T_{13,17}= 0$&
$T_{13,18}= w^2x^4yz^2$\\
$T_{13,19}= 0$&
$T_{13,20}= 0$&
$T_{13,21}= 0$\\
$T_{13,22}= 0$&
$T_{13,23}= w^3x^4y$&
$T_{13,24}= 0$\\
$T_{13,25}= 0$&
$T_{13,26}= 0$&
$T_{13,27}= 0$\\
$T_{13,28}= y$&
$T_{13,29}= wx^4y\zeta$&
$T_{13,30}= wx^2y\zeta$\\
$T_{13,31}= wx^4y\zeta$&
$T_{13,32}= wx^2yz^2\zeta$&
$T_{13,33}= 0$\\
$T_{13,34}= w^2x^4y$&
$T_{13,35}= w^2x^4yz^2$&
$T_{13,36}= 0$\\
$T_{13,37}= 2y$&
$T_{13,38}= wx^4y\zeta$&
$T_{13,39}= wx^2y+wx^2yz^2$\\
$T_{13,40}= 3y$&
$T_{14,1}= 0$&
$T_{14,2}= wx^2yz^2$\\
$T_{14,3}= wx^4y$&
$T_{14,4}= 0$&
$T_{14,5}= 0$\\
$T_{14,6}= wx^2y$&
$T_{14,7}= 0$&
$T_{14,8}= w^2x^4yz^2$\\
$T_{14,9}= 0$&
$T_{14,10}= 0$&
$T_{14,11}= 0$\\
$T_{14,12}= w^2x^4yz^2$&
$T_{14,13}= 0$&
$T_{14,14}= 0$\\
$T_{14,15}= 0$&
$T_{14,16}= 0$&
$T_{14,17}= 0$\\
$T_{14,18}= w^2x^4y$&
$T_{14,19}= 0$&
$T_{14,20}= 0$\\
$T_{14,21}= 0$&
$T_{14,22}= 0$&
$T_{14,23}= w^3x^4y$\\
$T_{14,24}= 0$&
$T_{14,25}= 0$&
$T_{14,26}= 0$\\
$T_{14,27}= 0$&
$T_{14,28}= y$&
$T_{14,29}= wx^4y\zeta$\\
$T_{14,30}= wx^2yz^2\zeta$&
$T_{14,31}= wx^4y\zeta$&
$T_{14,32}= wx^2y\zeta$\\
$T_{14,33}= 0$&
$T_{14,34}= w^2x^4yz^2$&
$T_{14,35}= w^2x^4y$\\
$T_{14,36}= 0$&
$T_{14,37}= 2y$&
$T_{14,38}= wx^4y\zeta$\\
$T_{14,39}= wx^2yz^2+wx^2y$&
$T_{14,40}= 3y$&
$T_{15,1}= 0$\\
$T_{15,2}= wx^2y$&
$T_{15,3}= wx^4y\zeta$&
$T_{15,4}= 0$\\
$T_{15,5}= 0$&
$T_{15,6}= wx^2y$&
$T_{15,7}= 0$\\
$T_{15,8}= w^2x^4y$&
$T_{15,9}= 0$&
$T_{15,10}= 0$\\
$T_{15,11}= 0$&
$T_{15,12}= 0$&
$T_{15,13}= 0$\\
$T_{15,14}= 0$&
$T_{15,15}= w^2x^4y$&
$T_{15,16}= 0$\\
$T_{15,17}= 0$&
$T_{15,18}= w^2x^4y$&
$T_{15,19}= 0$\\
$T_{15,20}= 0$&
$T_{15,21}= 0$&
$T_{15,22}= 0$\\
$T_{15,23}= w^3x^4yz^{-2}$&
$T_{15,24}= 0$&
$T_{15,25}= 0$\\
$T_{15,26}= 0$&
$T_{15,27}= 0$&
$T_{15,28}= y$\\
$T_{15,29}= wx^4y$&
$T_{15,30}= wx^2y$&
$T_{15,31}= wx^4y$\\
$T_{15,32}= wx^2y$&
$T_{15,33}= 0$&
$T_{15,34}= w^2x^4y$\\
$T_{15,35}= w^2x^4y$&
$T_{15,36}= 0$&
$T_{15,37}= 2y$\\
$T_{15,38}= wx^4y\zeta$&
$T_{15,39}= 2wx^2y$&
$T_{15,40}= 3y$\\
$T_{16,1}= wx^4y\zeta$&
$T_{16,2}= 0$&
$T_{16,3}= 0$\\
$T_{16,4}= wx^2y$&
$T_{16,5}= 0$&
$T_{16,6}= wx^2y$\\
$T_{16,7}= 0$&
$T_{16,8}= 0$&
$T_{16,9}= w^2x^4y$\\
$T_{16,10}= 0$&
$T_{16,11}= 0$&
$T_{16,12}= 0$\\
$T_{16,13}= 0$&
$T_{16,14}= w^2x^6y\zeta$&
$T_{16,15}= 0$\\
$T_{16,16}= 0$&
$T_{16,17}= 0$&
$T_{16,18}= 0$\\
$T_{16,19}= w^2x^2yz^{-2}$&
$T_{16,20}= 0$&
$T_{16,21}= 0$\\
$T_{16,22}= 0$&
$T_{16,23}= 0$&
$T_{16,24}= 0$\\
$T_{16,25}= 0$&
$T_{16,26}= w^3x^4yz^{-1}$&
$T_{16,27}= 0$\\
$T_{16,28}= y$&
$T_{16,29}= wx^4y$&
$T_{16,30}= $\\
$T_{16,31}= 0$&
$T_{16,32}= wx^2y$&
$T_{16,33}= 0$\\
$T_{16,34}= 0$&
$T_{16,35}= w^2x^4y$&
$T_{16,36}= $\\
$T_{16,37}= y$&
$T_{16,38}= wx^4y\zeta$&
$T_{16,39}= 0$\\
$T_{16,40}= y$&
$T_{17,1}= wx^4y\zeta$&
$T_{17,2}= 0$\\
$T_{17,3}= 0$&
$T_{17,4}= wx^2yz$&
$T_{17,5}= 0$\\
$T_{17,6}= wx^2y$&
$T_{17,7}= 0$&
$T_{17,8}= 0$\\
$T_{17,9}= w^2x^4yz$&
$T_{17,10}= 0$&
$T_{17,11}= 0$\\
$T_{17,12}= 0$&
$T_{17,13}= 0$&
$T_{17,14}= w^2x^6y\zeta$\\
$T_{17,15}= 0$&
$T_{17,16}= 0$&
$T_{17,17}= 0$\\
$T_{17,18}= 0$&
$T_{17,19}= w^2x^2yz^{-1}$&
$T_{17,20}= 0$\\
$T_{17,21}= 0$&
$T_{17,22}= 0$&
$T_{17,23}= 0$\\
$T_{17,24}= 0$&
$T_{17,25}= 0$&
$T_{17,26}= w^3x^4y$\\
$T_{17,27}= 0$&
$T_{17,28}= y$&
$T_{17,29}= wx^4y$\\
$T_{17,30}= $&
$T_{17,31}= 0$&
$T_{17,32}= wx^2yz$\\
$T_{17,33}= 0$&
$T_{17,34}= 0$&
$T_{17,35}= w^2x^4yz$\\
$T_{17,36}= $&
$T_{17,37}= y$&
$T_{17,38}= wx^4y\zeta$\\
$T_{17,39}= 0$&
$T_{17,40}= y$&
$T_{18,1}= wx^4y\zeta$\\
$T_{18,2}= 0$&
$T_{18,3}= 0$&
$T_{18,4}= wx^2y$\\
$T_{18,5}= 0$&
$T_{18,6}= wx^2yz$&
$T_{18,7}= 0$\\
$T_{18,8}= 0$&
$T_{18,9}= w^2x^4y$&
$T_{18,10}= 0$\\
$T_{18,11}= 0$&
$T_{18,12}= 0$&
$T_{18,13}= 0$\\
$T_{18,14}= w^2x^6yz\zeta$&
$T_{18,15}= 0$&
$T_{18,16}= 0$\\
$T_{18,17}= 0$&
$T_{18,18}= 0$&
$T_{18,19}= w^2x^2yz^{-1}$\\
$T_{18,20}= 0$&
$T_{18,21}= 0$&
$T_{18,22}= 0$\\
$T_{18,23}= 0$&
$T_{18,24}= 0$&
$T_{18,25}= 0$\\
$T_{18,26}= w^3x^4y$&
$T_{18,27}= 0$&
$T_{18,28}= y$\\
$T_{18,29}= wx^4y$&
$T_{18,30}= $&
$T_{18,31}= 0$\\
$T_{18,32}= wx^2y$&
$T_{18,33}= 0$&
$T_{18,34}= 0$\\
$T_{18,35}= w^2x^4y$&
$T_{18,36}= $&
$T_{18,37}= y$\\
$T_{18,38}= wx^4y\zeta$&
$T_{18,39}= 0$&
$T_{18,40}= y$\\
$T_{19,1}= wx^4y\zeta$&
$T_{19,2}= 0$&
$T_{19,3}= 0$\\
$T_{19,4}= wx^2yz$&
$T_{19,5}= 0$&
$T_{19,6}= wx^2yz$\\
$T_{19,7}= 0$&
$T_{19,8}= 0$&
$T_{19,9}= w^2x^4yz$\\
$T_{19,10}= 0$&
$T_{19,11}= 0$&
$T_{19,12}= 0$\\
$T_{19,13}= 0$&
$T_{19,14}= w^2x^6yz\zeta$&
$T_{19,15}= 0$\\
$T_{19,16}= 0$&
$T_{19,17}= 0$&
$T_{19,18}= 0$\\
$T_{19,19}= w^2x^2y$&
$T_{19,20}= 0$&
$T_{19,21}= 0$\\
$T_{19,22}= 0$&
$T_{19,23}= 0$&
$T_{19,24}= 0$\\
$T_{19,25}= 0$&
$T_{19,26}= w^3x^4yz$&
$T_{19,27}= 0$\\
$T_{19,28}= y$&
$T_{19,29}= wx^4y$&
$T_{19,30}= $\\
$T_{19,31}= 0$&
$T_{19,32}= wx^2yz$&
$T_{19,33}= 0$\\
$T_{19,34}= 0$&
$T_{19,35}= w^2x^4yz$&
$T_{19,36}= $\\
$T_{19,37}= y$&
$T_{19,38}= wx^4y\zeta$&
$T_{19,39}= 0$\\
$T_{19,40}= y$&
$T_{20,1}= 0$&
$T_{20,2}= wx^2y$\\
$T_{20,3}= 0$&
$T_{20,4}= wx^2yz$&
$T_{20,5}= 0$\\
$T_{20,6}= wx^2y$&
$T_{20,7}= 0$&
$T_{20,8}= 0$\\
$T_{20,9}= 0$&
$T_{20,10}= w^2x^2yz^{-1}$&
$T_{20,11}= 0$\\
$T_{20,12}= w^2x^4y\zeta$&
$T_{20,13}= 0$&
$T_{20,14}= 0$\\
$T_{20,15}= w^2x^4y$&
$T_{20,16}= 0$&
$T_{20,17}= 0$\\
$T_{20,18}= 0$&
$T_{20,19}= w^2x^2yz^{-1}$&
$T_{20,20}= 0$\\
$T_{20,21}= 0$&
$T_{20,22}= 0$&
$T_{20,23}= 0$\\
$T_{20,24}= 0$&
$T_{20,25}= 0$&
$T_{20,26}= 0$\\
$T_{20,27}= w^3x^2yz^{-2}$&
$T_{20,28}= y$&
$T_{20,29}= 0$\\
$T_{20,30}= wx^2y+wx^2yz$&
$T_{20,31}= 0$&
$T_{20,32}= wx^2yz+wx^2y$\\
$T_{20,33}= 0$&
$T_{20,34}= 0$&
$T_{20,35}= 0$\\
$T_{20,36}= 2w^2x^2yz^{-1}$&
$T_{20,37}= 2y$&
$T_{20,38}= 0$\\
$T_{20,39}= 2wx^2y+wx^2yz$&
$T_{20,40}= 3y$&
$T_{21,1}= 0$\\
$T_{21,2}= wx^2yz$&
$T_{21,3}= 0$&
$T_{21,4}= wx^2yz$\\
$T_{21,5}= 0$&
$T_{21,6}= wx^2y$&
$T_{21,7}= 0$\\
$T_{21,8}= 0$&
$T_{21,9}= 0$&
$T_{21,10}= w^2x^2y$\\
$T_{21,11}= 0$&
$T_{21,12}= w^2x^4yz\zeta$&
$T_{21,13}= 0$\\
$T_{21,14}= 0$&
$T_{21,15}= w^2x^4yz$&
$T_{21,16}= 0$\\
$T_{21,17}= 0$&
$T_{21,18}= 0$&
$T_{21,19}= w^2x^2yz^{-1}$\\
$T_{21,20}= 0$&
$T_{21,21}= 0$&
$T_{21,22}= 0$\\
$T_{21,23}= 0$&
$T_{21,24}= 0$&
$T_{21,25}= 0$\\
$T_{21,26}= 0$&
$T_{21,27}= w^3x^2yz^{-1}$&
$T_{21,28}= y$\\
$T_{21,29}= 0$&
$T_{21,30}= 2wx^2yz$&
$T_{21,31}= 0$\\
$T_{21,32}= wx^2yz+wx^2y$&
$T_{21,33}= 0$&
$T_{21,34}= 0$\\
$T_{21,35}= 0$&
$T_{21,36}= w^2x^2y+w^2x^2yz^{-1}$&
$T_{21,37}= 2y$\\
$T_{21,38}= 0$&
$T_{21,39}= 2wx^2yz+wx^2y$&
$T_{21,40}= 3y$\\
$T_{22,1}= 0$&
$T_{22,2}= wx^2y$&
$T_{22,3}= 0$\\
$T_{22,4}= wx^2yz$&
$T_{22,5}= 0$&
$T_{22,6}= wx^2yz$\\
$T_{22,7}= 0$&
$T_{22,8}= 0$&
$T_{22,9}= 0$\\
$T_{22,10}= w^2x^2yz^{-1}$&
$T_{22,11}= 0$&
$T_{22,12}= w^2x^4yz\zeta$\\
$T_{22,13}= 0$&
$T_{22,14}= 0$&
$T_{22,15}= w^2x^4yz$\\
$T_{22,16}= 0$&
$T_{22,17}= 0$&
$T_{22,18}= 0$\\
$T_{22,19}= w^2x^2y$&
$T_{22,20}= 0$&
$T_{22,21}= 0$\\
$T_{22,22}= 0$&
$T_{22,23}= 0$&
$T_{22,24}= 0$\\
$T_{22,25}= 0$&
$T_{22,26}= 0$&
$T_{22,27}= w^3x^2yz^{-1}$\\
$T_{22,28}= y$&
$T_{22,29}= 0$&
$T_{22,30}= wx^2y+wx^2yz$\\
$T_{22,31}= 0$&
$T_{22,32}= 2wx^2yz$&
$T_{22,33}= 0$\\
$T_{22,34}= 0$&
$T_{22,35}= 0$&
$T_{22,36}= w^2x^2yz^{-1}+w^2x^2y$\\
$T_{22,37}= 2y$&
$T_{22,38}= 0$&
$T_{22,39}= 2wx^2yz+wx^2y$\\
$T_{22,40}= 3y$&
$T_{23,1}= 0$&
$T_{23,2}= wx^2yz$\\
$T_{23,3}= 0$&
$T_{23,4}= wx^2yz$&
$T_{23,5}= 0$\\
$T_{23,6}= wx^2yz$&
$T_{23,7}= 0$&
$T_{23,8}= 0$\\
$T_{23,9}= 0$&
$T_{23,10}= w^2x^2y$&
$T_{23,11}= 0$\\
$T_{23,12}= w^2x^4yz^2\zeta$&
$T_{23,13}= 0$&
$T_{23,14}= 0$\\
$T_{23,15}= w^2x^4yz^2$&
$T_{23,16}= 0$&
$T_{23,17}= 0$\\
$T_{23,18}= 0$&
$T_{23,19}= w^2x^2y$&
$T_{23,20}= 0$\\
$T_{23,21}= 0$&
$T_{23,22}= 0$&
$T_{23,23}= 0$\\
$T_{23,24}= 0$&
$T_{23,25}= 0$&
$T_{23,26}= 0$\\
$T_{23,27}= w^3x^2y$&
$T_{23,28}= y$&
$T_{23,29}= 0$\\
$T_{23,30}= 2wx^2yz$&
$T_{23,31}= 0$&
$T_{23,32}= 2wx^2yz$\\
$T_{23,33}= 0$&
$T_{23,34}= 0$&
$T_{23,35}= 0$\\
$T_{23,36}= 2w^2x^2y$&
$T_{23,37}= 2y$&
$T_{23,38}= 0$\\
$T_{23,39}= 3wx^2yz$&
$T_{23,40}= 3y$&
$T_{24,1}= 0$\\
$T_{24,2}= wx^2y$&
$T_{24,3}= 0$&
$T_{24,4}= wx^2yz$\\
$T_{24,5}= 0$&
$T_{24,6}= wx^2y$&
$T_{24,7}= 0$\\
$T_{24,8}= 0$&
$T_{24,9}= 0$&
$T_{24,10}= w^2x^2yz^{-1}$\\
$T_{24,11}= 0$&
$T_{24,12}= w^2x^4y\zeta$&
$T_{24,13}= 0$\\
$T_{24,14}= 0$&
$T_{24,15}= w^2x^4y$&
$T_{24,16}= 0$\\
$T_{24,17}= 0$&
$T_{24,18}= 0$&
$T_{24,19}= w^2x^2yz^{-1}$\\
$T_{24,20}= 0$&
$T_{24,21}= 0$&
$T_{24,22}= 0$\\
$T_{24,23}= 0$&
$T_{24,24}= 0$&
$T_{24,25}= 0$\\
$T_{24,26}= 0$&
$T_{24,27}= w^3x^2yz^{-2}$&
$T_{24,28}= y$\\
$T_{24,29}= 0$&
$T_{24,30}= wx^2y+wx^2yz$&
$T_{24,31}= 0$\\
$T_{24,32}= wx^2yz+wx^2y$&
$T_{24,33}= 0$&
$T_{24,34}= 0$\\
$T_{24,35}= 0$&
$T_{24,36}= 2w^2x^2yz^{-1}$&
$T_{24,37}= 2y$\\
$T_{24,38}= 0$&
$T_{24,39}= 2wx^2y+wx^2yz$&
$T_{24,40}= 3y$\\
$T_{25,1}= 0$&
$T_{25,2}= wx^2yz$&
$T_{25,3}= 0$\\
$T_{25,4}= wx^2yz$&
$T_{25,5}= 0$&
$T_{25,6}= wx^2y$\\
$T_{25,7}= 0$&
$T_{25,8}= 0$&
$T_{25,9}= 0$\\
$T_{25,10}= w^2x^2y$&
$T_{25,11}= 0$&
$T_{25,12}= w^2x^4yz\zeta$\\
$T_{25,13}= 0$&
$T_{25,14}= 0$&
$T_{25,15}= w^2x^4yz$\\
$T_{25,16}= 0$&
$T_{25,17}= 0$&
$T_{25,18}= 0$\\
$T_{25,19}= w^2x^2yz^{-1}$&
$T_{25,20}= 0$&
$T_{25,21}= 0$\\
$T_{25,22}= 0$&
$T_{25,23}= 0$&
$T_{25,24}= 0$\\
$T_{25,25}= 0$&
$T_{25,26}= 0$&
$T_{25,27}= w^3x^2yz^{-1}$\\
$T_{25,28}= y$&
$T_{25,29}= 0$&
$T_{25,30}= 2wx^2yz$\\
$T_{25,31}= 0$&
$T_{25,33}= 0$&
$T_{25,32}= wx^2yz+wx^2y$\\
$T_{25,34}= 0$&
$T_{25,35}= 0$&
$T_{25,36}= w^2x^2y+w^2x^2yz^{-1}$\\
$T_{25,37}= 2y$&
$T_{25,38}= 0$&
$T_{25,39}= 2wx^2yz+wx^2y$\\
$T_{25,40}= 3y$&
$T_{26,1}= 0$&
$T_{26,2}= wx^2y$\\
$T_{26,3}= 0$&
$T_{26,4}= wx^2yz$&
$T_{26,5}= 0$\\
$T_{26,6}= wx^2yz$&
$T_{26,7}= 0$&
$T_{26,8}= 0$\\
$T_{26,9}= 0$&
$T_{26,10}= w^2x^2yz^{-1}$&
$T_{26,11}= 0$\\
$T_{26,12}= w^2x^4yz\zeta$&
$T_{26,13}= 0$&
$T_{26,14}= 0$\\
$T_{26,15}= w^2x^4yz$&
$T_{26,16}= 0$&
$T_{26,17}= 0$\\
$T_{26,18}= 0$&
$T_{26,19}= w^2x^2y$&
$T_{26,20}= 0$\\
$T_{26,21}= 0$&
$T_{26,22}= 0$&
$T_{26,23}= 0$\\
$T_{26,24}= 0$&
$T_{26,25}= 0$&
$T_{26,26}= 0$\\
$T_{26,27}= w^3x^2yz^{-1}$&
$T_{26,28}= y$&
$T_{26,29}= 0$\\
$T_{26,30}= wx^2y+wx^2yz$&
$T_{26,31}= 0$&
$T_{26,32}= 2wx^2yz$\\
$T_{26,33}= 0$&
$T_{26,34}= 0$&
$T_{26,35}= 0$\\
$T_{26,36}= w^2x^2yz^{-1}+w^2x^2y$&
$T_{26,37}= 2y$&
$T_{26,38}= 0$\\
$T_{26,39}= 2wx^2yz+wx^2y$&
$T_{26,40}= 3y$&
$T_{27,1}= 0$\\
$T_{27,2}= wx^2yz$&
$T_{27,3}= 0$&
$T_{27,4}= wx^2yz$\\
$T_{27,5}= 0$&
$T_{27,6}= wx^2yz$&
$T_{27,7}= 0$\\
$T_{27,8}= 0$&
$T_{27,9}= 0$&
$T_{27,10}= w^2x^2y$\\
$T_{27,11}= 0$&
$T_{27,12}= w^2x^4yz^2\zeta$&
$T_{27,13}= 0$\\
$T_{27,14}= 0$&
$T_{27,15}= w^2x^4yz^2$&
$T_{27,16}= 0$\\
$T_{27,17}= 0$&
$T_{27,18}= 0$&
$T_{27,19}= w^2x^2y$\\
$T_{27,20}= 0$&
$T_{27,21}= 0$&
$T_{27,22}= 0$\\
$T_{27,23}= 0$&
$T_{27,24}= 0$&
$T_{27,25}= 0$\\
$T_{27,26}= 0$&
$T_{27,27}= w^3x^2y$&
$T_{27,28}= y$\\
$T_{27,29}= 0$&
$T_{27,30}= 2wx^2yz$&
$T_{27,31}= 0$\\
$T_{27,32}= 2wx^2yz$&
$T_{27,33}= 0$&
$T_{27,34}= 0$\\
$T_{27,35}= 0$&
$T_{27,36}= 2w^2x^2y$&
$T_{27,37}= 2y$\\
$T_{27,38}= 0$&
$T_{27,39}= 3wx^2yz$&
$T_{27,40}= 3y$\\
$T_{28,1}= wx^4y\zeta$&
$T_{28,2}= 0$&
$T_{28,3}= wx^4y\zeta$\\
$T_{28,4}= 0$&
$T_{28,5}= wx^4y\zeta$&
$T_{28,6}= 0$\\
$T_{28,7}= w^2x^6yz^2\zeta$&
$T_{28,8}= 0$&
$T_{28,9}= 0$\\
$T_{28,10}= 0$&
$T_{28,11}= w^2x^8y\zeta^2$&
$T_{28,12}= 0$\\
$T_{28,13}= 0$&
$T_{28,14}= 0$&
$T_{28,15}= 0$\\
$T_{28,16}= w^2x^6yz^2\zeta$&
$T_{28,17}= 0$&
$T_{28,18}= 0$\\
$T_{28,19}= 0$&
$T_{28,20}= w^3x^8yz^2\zeta$&
$T_{28,21}= 0$\\
$T_{28,22}= 0$&
$T_{28,23}= 0$&
$T_{28,24}= 0$\\
$T_{28,25}= 0$&
$T_{28,26}= 0$&
$T_{28,27}= 0$\\
$T_{28,28}= y$&
$T_{28,29}= 0$&
$T_{28,30}= 0$\\
$T_{28,31}= 0$&
$T_{28,32}= 0$&
$T_{28,33}= 0$\\
$T_{28,34}= 0$&
$T_{28,35}= 0$&
$T_{28,36}= 0$\\
$T_{28,37}= 0$&
$T_{28,38}= 0$&
$T_{28,39}= 0$\\
$T_{28,40}= 0$&
$T_{29,1}= 0$&
$T_{29,2}= 0$\\
$T_{29,3}= 0$&
$T_{29,4}= 0$&
$T_{29,5}= 0$\\
$T_{29,6}= 0$&
$T_{29,7}= 0$&
$T_{29,8}= 0$\\
$T_{29,9}= 0$&
$T_{29,10}= 0$&
$T_{29,11}= 0$\\
$T_{29,12}= 0$&
$T_{29,13}= 0$&
$T_{29,14}= 0$\\
$T_{29,15}= 0$&
$T_{29,16}= 0$&
$T_{29,17}= 0$\\
$T_{29,18}= 0$&
$T_{29,19}= 0$&
$T_{29,20}= 0$\\
$T_{29,21}= 0$&
$T_{29,22}= 0$&
$T_{29,23}= 0$\\
$T_{29,24}= 0$&
$T_{29,25}= 0$&
$T_{29,26}= 0$\\
$T_{29,27}= 0$&
$T_{29,28}= 0$&
$T_{29,29}= 0$\\
$T_{29,30}= wx^2yz^2$&
$T_{29,31}= wx^4y\zeta$&
$T_{29,32}= 0$\\
$T_{29,33}= 0$&
$T_{29,34}= w^2x^4yz^2$&
$T_{29,35}= 0$\\
$T_{29,36}= 0$&
$T_{29,37}= y$&
$T_{29,38}= wx^4y\zeta$\\
$T_{29,39}= 0$&
$T_{29,40}= y$&
$T_{30,1}= 0$\\
$T_{30,2}= 0$&
$T_{30,3}= 0$&
$T_{30,4}= 0$\\
$T_{30,5}= 0$&
$T_{30,6}= 0$&
$T_{30,7}= 0$\\
$T_{30,8}= 0$&
$T_{30,9}= 0$&
$T_{30,10}= 0$\\
$T_{30,11}= 0$&
$T_{30,12}= 0$&
$T_{30,13}= 0$\\
$T_{30,14}= 0$&
$T_{30,15}= 0$&
$T_{30,16}= 0$\\
$T_{30,17}= 0$&
$T_{30,18}= 0$&
$T_{30,19}= 0$\\
$T_{30,20}= 0$&
$T_{30,21}= 0$&
$T_{30,22}= 0$\\
$T_{30,23}= 0$&
$T_{30,24}= 0$&
$T_{30,25}= 0$\\
$T_{30,26}= 0$&
$T_{30,27}= 0$&
$T_{30,28}= 0$\\
$T_{30,29}= 0$&
$T_{30,30}= wx^2y$&
$T_{30,31}= wx^4y\zeta$\\
$T_{30,32}= 0$&
$T_{30,33}= 0$&
$T_{30,34}= w^2x^4y$\\
$T_{30,35}= 0$&
$T_{30,36}= 0$&
$T_{30,37}= y$\\
$T_{30,38}= wx^4y\zeta$&
$T_{30,39}= 0$&
$T_{30,40}= y$\\
$T_{31,1}= 0$&
$T_{31,2}= 0$&
$T_{31,3}= 0$\\
$T_{31,4}= 0$&
$T_{31,5}= 0$&
$T_{31,6}= 0$\\
$T_{31,7}= 0$&
$T_{31,8}= 0$&
$T_{31,9}= 0$\\
$T_{31,10}= 0$&
$T_{31,11}= 0$&
$T_{31,12}= 0$\\
$T_{31,13}= 0$&
$T_{31,14}= 0$&
$T_{31,15}= 0$\\
$T_{31,16}= 0$&
$T_{31,17}= 0$&
$T_{31,18}= 0$\\
$T_{31,19}= 0$&
$T_{31,20}= 0$&
$T_{31,21}= 0$\\
$T_{31,22}= 0$&
$T_{31,23}= 0$&
$T_{31,24}= 0$\\
$T_{31,25}= 0$&
$T_{31,26}= 0$&
$T_{31,27}= 0$\\
$T_{31,28}= 0$&
$T_{31,29}= wx^4y\zeta$&
$T_{31,30}= 0$\\
$T_{31,31}= 0$&
$T_{31,32}= wx^2yz^2$&
$T_{31,33}= 0$\\
$T_{31,34}= 0$&
$T_{31,35}= w^2x^4yz^2$&
$T_{31,36}= 0$\\
$T_{31,37}= y$&
$T_{31,38}= wx^4y\zeta$&
$T_{31,39}= 0$\\
$T_{31,40}= y$&
$T_{32,1}= 0$&
$T_{32,2}= 0$\\
$T_{32,3}= 0$&
$T_{32,4}= 0$&
$T_{32,5}= 0$\\
$T_{32,6}= 0$&
$T_{32,7}= 0$&
$T_{32,8}= 0$\\
$T_{32,9}= 0$&
$T_{32,10}= 0$&
$T_{32,11}= 0$\\
$T_{32,12}= 0$&
$T_{32,13}= 0$&
$T_{32,14}= 0$\\
$T_{32,15}= 0$&
$T_{32,16}= 0$&
$T_{32,17}= 0$\\
$T_{32,18}= 0$&
$T_{32,19}= 0$&
$T_{32,20}= 0$\\
$T_{32,21}= 0$&
$T_{32,22}= 0$&
$T_{32,23}= 0$\\
$T_{32,24}= 0$&
$T_{32,25}= 0$&
$T_{32,26}= 0$\\
$T_{32,27}= 0$&
$T_{32,28}= 0$&
$T_{32,29}= wx^4y\zeta$\\
$T_{32,30}= 0$&
$T_{32,31}= 0$&
$T_{32,32}= wx^2y$\\
$T_{32,33}= 0$&
$T_{32,34}= 0$&
$T_{32,35}= w^2x^4y$\\
$T_{32,36}= 0$&
$T_{32,37}= y$&
$T_{32,38}= wx^4y\zeta$\\
$T_{32,39}= 0$&
$T_{32,40}= y$&
$T_{33,1}= 0$\\
$T_{33,2}= 0$&
$T_{33,3}= 0$&
$T_{33,4}= 0$\\
$T_{33,5}= 0$&
$T_{33,6}= 0$&
$T_{33,7}= 0$\\
$T_{33,8}= 0$&
$T_{33,9}= 0$&
$T_{33,10}= 0$\\
$T_{33,11}= 0$&
$T_{33,12}= 0$&
$T_{33,13}= 0$\\
$T_{33,14}= 0$&
$T_{33,15}= 0$&
$T_{33,16}= 0$\\
$T_{33,17}= 0$&
$T_{33,18}= 0$&
$T_{33,19}= 0$\\
$T_{33,20}= 0$&
$T_{33,21}= 0$&
$T_{33,22}= 0$\\
$T_{33,23}= 0$&
$T_{33,24}= 0$&
$T_{33,25}= 0$\\
$T_{33,26}= 0$&
$T_{33,27}= 0$&
$T_{33,28}= 0$\\
$T_{33,29}= 0$&
$T_{33,30}= wx^2y$&
$T_{33,31}= 0$\\
$T_{33,32}= wx^2y$&
$T_{33,33}= 0$&
$T_{33,34}= 0$\\
$T_{33,35}= 0$&
$T_{33,36}= w^2x^2yz^{-2}$&
$T_{33,37}= y$\\
$T_{33,38}= 0$&
$T_{33,39}= 2wx^2y$&
$T_{33,40}= 2y$\\
$T_{34,1}= 0$&
$T_{34,2}= 0$&
$T_{34,3}= 0$\\
$T_{34,4}= 0$&
$T_{34,5}= 0$&
$T_{34,6}= 0$\\
$T_{34,7}= 0$&
$T_{34,8}= 0$&
$T_{34,9}= 0$\\
$T_{34,10}= 0$&
$T_{34,11}= 0$&
$T_{34,12}= 0$\\
$T_{34,13}= 0$&
$T_{34,14}= 0$&
$T_{34,15}= 0$\\
$T_{34,16}= 0$&
$T_{34,17}= 0$&
$T_{34,18}= 0$\\
$T_{34,19}= 0$&
$T_{34,20}= 0$&
$T_{34,21}= 0$\\
$T_{34,22}= 0$&
$T_{34,23}= 0$&
$T_{34,24}= 0$\\
$T_{34,25}= 0$&
$T_{34,26}= 0$&
$T_{34,27}= 0$\\
$T_{34,28}= 0$&
$T_{34,29}= 0$&
$T_{34,30}= wx^2yz$\\
$T_{34,31}= 0$&
$T_{34,32}= wx^2y$&
$T_{34,33}= 0$\\
$T_{34,34}= 0$&
$T_{34,35}= 0$&
$T_{34,36}= w^2x^2yz^{-1}$\\
$T_{34,37}= y$&
$T_{34,38}= 0$&
$T_{34,39}= wx^2yz+wx^2y$\\
$T_{34,40}= 2y$&
$T_{35,1}= 0$&
$T_{35,2}= 0$\\
$T_{35,3}= 0$&
$T_{35,4}= 0$&
$T_{35,5}= 0$\\
$T_{35,6}= 0$&
$T_{35,7}= 0$&
$T_{35,8}= 0$\\
$T_{35,9}= 0$&
$T_{35,10}= 0$&
$T_{35,11}= 0$\\
$T_{35,12}= 0$&
$T_{35,13}= 0$&
$T_{35,14}= 0$\\
$T_{35,15}= 0$&
$T_{35,16}= 0$&
$T_{35,17}= 0$\\
$T_{35,18}= 0$&
$T_{35,19}= 0$&
$T_{35,20}= 0$\\
$T_{35,21}= 0$&
$T_{35,22}= 0$&
$T_{35,23}= 0$\\
$T_{35,24}= 0$&
$T_{35,25}= 0$&
$T_{35,26}= 0$\\
$T_{35,27}= 0$&
$T_{35,28}= 0$&
$T_{35,29}= 0$\\
$T_{35,30}= wx^2y$&
$T_{35,31}= 0$&
$T_{35,32}= wx^2yz$\\
$T_{35,33}= 0$&
$T_{35,34}= 0$&
$T_{35,35}= 0$\\
$T_{35,36}= w^2x^2yz^{-1}$&
$T_{35,37}= y$&
$T_{35,38}= 0$\\
$T_{35,39}= wx^2y+wx^2yz$&
$T_{35,40}= 2y$&
$T_{36,1}= 0$\\
$T_{36,2}= 0$&
$T_{36,3}= 0$&
$T_{36,4}= 0$\\
$T_{36,5}= 0$&
$T_{36,6}= 0$&
$T_{36,7}= 0$\\
$T_{36,8}= 0$&
$T_{36,9}= 0$&
$T_{36,10}= 0$\\
$T_{36,11}= 0$&
$T_{36,12}= 0$&
$T_{36,13}= 0$\\
$T_{36,14}= 0$&
$T_{36,15}= 0$&
$T_{36,16}= 0$\\
$T_{36,17}= 0$&
$T_{36,18}= 0$&
$T_{36,19}= 0$\\
$T_{36,20}= 0$&
$T_{36,21}= 0$&
$T_{36,22}= 0$\\
$T_{36,23}= 0$&
$T_{36,24}= 0$&
$T_{36,25}= 0$\\
$T_{36,26}= 0$&
$T_{36,27}= 0$&
$T_{36,28}= 0$\\
$T_{36,29}= 0$&
$T_{36,30}= wx^2yz$&
$T_{36,31}= 0$\\
$T_{36,32}= wx^2yz$&
$T_{36,33}= 0$&
$T_{36,34}= 0$\\
$T_{36,35}= 0$&
$T_{36,36}= w^2x^2y$&
$T_{36,37}= y$\\
$T_{36,38}= 0$&
$T_{36,39}= 2wx^2yz$&
$T_{36,40}= 2y$\\
$T_{37,1}= 0$&
$T_{37,2}= 0$&
$T_{37,3}= 0$\\
$T_{37,4}= 0$&
$T_{37,5}= 0$&
$T_{37,6}= 0$\\
$T_{37,7}= 0$&
$T_{37,8}= 0$&
$T_{37,9}= 0$\\
$T_{37,10}= 0$&
$T_{37,11}= 0$&
$T_{37,12}= 0$\\
$T_{37,13}= 0$&
$T_{37,14}= 0$&
$T_{37,15}= 0$\\
$T_{37,16}= 0$&
$T_{37,17}= 0$&
$T_{37,18}= 0$\\
$T_{37,19}= 0$&
$T_{37,20}= 0$&
$T_{37,21}= 0$\\
$T_{37,22}= 0$&
$T_{37,23}= 0$&
$T_{37,24}= 0$\\
$T_{37,25}= 0$&
$T_{37,26}= 0$&
$T_{37,27}= 0$\\
$T_{37,28}= 0$&
$T_{37,29}= wx^4y\zeta$&
$T_{37,30}= 0$\\
$T_{37,31}= wx^4y\zeta$&
$T_{37,32}= 0$&
$T_{37,33}= w^2x^6yz^2\zeta$\\
$T_{37,34}= 0$&
$T_{37,35}= 0$&
$T_{37,36}= 0$\\
$T_{37,37}= y$&
$T_{37,38}= 0$&
$T_{37,39}= 0$\\
$T_{37,40}= 0$&
$T_{38,1}= 0$&
$T_{38,2}= 0$\\
$T_{38,3}= 0$&
$T_{38,4}= 0$&
$T_{38,5}= 0$\\
$T_{38,6}= 0$&
$T_{38,7}= 0$&
$T_{38,8}= 0$\\
$T_{38,9}= 0$&
$T_{38,10}= 0$&
$T_{38,11}= 0$\\
$T_{38,12}= 0$&
$T_{38,13}= 0$&
$T_{38,14}= 0$\\
$T_{38,15}= 0$&
$T_{38,16}= 0$&
$T_{38,17}= 0$\\
$T_{38,18}= 0$&
$T_{38,19}= 0$&
$T_{38,20}= 0$\\
$T_{38,21}= 0$&
$T_{38,22}= 0$&
$T_{38,23}= 0$\\
$T_{38,24}= 0$&
$T_{38,25}= 0$&
$T_{38,26}= 0$\\
$T_{38,27}= 0$&
$T_{38,28}= 0$&
$T_{38,29}= 0$\\
$T_{38,30}= 0$&
$T_{38,31}= 0$&
$T_{38,32}= 0$\\
$T_{38,33}= 0$&
$T_{38,34}= 0$&
$T_{38,35}= 0$\\
$T_{38,36}= 0$&
$T_{38,37}= 0$&
$T_{38,38}= 0$\\
$T_{38,39}= wx^2yz^2$&
$T_{38,40}= y$&
$T_{39,1}= 0$\\
$T_{39,2}= 0$&
$T_{39,3}= 0$&
$T_{39,4}= 0$\\
$T_{39,5}= 0$&
$T_{39,6}= 0$&
$T_{39,7}= 0$\\
$T_{39,8}= 0$&
$T_{39,9}= 0$&
$T_{39,10}= 0$\\
$T_{39,11}= 0$&
$T_{39,12}= 0$&
$T_{39,13}= 0$\\
$T_{39,14}= 0$&
$T_{39,15}= 0$&
$T_{39,16}= 0$\\
$T_{39,17}= 0$&
$T_{39,18}= 0$&
$T_{39,19}= 0$\\
$T_{39,20}= 0$&
$T_{39,21}= 0$&
$T_{39,22}= 0$\\
$T_{39,23}= 0$&
$T_{39,24}= 0$&
$T_{39,25}= 0$\\
$T_{39,26}= 0$&
$T_{39,27}= 0$&
$T_{39,28}= 0$\\
$T_{39,29}= 0$&
$T_{39,30}= 0$&
$T_{39,31}= 0$\\
$T_{39,32}= 0$&
$T_{39,33}= 0$&
$T_{39,34}= 0$\\
$T_{39,35}= 0$&
$T_{39,36}= 0$&
$T_{39,37}= 0$\\
$T_{39,38}= 0$&
$T_{39,39}= wx^2y$&
$T_{39,40}= y$\\
$T_{40,1}= 0$&
$T_{40,2}= 0$&
$T_{40,3}= 0$\\
$T_{40,4}= 0$&
$T_{40,5}= 0$&
$T_{40,6}= 0$\\
$T_{40,7}= 0$&
$T_{40,8}= 0$&
$T_{40,9}= 0$\\
$T_{40,10}= 0$&
$T_{40,11}= 0$&
$T_{40,12}= 0$\\
$T_{40,13}= 0$&
$T_{40,14}= 0$&
$T_{40,15}= 0$\\
$T_{40,16}= 0$&
$T_{40,17}= 0$&
$T_{40,18}= 0$\\
$T_{40,19}= 0$&
$T_{40,20}= 0$&
$T_{40,21}= 0$\\
$T_{40,22}= 0$&
$T_{40,23}= 0$&
$T_{40,24}= 0$\\
$T_{40,25}= 0$&
$T_{40,26}= 0$&
$T_{40,27}= 0$\\
$T_{40,28}= 0$&
$T_{40,29}= 0$&
$T_{40,30}= 0$\\
$T_{40,31}= 0$&
$T_{40,32}= 0$&
$T_{40,33}= 0$\\
$T_{40,34}= 0$&
$T_{40,35}= 0$&
$T_{40,36}= 0$\\
$T_{40,37}= 0$&
$T_{40,38}= wx^4y\zeta$&
$T_{40,39}= 0$\\
$T_{40,40}= y$&\\
\hline
\caption{\label{tab2} Matrice $\mathcal{M}_{F3}$.}
\end{longtable} 
\normalsize
 \begin{spacing}{0.30}
\subsection*{Matrice $\mathcal{M}_{3}$}
 \end{spacing}
\addcontentsline{toc}{section}{Matrice $\mathcal{M}_{3}$}
En suivant l'ordre de numérotation, dans le tableau \ref{tab6},  des états de $\mathcal{A}_{3}$ nous avons la matrice $\mathcal{M}_{3}$ telle que présentée dans le tableau \ref{tab7}.
\small
\begin{longtable}{|c|c|c|} 
\hline
$e_{1}=(100,000,\{\{0\}\})$&
$e_{2}=(100,100,\{\{0\}\})$&
$e_{3}=(010,000,\{\{0\}\})$\\
$e_{4}=(010,010)\{\{0\}\})$&
$e_{5}=(001,000,\{\{0\}\}$&
$e_{6}=(001,001,\{\{0\}\})$\\
$e_{7}=(110,110,\{\{0\}\})$&
$e_{8}=(110,210,\{\{0\}\})$&
$e_{9}=(110,120,\{\{0\}\})$\\
$e_{10}=(110,220,\{\{0\})$&
$e_{11}=(101,000,\{\{0\},\{1\}\})$& \\
$e_{12}=(101,101,\{\{0\},\{1\}\})$ &
$e_{13}=(101,100,\{\{0\},\{1\}\})$& \\
$e_{14}=(101,001,\{\{0\},\{1\})$&
$e_{15}=(101,101,\{\{0,1\}\})$ &\\
$e_{16}=(011,011,\{\{0\}\})$&
$e_{17}=(011,021,\{\{0\}\})$&
$e_{18}=(011,012,\{\{0\}\})$\\
$e_{19}=(011,022,\{\{0\}\})$&
$e_{20}=(111,121,\{\{0\}\})$&
$e_{21}=(111,221,\{\{0\}\})$\\
$e_{22}=(111,122,\{\{0\}\})$&
$e_{23}=(111,222,\{\{0\}\})$&
$e_{24}=(111,131,\{\{0\}\})$\\
$e_{25}=(111,231,\{\{0\}\})$&
$e_{26}=(111,132,\{\{0\}\})$&
$e_{27}=(111,232,\{\{0\}\})$\}\\
&
$e_{28}=(10,00,\{\{0\}\})$&
$e_{29}=(10,10,\{\{0\}\})$\\
$e_{30}=(01,00,\{\{0\}\})$&
$e_{31}=(01,01,\{\{0\}\})$&
$e_{32}=(11,11,\{\{0\}\})$\\
$e_{33}=(11,21,\{\{0\}\})$&
$e_{34}=(11,12,\{\{0\}\})$&
$e_{35}=(11,22,\{\{0\}\})$\\
$e_{36}=(1,0,\{\{0\}\})$ &
$e_{37}=(1,1,\{\{0\}\})$ & \\
\hline
\caption{\label{tab6} Les états de $\mathcal{A}_{3}$.}
\end{longtable}
\normalsize
\tiny
\begin{longtable}{|c|c|c|} 
\hline
$T_{1,1}= 0$&

$T_{1,2}= yz^2wx^2$&

$T_{1,3}= 0$\\

$T_{1,4}= 0$&

$T_{1,5}= 0$&

$T_{1,6}= 0$\\

$T_{1,7}= 0$&

$T_{1,8}= yz^2w^2x^4$&

$T_{1,9}= 0$\\

$T_{1,10}= 0$&

$T_{1,11}= 0$&

$T_{1,12}= 0$\\

$T_{1,13}= yz^2w^2x^6$&

$T_{1,14}= 0$&

$T_{1,15}= 0$\\

$T_{1,16}= 0$&

$T_{1,17}= 0$&

$T_{1,18}= 0$\\

$T_{1,19}= 0$&

$T_{1,20}= 0$&

$T_{1,21}= yz^2w^3x^6$\\

$T_{1,22}= 0$&

$T_{1,23}= 0$&

$T_{1,24}= 0$\\

$T_{1,25}= 0$&

$T_{1,26}= 0$&

$T_{1,27}= 0$\\

$T_{1,28}= 0$&

$T_{1,29}= 0$&

$T_{1,30}= 0$\\

$T_{1,31}= 0$&

$T_{1,32}= 0$&

$T_{1,33}= 0$\\

$T_{1,34}= 0$&

$T_{1,35}= 0$&

$T_{1,36}= 0$\\

$T_{1,37}= 0$&

$T_{2,1}= 0$&

$T_{2,2}= ywx^2$\\

$T_{2,3}= 0$&

$T_{2,4}= 0$&

$T_{2,5}= 0$\\

$T_{2,6}= 0$&

$T_{2,7}= 0$&

$T_{2,8}= yw^2x^4$\\

$T_{2,9}= 0$&

$T_{2,10}= 0$&

$T_{2,11}= 0$\\

$T_{2,12}= 0$&

$T_{2,13}= yw^2x^6$&

$T_{2,14}= 0$\\

$T_{2,15}= 0$&

$T_{2,16}= 0$&

$T_{2,17}= 0$\\

$T_{2,18}= 0$&

$T_{2,19}= 0$&

$T_{2,20}= 0$\\

$T_{2,21}= yw^3x^6$&

$T_{2,22}= 0$&

$T_{2,23}= 0$\\

$T_{2,24}= 0$&

$T_{2,25}= 0$&

$T_{2,26}= 0$\\

$T_{2,27}= 0$&

$T_{2,28}= 0$&

$T_{2,29}= 0$\\

$T_{2,30}= 0$&

$T_{2,31}= 0$&

$T_{2,32}= 0$\\

$T_{2,33}= 0$&

$T_{2,34}= 0$&

$T_{2,35}= 0$\\

$T_{2,36}= 0$&

$T_{2,37}= 0$&

$T_{3,1}= 0$\\

$T_{3,2}= 0$&

$T_{3,3}= 0$&

$T_{3,4}= yz^2wx^2$\\

$T_{3,5}= 0$&

$T_{3,6}= 0$&

$T_{3,7}= 0$\\

$T_{3,8}= 0$&

$T_{3,9}= yz^2w^2x^4$&

$T_{3,10}= 0$\\

$T_{3,11}= 0$&

$T_{3,12}= 0$&

$T_{3,13}= 0$\\

$T_{3,14}= 0$&

$T_{3,15}= 0$&

$T_{3,16}= 0$\\

$T_{3,17}= yz^2w^2x^4$&

$T_{3,18}= 0$&

$T_{3,19}= 0$\\

$T_{3,20}= 0$&

$T_{3,21}= 0$&

$T_{3,22}= 0$\\

$T_{3,23}= 0$&

$T_{3,24}= yz^3w^3x^6$&

$T_{3,25}= 0$\\

$T_{3,26}= 0$&

$T_{3,27}= 0$&

$T_{3,28}= 0$\\

$T_{3,29}= 0$&

$T_{3,30}= 0$&

$T_{3,31}= 0$\\

$T_{3,32}= 0$&

$T_{3,33}= 0$&

$T_{3,34}= 0$\\

$T_{3,35}= 0$&

$T_{3,36}= 0$&

$T_{3,37}= 0$\\

$T_{4,1}= 0$&

$T_{4,2}= 0$&

$T_{4,3}= 0$\\

$T_{4,4}= ywx^2$&

$T_{4,5}= 0$&

$T_{4,6}= 0$\\

$T_{4,7}= 0$&

$T_{4,8}= 0$&

$T_{4,9}= yw^2x^4$\\

$T_{4,10}= 0$&

$T_{4,11}= 0$&

$T_{4,12}= 0$\\

$T_{4,13}= 0$&

$T_{4,14}= 0$&

$T_{4,15}= 0$\\

$T_{4,16}= 0$&

$T_{4,17}= yw^2x^4$&

$T_{4,18}= 0$\\

$T_{4,19}= 0$&

$T_{4,20}= 0$&

$T_{4,21}= 0$\\

$T_{4,22}= 0$&

$T_{4,23}= 0$&

$T_{4,24}= yzw^3x^6$\\

$T_{4,25}= 0$&

$T_{4,26}= 0$&

$T_{4,27}= 0$\\

$T_{4,28}= 0$&

$T_{4,29}= 0$&

$T_{4,30}= 0$\\

$T_{4,31}= 0$&

$T_{4,32}= 0$&

$T_{4,33}= 0$\\

$T_{4,34}= 0$&

$T_{4,35}= 0$&

$T_{4,36}= 0$\\

$T_{4,37}= 0$&

$T_{5,1}= 0$&

$T_{5,2}= 0$\\

$T_{5,3}= 0$&

$T_{5,4}= 0$&

$T_{5,5}= 0$\\

$T_{5,6}= yz^2wx^2$&

$T_{5,7}= 0$&

$T_{5,8}= 0$\\

$T_{5,9}= 0$&

$T_{5,10}= 0$&

$T_{5,11}= 0$\\

$T_{5,12}= 0$&

$T_{5,13}= 0$&

$T_{5,14}= yz^2w^2x^6$\\

$T_{5,15}= 0$&

$T_{5,16}= 0$&

$T_{5,17}= 0$\\

$T_{5,18}= yz^2w^2x^4$&

$T_{5,19}= 0$&

$T_{5,20}= 0$\\

$T_{5,21}= 0$&

$T_{5,22}= yz^2w^3x^6$&

$T_{5,23}= 0$\\

$T_{5,24}= 0$&

$T_{5,25}= 0$&

$T_{5,26}= 0$\\

$T_{5,27}= 0$&

$T_{5,28}= 0$&

$T_{5,29}= 0$\\

$T_{5,30}= 0$&

$T_{5,31}= 0$&

$T_{5,32}= 0$\\

$T_{5,33}= 0$&

$T_{5,34}= 0$&

$T_{5,35}= 0$\\

$T_{5,36}= 0$&

$T_{5,37}= 0$&

$T_{6,1}= 0$\\

$T_{6,2}= 0$&

$T_{6,3}= 0$&

$T_{6,4}= 0$\\

$T_{6,5}= 0$&

$T_{6,6}= ywx^2$&

$T_{6,7}= 0$\\

$T_{6,8}= 0$&

$T_{6,9}= 0$&

$T_{6,10}= 0$\\

$T_{6,11}= 0$&

$T_{6,12}= 0$&

$T_{6,13}= 0$\\

$T_{6,14}= yw^2x^6$&

$T_{6,15}= 0$&

$T_{6,16}= 0$\\

$T_{6,17}= 0$&

$T_{6,18}= yw^2x^4$&

$T_{6,19}= 0$\\

$T_{6,20}= 0$&

$T_{6,21}= 0$&

$T_{6,22}= yw^3x^6$\\

$T_{6,23}= 0$&

$T_{6,24}= 0$&

$T_{6,25}= 0$\\

$T_{6,26}= 0$&

$T_{6,27}= 0$&

$T_{6,28}= 0$\\

$T_{6,29}= 0$&

$T_{6,30}= 0$&

$T_{6,31}= 0$\\

$T_{6,32}= 0$&

$T_{6,33}= 0$&

$T_{6,34}= 0$\\

$T_{6,35}= 0$&

$T_{6,36}= 0$&

$T_{6,37}= 0$\\

$T_{7,1}= 0$&

$T_{7,2}= ywx^2$&

$T_{7,3}= 0$\\

$T_{7,4}= ywx^2$&

$T_{7,5}= 0$&

$T_{7,6}= 0$\\

$T_{7,7}= 0$&

$T_{7,8}= 0$&

$T_{7,9}= 0$\\

$T_{7,10}= yz^{-2}w^2x^2$&

$T_{7,11}= 0$&

$T_{7,12}= 0$\\

$T_{7,13}= yw^2x^6$&

$T_{7,14}= 0$&

$T_{7,15}= 0$\\

$T_{7,16}= 0$&

$T_{7,17}= yw^2x^4$&

$T_{7,18}= 0$\\

$T_{7,19}= 0$&

$T_{7,20}= 0$&

$T_{7,21}= 0$\\

$T_{7,22}= 0$&

$T_{7,23}= 0$&

$T_{7,24}= 0$\\

$T_{7,25}= yz^{-1}w^3x^4$&

$T_{7,26}= 0$&

$T_{7,27}= 0$\\

$T_{7,28}= 0$&

$T_{7,29}= ywx^2$&

$T_{7,30}= 0$\\

$T_{7,31}= 0$&

$T_{7,32}= 0$&

$T_{7,33}= yw^2x^4$\\

$T_{7,34}= 0$&

$T_{7,35}= 0$&

$T_{7,36}= 0$\\

$T_{7,37}= 0$&

$T_{8,1}= 0$&

$T_{8,2}= yzwx^2$\\

$T_{8,3}= 0$&

$T_{8,4}= ywx^2$&

$T_{8,5}= 0$\\

$T_{8,6}= 0$&

$T_{8,7}= 0$&

$T_{8,8}= 0$\\

$T_{8,9}= 0$&

$T_{8,10}= yz^{-1}w^2x^2$&

$T_{8,11}= 0$\\

$T_{8,12}= 0$&

$T_{8,13}= yzw^2x^6$&

$T_{8,14}= 0$\\

$T_{8,15}= 0$&

$T_{8,16}= 0$&

$T_{8,17}= yw^2x^4$\\

$T_{8,18}= 0$&

$T_{8,19}= 0$&

$T_{8,20}= 0$\\

$T_{8,21}= 0$&

$T_{8,22}= 0$&

$T_{8,23}= 0$\\

$T_{8,24}= 0$&

$T_{8,25}= yw^3x^4$&

$T_{8,26}= 0$\\

$T_{8,27}= 0$&

$T_{8,28}= 0$&

$T_{8,29}= ywx^2$\\

$T_{8,30}= 0$&

$T_{8,31}= 0$&

$T_{8,32}= 0$\\

$T_{8,33}= yw^2x^4$&

$T_{8,34}= 0$&

$T_{8,35}= 0$\\

$T_{8,36}= 0$&

$T_{8,37}= 0$&

$T_{9,1}= 0$\\

$T_{9,2}= ywx^2$&

$T_{9,3}= 0$&

$T_{9,4}= yzwx^2$\\

$T_{9,5}= 0$&

$T_{9,6}= 0$&

$T_{9,7}= 0$\\

$T_{9,8}= 0$&

$T_{9,9}= 0$&

$T_{9,10}= yz^{-1}w^2x^2$\\

$T_{9,11}= 0$&

$T_{9,12}= 0$&

$T_{9,13}= yw^2x^6$\\

$T_{9,14}= 0$&

$T_{9,15}= 0$&

$T_{9,16}= 0$\\

$T_{9,17}= yzw^2x^4$&

$T_{9,18}= 0$&

$T_{9,19}= 0$\\

$T_{9,20}= 0$&

$T_{9,21}= 0$&

$T_{9,22}= 0$\\

$T_{9,23}= 0$&

$T_{9,24}= 0$&

$T_{9,25}= yw^3x^4$\\

$T_{9,26}= 0$&

$T_{9,27}= 0$&

$T_{9,28}= 0$\\

$T_{9,29}= yzwx^2$&

$T_{9,30}= 0$&

$T_{9,31}= 0$\\

$T_{9,32}= 0$&

$T_{9,33}= yzw^2x^4$&

$T_{9,34}= 0$\\

$T_{9,35}= 0$&

$T_{9,36}= 0$&

$T_{9,37}= 0$\\

$T_{10,1}= 0$&

$T_{10,2}= yzwx^2$&

$T_{10,3}= 0$\\

$T_{10,4}= yzwx^2$&

$T_{10,5}= 0$&

$T_{10,6}= 0$\\

$T_{10,7}= 0$&

$T_{10,8}= 0$&

$T_{10,9}= 0$\\

$T_{10,10}= yw^2x^2$&

$T_{10,11}= 0$&

$T_{10,12}= 0$\\

$T_{10,13}= yzw^2x^6$&

$T_{10,14}= 0$&

$T_{10,15}= 0$\\

$T_{10,16}= 0$&

$T_{10,17}= yzw^2x^4$&

$T_{10,18}= 0$\\

$T_{10,19}= 0$&

$T_{10,20}= 0$&

$T_{10,21}= 0$\\

$T_{10,22}= 0$&

$T_{10,23}= 0$&

$T_{10,24}= 0$\\

$T_{10,25}= yzw^3x^4$&

$T_{10,26}= 0$&

$T_{10,27}= 0$\\

$T_{10,28}= 0$&

$T_{10,29}= yzwx^2$&

$T_{10,30}= 0$\\

$T_{10,31}= 0$&

$T_{10,32}= 0$&

$T_{10,33}= yzw^2x^4$\\

$T_{10,34}= 0$&

$T_{10,35}= 0$&

$T_{10,36}= 0$\\

$T_{10,37}= 0$&

$T_{11,1}= 0$&

$T_{11,2}= 0$\\

$T_{11,3}= 0$&

$T_{11,4}= 0$&

$T_{11,5}= 0$\\

$T_{11,6}= 0$&

$T_{11,7}= 0$&

$T_{11,8}= 0$\\

$T_{11,9}= 0$&

$T_{11,10}= 0$&

$T_{11,11}= 0$\\

$T_{11,12}= yz^4w^2x^4$&

$T_{11,13}= 0$&

$T_{11,14}= 0$\\

$T_{11,15}= 0$&

$T_{11,16}= 0$&

$T_{11,17}= 0$\\

$T_{11,18}= 0$&

$T_{11,19}= 0$&

$T_{11,20}= 0$\\

$T_{11,21}= 0$&

$T_{11,22}= 0$&

$T_{11,23}= yz^2w^3x^4$\\

$T_{11,24}= 0$&

$T_{11,25}= 0$&

$T_{11,26}= 0$\\

$T_{11,27}= 0$&

$T_{11,28}= 0$&

$T_{11,29}= 0$\\

$T_{11,30}= 0$&

$T_{11,31}= 0$&

$T_{11,32}= 0$\\

$T_{11,33}= 0$&

$T_{11,34}= 0$&

$T_{11,35}= 0$\\

$T_{11,36}= 0$&

$T_{11,37}= 0$&

$T_{12,1}= 0$\\

$T_{12,2}= 0$&

$T_{12,3}= 0$&

$T_{12,4}= 0$\\

$T_{12,5}= 0$&

$T_{12,6}= 0$&

$T_{12,7}= 0$\\

$T_{12,8}= 0$&

$T_{12,9}= 0$&

$T_{12,10}= 0$\\

$T_{12,11}= 0$&

$T_{12,12}= yw^2x^4$&

$T_{12,13}= 0$\\

$T_{12,14}= 0$&

$T_{12,15}= 0$&

$T_{12,16}= 0$\\

$T_{12,17}= 0$&

$T_{12,18}= 0$&

$T_{12,19}= 0$\\

$T_{12,20}= 0$&

$T_{12,21}= 0$&

$T_{12,22}= 0$\\

$T_{12,23}= yz^{-2}w^3x^4$&

$T_{12,24}= 0$&

$T_{12,25}= 0$\\

$T_{12,26}= 0$&

$T_{12,27}= 0$&

$T_{12,28}= 0$\\

$T_{12,29}= 0$&

$T_{12,30}= 0$&

$T_{12,31}= 0$\\

$T_{12,32}= 0$&

$T_{12,33}= 0$&

$T_{12,34}= 0$\\

$T_{12,35}= 0$&

$T_{12,36}= 0$&

$T_{12,37}= 0$\\

$T_{13,1}= 0$&

$T_{13,2}= 0$&

$T_{13,3}= 0$\\

$T_{13,4}= 0$&

$T_{13,5}= 0$&

$T_{13,6}= 0$\\

$T_{13,7}= 0$&

$T_{13,8}= 0$&

$T_{13,9}= 0$\\

$T_{13,10}= 0$&

$T_{13,11}= 0$&

$T_{13,12}= yz^2w^2x^4$\\

$T_{13,13}= 0$&

$T_{13,14}= 0$&

$T_{13,15}= 0$\\

$T_{13,16}= 0$&

$T_{13,17}= 0$&

$T_{13,18}= 0$\\

$T_{13,19}= 0$&

$T_{13,20}= 0$&

$T_{13,21}= 0$\\

$T_{13,22}= 0$&

$T_{13,23}= yzw^3x^4$&

$T_{13,24}= 0$\\

$T_{13,25}= 0$&

$T_{13,26}= 0$&

$T_{13,27}= 0$\\

$T_{13,28}= 0$&

$T_{13,29}= 0$&

$T_{13,30}= 0$\\

$T_{13,31}= 0$&

$T_{13,32}= 0$&

$T_{13,33}= 0$\\

$T_{13,34}= 0$&

$T_{13,35}= 0$&

$T_{13,36}= 0$\\

$T_{13,37}= 0$&

$T_{14,1}= 0$&

$T_{14,2}= 0$\\

$T_{14,3}= 0$&

$T_{14,4}= 0$&

$T_{14,5}= 0$\\

$T_{14,6}= 0$&

$T_{14,7}= 0$&

$T_{14,8}= 0$\\

$T_{14,9}= 0$&

$T_{14,10}= 0$&

$T_{14,11}= 0$\\

$T_{14,12}= yz^2w^2x^4$&

$T_{14,13}= 0$&

$T_{14,14}= 0$\\

$T_{14,15}= 0$&

$T_{14,16}= 0$&

$T_{14,17}= 0$\\

$T_{14,18}= 0$&

$T_{14,19}= 0$&

$T_{14,20}= 0$\\

$T_{14,21}= 0$&

$T_{14,22}= 0$&

$T_{14,23}= yw^3x^4$\\

$T_{14,24}= 0$&

$T_{14,25}= 0$&

$T_{14,26}= 0$\\

$T_{14,27}= 0$&

$T_{14,28}= 0$&

$T_{14,29}= 0$\\

$T_{14,30}= 0$&

$T_{14,31}= 0$&

$T_{14,32}= 0$\\

$T_{14,33}= 0$&

$T_{14,34}= 0$&

$T_{14,35}= 0$\\

$T_{14,36}= 0$&

$T_{14,37}= 0$&

$T_{15,1}= 0$\\

$T_{15,2}= ywx^2$&

$T_{15,3}= 0$&

$T_{15,4}= 0$\\

$T_{15,5}= 0$&

$T_{15,6}= ywx^2$&

$T_{15,7}= 0$\\

$T_{15,8}= yw^2x^4$&

$T_{15,9}= 0$&

$T_{15,10}= 0$\\

$T_{15,11}= 0$&

$T_{15,12}= 0$&

$T_{15,13}= 0$\\

$T_{15,14}= 0$&

$T_{15,15}= yw^2x^4$&

$T_{15,16}= 0$\\

$T_{15,17}= 0$&

$T_{15,18}= yw^2x^4$&

$T_{15,19}= 0$\\

$T_{15,20}= 0$&

$T_{15,21}= 0$&

$T_{15,22}= 0$\\

$T_{15,23}= yz^{-2}w^3x^4$&

$T_{15,24}= 0$&

$T_{15,25}= 0$\\

$T_{15,26}= 0$&

$T_{15,27}= 0$&

$T_{15,28}= 0$\\

$T_{15,29}= ywx^2$&

$T_{15,30}= 0$&

$T_{15,31}= ywx^2$\\

$T_{15,32}= 0$&

$T_{15,33}= yw^2x^4$&

$T_{15,34}= yw^2x^4$\\

$T_{15,35}= 0$&

$T_{15,36}= 0$&

$T_{15,37}= 2ywx^2$\\

$T_{16,1}= 0$&

$T_{16,2}= 0$&

$T_{16,3}= 0$\\

$T_{16,4}= ywx^2$&

$T_{16,5}= 0$&

$T_{16,6}= ywx^2$\\

$T_{16,7}= 0$&

$T_{16,8}= 0$&

$T_{16,9}= yw^2x^4$\\

$T_{16,10}= 0$&

$T_{16,11}= 0$&

$T_{16,12}= 0$\\

$T_{16,13}= 0$&

$T_{16,14}= yw^2x^6$&

$T_{16,15}= 0$\\

$T_{16,16}= 0$&

$T_{16,17}= 0$&

$T_{16,18}= 0$\\

$T_{16,19}= yz^{-2}w^2x^2$&

$T_{16,20}= 0$&

$T_{16,21}= 0$\\

$T_{16,22}= 0$&

$T_{16,23}= 0$&

$T_{16,24}= 0$\\

$T_{16,25}= 0$&

$T_{16,26}= yz^{-1}w^3x^4$&

$T_{16,27}= 0$\\

$T_{16,28}= 0$&

$T_{16,29}= 0$&

$T_{16,30}= 0$\\

$T_{16,31}= ywx^2$&

$T_{16,32}= 0$&

$T_{16,33}= 0$\\

$T_{16,34}= yw^2x^4$&

$T_{16,35}= 0$&

$T_{16,36}= 0$\\

$T_{16,37}= 0$&

$T_{17,1}= 0$&

$T_{17,2}= 0$\\

$T_{17,3}= 0$&

$T_{17,4}= yzwx^2$&

$T_{17,5}= 0$\\

$T_{17,6}= ywx^2$&

$T_{17,7}= 0$&

$T_{17,8}= 0$\\

$T_{17,9}= yzw^2x^4$&

$T_{17,10}= 0$&

$T_{17,11}= 0$\\

$T_{17,12}= 0$&

$T_{17,13}= 0$&

$T_{17,14}= yw^2x^6$\\

$T_{17,15}= 0$&

$T_{17,16}= 0$&

$T_{17,17}= 0$\\

$T_{17,18}= 0$&

$T_{17,19}= yz^{-1}w^2x^2$&

$T_{17,20}= 0$\\

$T_{17,21}= 0$&

$T_{17,22}= 0$&

$T_{17,23}= 0$\\

$T_{17,24}= 0$&

$T_{17,25}= 0$&

$T_{17,26}= yw^3x^4$\\

$T_{17,27}= 0$&

$T_{17,28}= 0$&

$T_{17,29}= 0$\\

$T_{17,30}= 0$&

$T_{17,31}= yzwx^2$&

$T_{17,32}= 0$\\

$T_{17,33}= 0$&

$T_{17,34}= yzw^2x^4$&

$T_{17,35}= 0$\\

$T_{17,36}= 0$&

$T_{17,37}= 0$&

$T_{18,1}= 0$\\

$T_{18,2}= 0$&

$T_{18,3}= 0$&

$T_{18,4}= ywx^2$\\

$T_{18,5}= 0$&

$T_{18,6}= yzwx^2$&

$T_{18,7}= 0$\\

$T_{18,8}= 0$&

$T_{18,9}= yw^2x^4$&

$T_{18,10}= 0$\\

$T_{18,11}= 0$&

$T_{18,12}= 0$&

$T_{18,13}= 0$\\

$T_{18,14}= yzw^2x^6$&

$T_{18,15}= 0$&

$T_{18,16}= 0$\\

$T_{18,17}= 0$&

$T_{18,18}= 0$&

$T_{18,19}= yz^{-1}w^2x^2$\\

$T_{18,20}= 0$&

$T_{18,21}= 0$&

$T_{18,22}= 0$\\

$T_{18,23}= 0$&

$T_{18,24}= 0$&

$T_{18,25}= 0$\\

$T_{18,26}= yw^3x^4$&

$T_{18,27}= 0$&

$T_{18,28}= 0$\\

$T_{18,29}= 0$&

$T_{18,30}= 0$&

$T_{18,31}= ywx^2$\\

$T_{18,32}= 0$&

$T_{18,33}= 0$&

$T_{18,34}= yw^2x^4$\\

$T_{18,35}= 0$&

$T_{18,36}= 0$&

$T_{18,37}= 0$\\

$T_{19,1}= 0$&

$T_{19,2}= 0$&

$T_{19,3}= 0$\\

$T_{19,4}= yzwx^2$&

$T_{19,5}= 0$&

$T_{19,6}= yzwx^2$\\

$T_{19,7}= 0$&

$T_{19,8}= 0$&

$T_{19,9}= yzw^2x^4$\\

$T_{19,10}= 0$&

$T_{19,11}= 0$&

$T_{19,12}= 0$\\

$T_{19,13}= 0$&

$T_{19,14}= yzw^2x^6$&

$T_{19,15}= 0$\\

$T_{19,16}= 0$&

$T_{19,17}= 0$&

$T_{19,18}= 0$\\

$T_{19,19}= yw^2x^2$&

$T_{19,20}= 0$&

$T_{19,21}= 0$\\

$T_{19,22}= 0$&

$T_{19,23}= 0$&

$T_{19,24}= 0$\\

$T_{19,25}= 0$&

$T_{19,26}= yzw^3x^4$&

$T_{19,27}= 0$\\

$T_{19,28}= 0$&

$T_{19,29}= 0$&

$T_{19,30}= 0$\\

$T_{19,31}= yzwx^2$&

$T_{19,32}= 0$&

$T_{19,33}= 0$\\

$T_{19,34}= yzw^2x^4$&

$T_{19,35}= 0$&

$T_{19,36}= 0$\\

$T_{19,37}= 0$&

$T_{20,1}= 0$&

$T_{20,2}= ywx^2$\\

$T_{20,3}= 0$&

$T_{20,4}= yzwx^2$&

$T_{20,5}= 0$\\

$T_{20,6}= ywx^2$&

$T_{20,7}= 0$&

$T_{20,8}= 0$\\

$T_{20,9}= 0$&

$T_{20,10}= yz^{-1}w^2x^2$&

$T_{20,11}= 0$\\

$T_{20,12}= 0$&

$T_{20,13}= 0$&

$T_{20,14}= 0$\\

$T_{20,15}= yw^2x^4$&

$T_{20,16}= 0$&

$T_{20,17}= 0$\\

$T_{20,18}= 0$&

$T_{20,19}= yz^{-1}w^2x^2$&

$T_{20,20}= 0$\\

$T_{20,21}= 0$&

$T_{20,22}= 0$&

$T_{20,23}= 0$\\

$T_{20,24}= 0$&

$T_{20,25}= 0$&

$T_{20,26}= 0$\\

$T_{20,27}= yz^{-2}w^3x^2$&

$T_{20,28}= 0$&

$T_{20,29}= yzwx^2+ywx^2$\\

$T_{20,30}= 0$&

$T_{20,31}= yzwx^2+ywx^2$&

$T_{20,32}= 0$\\

$T_{20,33}= 0$&

$T_{20,34}= 0$&

$T_{20,35}= 2yz^{-1}w^2x^2$\\

$T_{20,36}= 0$&

$T_{20,37}= 2yw^2x^2+yz^{-1}w^2x^2$&

$T_{21,1}= 0$\\

$T_{21,2}= yzwx^2$&

$T_{21,3}= 0$&

$T_{21,4}= yzwx^2$\\

$T_{21,5}= 0$&

$T_{21,6}= ywx^2$&

$T_{21,7}= 0$\\

$T_{21,8}= 0$&

$T_{21,9}= 0$&

$T_{21,10}= yw^2x^2$\\

$T_{21,11}= 0$&

$T_{21,12}= 0$&

$T_{21,13}= 0$\\

$T_{21,14}= 0$&

$T_{21,15}= yzw^2x^4$&

$T_{21,16}= 0$\\

$T_{21,17}= 0$&

$T_{21,18}= 0$&

$T_{21,19}= yz^{-1}w^2x^2$\\

$T_{21,20}= 0$&

$T_{21,21}= 0$&

$T_{21,22}= 0$\\

$T_{21,23}= 0$&

$T_{21,24}= 0$&

$T_{21,25}= 0$\\

$T_{21,26}= 0$&

$T_{21,27}= yz^{-1}w^3x^2$&

$T_{21,28}= 0$\\

$T_{21,29}= 2yzwx^2$&

$T_{21,30}= 0$&

$T_{21,31}= yzwx^2+ywx^2$\\

$T_{21,32}= 0$&

$T_{21,33}= 0$&

$T_{21,34}= 0$\\

$T_{21,35}= yw^2x^2+yz^{-1}w^2x^2$&

$T_{21,36}= 0$&

$T_{21,37}= 2yzwx^2+ywx^2$\\

$T_{22,1}= 0$&

$T_{22,2}= ywx^2$&

$T_{22,3}= 0$\\

$T_{22,4}= yzwx^2$&

$T_{22,5}= 0$&

$T_{22,6}= yzwx^2$\\

$T_{22,7}= 0$&

$T_{22,8}= 0$&

$T_{22,9}= 0$\\

$T_{22,10}= yz^{-1}w^2x^2$&

$T_{22,11}= 0$&

$T_{22,12}= 0$\\

$T_{22,13}= 0$&

$T_{22,14}= 0$&

$T_{22,15}= yzw^2x^4$\\

$T_{22,16}= 0$&

$T_{22,17}= 0$&

$T_{22,18}= 0$\\

$T_{22,19}= yw^2x^2$&

$T_{22,20}= 0$&

$T_{22,21}= 0$\\

$T_{22,22}= 0$&

$T_{22,23}= 0$&

$T_{22,24}= 0$\\

$T_{22,25}= 0$&

$T_{22,26}= 0$&

$T_{22,27}= yz^{-1}w^3x^2$\\

$T_{22,28}= 0$&

$T_{22,29}= yzwx^2+ywx^2$&

$T_{22,30}= 0$\\

$T_{22,31}= 2yzwx^2$&

$T_{22,32}= 0$&

$T_{22,33}= 0$\\

$T_{22,34}= 0$&

$T_{22,35}= yw^2x^2+yz^{-1}w^2x^2$&

$T_{22,36}= 0$\\

$T_{22,37}= 2yzwx^2+ywx^2$&

$T_{23,1}= 0$&

$T_{23,2}= yzwx^2$\\

$T_{23,3}= 0$&

$T_{23,4}= yzwx^2$&

$T_{23,5}= 0$\\

$T_{23,6}= yzwx^2$&

$T_{23,7}= 0$&

$T_{23,8}= 0$\\

$T_{23,9}= 0$&

$T_{23,10}= yw^2x^2$&

$T_{23,11}= 0$\\

$T_{23,12}= 0$&

$T_{23,13}= 0$&

$T_{23,14}= 0$\\

$T_{23,15}= yz^2w^2x^4$&

$T_{23,16}= 0$&

$T_{23,17}= 0$\\

$T_{23,18}= 0$&

$T_{23,19}= yw^2x^2$&

$T_{23,20}= 0$\\

$T_{23,21}= 0$&

$T_{23,22}= 0$&

$T_{23,23}= 0$\\

$T_{23,24}= 0$&

$T_{23,25}= 0$&

$T_{23,26}= 0$\\

$T_{23,27}= yw^3x^2$&

$T_{23,28}= 0$&

$T_{23,29}= 2yzwx^2$\\

$T_{23,30}= 0$&

$T_{23,31}= 2yzwx^2$&

$T_{23,32}= 0$\\

$T_{23,33}= 0$&

$T_{23,34}= 0$&

$T_{23,35}= 2yw^2x^2$\\

$T_{23,36}= 0$&

$T_{23,37}= 3yzwx^2$&

$T_{24,1}= 0$\\

$T_{24,2}= ywx^2$&

$T_{24,3}= 0$&

$T_{24,4}= yzwx^2$\\

$T_{24,5}= 0$&

$T_{24,6}= ywx^2$&

$T_{24,7}= 0$\\

$T_{24,8}= 0$&

$T_{24,9}= 0$&

$T_{24,10}= yz^{-1}w^2x^2$\\

$T_{24,11}= 0$&

$T_{24,12}= 0$&

$T_{24,13}= 0$\\

$T_{24,14}= 0$&

$T_{24,15}= yw^2x^4$&

$T_{24,16}= 0$\\

$T_{24,17}= 0$&

$T_{24,18}= 0$&

$T_{24,19}= yz^{-1}w^2x^2$\\

$T_{24,20}= 0$&

$T_{24,21}= 0$&

$T_{24,22}= 0$\\

$T_{24,23}= 0$&

$T_{24,24}= 0$&

$T_{24,25}= 0$\\

$T_{24,26}= 0$&

$T_{24,27}= yz^{-2}w^3x^2$&

$T_{24,28}= 0$\\

$T_{24,29}= yzwx^2+ywx^2$&

$T_{24,30}= 0$&

$T_{24,31}= yzwx^2+ywx^2$\\

$T_{24,32}= 0$&

$T_{24,33}= 0$&

$T_{24,34}= 0$\\

$T_{24,35}= 2yz^{-1}w^2x^2$&

$T_{24,36}= 0$&

$T_{24,37}= yzwx^2+2ywx^2$\\

$T_{25,1}= 0$&

$T_{25,2}= yzwx^2$&

$T_{25,3}= 0$\\

$T_{25,4}= yzwx^2$&

$T_{25,5}= 0$&

$T_{25,6}= ywx^2$\\

$T_{25,7}= 0$&

$T_{25,8}= 0$&

$T_{25,9}= 0$\\

$T_{25,10}= yw^2x^2$&

$T_{25,11}= 0$&

$T_{25,12}= 0$\\

$T_{25,13}= 0$&

$T_{25,14}= 0$&

$T_{25,15}= yzw^2x^4$\\

$T_{25,16}= 0$&

$T_{25,17}= 0$&

$T_{25,18}= 0$\\

$T_{25,19}= yz^{-1}w^2x^2$&

$T_{25,20}= 0$&

$T_{25,21}= 0$\\

$T_{25,22}= 0$&

$T_{25,23}= 0$&

$T_{25,24}= 0$\\

$T_{25,25}= 0$&

$T_{25,26}= 0$&

$T_{25,27}= yz^{-1}w^3x^2$\\

$T_{25,28}= 0$&

$T_{25,29}= 2yzwx^2$&

$T_{25,30}= 0$\\

$T_{25,31}= yzwx^2+ywx^2$&

$T_{25,32}= 0$&

$T_{25,33}= 0$\\

$T_{25,34}= 0$&

$T_{25,35}= yw^2x^2+yz^{-1}w^2x^2$&

$T_{25,36}= 0$\\

$T_{25,37}= 2yzwx^2+ywx^2$&

$T_{26,1}= 0$&

$T_{26,2}= ywx^2$\\

$T_{26,3}= 0$&

$T_{26,4}= yzwx^2$&

$T_{26,5}= 0$\\

$T_{26,6}= yzwx^2$&

$T_{26,7}= 0$&

$T_{26,8}= 0$\\

$T_{26,9}= 0$&

$T_{26,10}= yz^{-1}w^2x^2$&

$T_{26,11}= 0$\\

$T_{26,12}= 0$&

$T_{26,13}= 0$&

$T_{26,14}= 0$\\

$T_{26,15}= yzw^2x^4$&

$T_{26,16}= 0$&

$T_{26,17}= 0$\\

$T_{26,18}= 0$&

$T_{26,19}= yw^2x^2$&

$T_{26,20}= 0$\\

$T_{26,21}= 0$&

$T_{26,22}= 0$&

$T_{26,23}= 0$\\

$T_{26,24}= 0$&

$T_{26,25}= 0$&

$T_{26,26}= 0$\\

$T_{26,27}= yz^{-1}w^3x^2$&

$T_{26,28}= 0$&

$T_{26,29}= yzwx^2+ywx^2$\\

$T_{26,30}= 0$&

$T_{26,31}= 2yzwx^2$&

$T_{26,32}= 0$\\

$T_{26,33}= 0$&

$T_{26,34}= 0$&

$T_{26,35}= yw^2x^2+yz^{-1}w^2x^2$\\

$T_{26,36}= 0$&

$T_{26,37}= 2yzwx^2+ywx^2$&

$T_{27,1}= 0$\\

$T_{27,2}= yzwx^2$&

$T_{27,3}= 0$&

$T_{27,4}= yzwx^2$\\

$T_{27,5}= 0$&

$T_{27,6}= yzwx^2$&

$T_{27,7}= 0$\\

$T_{27,8}= 0$&

$T_{27,9}= 0$&

$T_{27,10}= yw^2x^2$\\

$T_{27,11}= 0$&

$T_{27,12}= 0$&

$T_{27,13}= 0$\\

$T_{27,14}= 0$&

$T_{27,15}= yz^2w^2x^4$&

$T_{27,16}= 0$\\

$T_{27,17}= 0$&

$T_{27,18}= 0$&

$T_{27,19}= yw^2x^2$\\

$T_{27,20}= 0$&

$T_{27,21}= 0$&

$T_{27,22}= 0$\\

$T_{27,23}= 0$&

$T_{27,24}= 0$&

$T_{27,25}= 0$\\

$T_{27,26}= 0$&

$T_{27,27}= yw^3x^2$&

$T_{27,28}= 0$\\

$T_{27,29}= 2yzwx^2$&

$T_{27,30}= 0$&

$T_{27,31}= 2yzwx^2$\\

$T_{27,32}= 0$&

$T_{27,33}= 0$&

$T_{27,34}= 0$\\

$T_{27,35}= 2yw^2x^2$&

$T_{27,36}= 0$&

$T_{27,37}= 3yzwx^2$\\

$T_{28,1}= 0$&

$T_{28,2}= 0$&

$T_{28,3}= 0$\\

$T_{28,4}= 0$&

$T_{28,5}= 0$&

$T_{28,6}= 0$\\

$T_{28,7}= 0$&

$T_{28,8}= 0$&

$T_{28,9}= 0$\\

$T_{28,10}= 0$&

$T_{28,11}= 0$&

$T_{28,12}= 0$\\

$T_{28,13}= 0$&

$T_{28,14}= 0$&

$T_{28,15}= 0$\\

$T_{28,16}= 0$&

$T_{28,17}= 0$&

$T_{28,18}= 0$\\

$T_{28,19}= 0$&

$T_{28,20}= 0$&

$T_{28,21}= 0$\\

$T_{28,22}= 0$&

$T_{28,23}= 0$&

$T_{28,24}= 0$\\

$T_{28,25}= 0$&

$T_{28,26}= 0$&

$T_{28,27}= 0$\\

$T_{28,28}= 0$&

$T_{28,29}= yz^2wx^2$&

$T_{28,30}= 0$\\

$T_{28,31}= 0$&

$T_{28,32}= 0$&

$T_{28,33}= yz^2w^2x^4$\\

$T_{28,34}= 0$&

$T_{28,35}= 0$&

$T_{28,36}= 0$\\

$T_{28,37}= 0$&

$T_{29,1}= 0$&

$T_{29,2}= 0$\\

$T_{29,3}= 0$&

$T_{29,4}= 0$&

$T_{29,5}= 0$\\

$T_{29,6}= 0$&

$T_{29,7}= 0$&

$T_{29,8}= 0$\\

$T_{29,9}= 0$&

$T_{29,10}= 0$&

$T_{29,11}= 0$\\

$T_{29,12}= 0$&

$T_{29,13}= 0$&

$T_{29,14}= 0$\\

$T_{29,15}= 0$&

$T_{29,16}= 0$&

$T_{29,17}= 0$\\

$T_{29,18}= 0$&

$T_{29,19}= 0$&

$T_{29,20}= 0$\\

$T_{29,21}= 0$&

$T_{29,22}= 0$&

$T_{29,23}= 0$\\

$T_{29,24}= 0$&

$T_{29,25}= 0$&

$T_{29,26}= 0$\\

$T_{29,27}= 0$&

$T_{29,28}= 0$&

$T_{29,29}= ywx^2$\\

$T_{29,30}= 0$&

$T_{29,31}= 0$&

$T_{29,32}= 0$\\

$T_{29,33}= yw^2x^4$&

$T_{29,34}= 0$&

$T_{29,35}= 0$\\

$T_{29,36}= 0$&

$T_{29,37}= 0$&

$T_{30,1}= 0$\\

$T_{30,2}= 0$&

$T_{30,3}= 0$&

$T_{30,4}= 0$\\

$T_{30,5}= 0$&

$T_{30,6}= 0$&

$T_{30,7}= 0$\\

$T_{30,8}= 0$&

$T_{30,9}= 0$&

$T_{30,10}= 0$\\

$T_{30,11}= 0$&

$T_{30,12}= 0$&

$T_{30,13}= 0$\\

$T_{30,14}= 0$&

$T_{30,15}= 0$&

$T_{30,16}= 0$\\

$T_{30,17}= 0$&

$T_{30,18}= 0$&

$T_{30,19}= 0$\\

$T_{30,20}= 0$&

$T_{30,21}= 0$&

$T_{30,22}= 0$\\

$T_{30,23}= 0$&

$T_{30,24}= 0$&

$T_{30,25}= 0$\\

$T_{30,26}= 0$&

$T_{30,27}= 0$&

$T_{30,28}= 0$\\

$T_{30,29}= 0$&

$T_{30,30}= 0$&

$T_{30,31}= yz^2wx^2$\\

$T_{30,32}= 0$&

$T_{30,33}= 0$&

$T_{30,34}= yz^2w^2x^4$\\

$T_{30,35}= 0$&

$T_{30,36}= 0$&

$T_{30,37}= 0$\\

$T_{31,1}= 0$&

$T_{31,2}= 0$&

$T_{31,3}= 0$\\

$T_{31,4}= 0$&

$T_{31,5}= 0$&

$T_{31,6}= 0$\\

$T_{31,7}= 0$&

$T_{31,8}= 0$&

$T_{31,9}= 0$\\

$T_{31,10}= 0$&

$T_{31,11}= 0$&

$T_{31,12}= 0$\\

$T_{31,13}= 0$&

$T_{31,14}= 0$&

$T_{31,15}= 0$\\

$T_{31,16}= 0$&

$T_{31,17}= 0$&

$T_{31,18}= 0$\\

$T_{31,19}= 0$&

$T_{31,20}= 0$&

$T_{31,21}= 0$\\

$T_{31,22}= 0$&

$T_{31,23}= 0$&

$T_{31,24}= 0$\\

$T_{31,25}= 0$&

$T_{31,26}= 0$&

$T_{31,27}= 0$\\

$T_{31,28}= 0$&

$T_{31,29}= 0$&

$T_{31,30}= 0$\\

$T_{31,31}= ywx^2$&

$T_{31,32}= 0$&

$T_{31,33}= 0$\\

$T_{31,34}= yw^2x^4$&

$T_{31,35}= 0$&

$T_{31,36}= 0$\\

$T_{31,37}= 0$&

$T_{32,1}= 0$&

$T_{32,2}= 0$\\

$T_{32,3}= 0$&

$T_{32,4}= 0$&

$T_{32,5}= 0$\\

$T_{32,6}= 0$&

$T_{32,7}= 0$&

$T_{32,8}= 0$\\

$T_{32,9}= 0$&

$T_{32,10}= 0$&

$T_{32,11}= 0$\\

$T_{32,12}= 0$&

$T_{32,13}= 0$&

$T_{32,14}= 0$\\

$T_{32,15}= 0$&

$T_{32,16}= 0$&

$T_{32,17}= 0$\\

$T_{32,18}= 0$&

$T_{32,19}= 0$&

$T_{32,20}= 0$\\

$T_{32,21}= 0$&

$T_{32,22}= 0$&

$T_{32,23}= 0$\\

$T_{32,24}= 0$&

$T_{32,25}= 0$&

$T_{32,26}= 0$\\

$T_{32,27}= 0$&

$T_{32,28}= 0$&

$T_{32,29}= ywx^2$\\

$T_{32,30}= 0$&

$T_{32,31}= ywx^2$&

$T_{32,32}= 0$\\

$T_{32,33}= 0$&

$T_{32,34}= 0$&

$T_{32,35}= yz^{-2}w^2x^2$\\

$T_{32,36}= 0$&

$T_{32,37}= 2ywx^2$&

$T_{33,1}= 0$\\

$T_{33,2}= 0$&

$T_{33,3}= 0$&

$T_{33,4}= 0$\\

$T_{33,5}= 0$&

$T_{33,6}= 0$&

$T_{33,7}= 0$\\

$T_{33,8}= 0$&

$T_{33,9}= 0$&

$T_{33,10}= 0$\\

$T_{33,11}= 0$&

$T_{33,12}= 0$&

$T_{33,13}= 0$\\

$T_{33,14}= 0$&

$T_{33,15}= 0$&

$T_{33,16}= 0$\\

$T_{33,17}= 0$&

$T_{33,18}= 0$&

$T_{33,19}= 0$\\

$T_{33,20}= 0$&

$T_{33,21}= 0$&

$T_{33,22}= 0$\\

$T_{33,23}= 0$&

$T_{33,24}= 0$&

$T_{33,25}= 0$\\

$T_{33,26}= 0$&

$T_{33,27}= 0$&

$T_{33,28}= 0$\\

$T_{33,29}= yzwx^2$&

$T_{33,30}= 0$&

$T_{33,31}= ywx^2$\\

$T_{33,32}= 0$&

$T_{33,33}= 0$&

$T_{33,34}= 0$\\

$T_{33,35}= yz^{-1}w^2x^2$&

$T_{33,36}= 0$&

$T_{33,37}= yzwx^2+ywx^2$\\

$T_{34,1}= 0$&

$T_{34,2}= 0$&

$T_{34,3}= 0$\\

$T_{34,4}= 0$&

$T_{34,5}= 0$&

$T_{34,6}= 0$\\

$T_{34,7}= 0$&

$T_{34,8}= 0$&

$T_{34,9}= 0$\\

$T_{34,10}= 0$&

$T_{34,11}= 0$&

$T_{34,12}= 0$\\

$T_{34,13}= 0$&

$T_{34,14}= 0$&

$T_{34,15}= 0$\\

$T_{34,16}= 0$&

$T_{34,17}= 0$&

$T_{34,18}= 0$\\

$T_{34,19}= 0$&

$T_{34,20}= 0$&

$T_{34,21}= 0$\\

$T_{34,22}= 0$&

$T_{34,23}= 0$&

$T_{34,24}= 0$\\

$T_{34,25}= 0$&

$T_{34,26}= 0$&

$T_{34,27}= 0$\\

$T_{34,28}= 0$&

$T_{34,29}= ywx^2$&

$T_{34,30}= 0$\\

$T_{34,31}= yzwx^2$&

$T_{34,32}= 0$&

$T_{34,33}= 0$\\

$T_{34,34}= 0$&

$T_{34,35}= yz^{-1}w^2x^2$&

$T_{34,36}= 0$\\

$T_{34,37}= yzwx^2+ywx^2$&

$T_{35,1}= 0$&

$T_{35,2}= 0$\\

$T_{35,3}= 0$&

$T_{35,4}= 0$&

$T_{35,5}= 0$\\

$T_{35,6}= 0$&

$T_{35,7}= 0$&

$T_{35,8}= 0$\\

$T_{35,9}= 0$&

$T_{35,10}= 0$&

$T_{35,11}= 0$\\

$T_{35,12}= 0$&

$T_{35,13}= 0$&

$T_{35,14}= 0$\\

$T_{35,15}= 0$&

$T_{35,16}= 0$&

$T_{35,17}= 0$\\

$T_{35,18}= 0$&

$T_{35,19}= 0$&

$T_{35,20}= 0$\\

$T_{35,21}= 0$&

$T_{35,22}= 0$&

$T_{35,23}= 0$\\

$T_{35,24}= 0$&

$T_{35,25}= 0$&

$T_{35,26}= 0$\\

$T_{35,27}= 0$&

$T_{35,28}= 0$&

$T_{35,29}= yzwx^2$\\

$T_{35,30}= 0$&

$T_{35,31}= yzwx^2$&

$T_{35,32}= 0$\\

$T_{35,33}= 0$&

$T_{35,34}= 0$&

$T_{35,35}= yw^2x^2$\\

$T_{35,36}= 0$&

$T_{35,37}= 2yzwx^2$&

$T_{36,1}= 0$\\

$T_{36,2}= 0$&

$T_{36,3}= 0$&

$T_{36,4}= 0$\\

$T_{36,5}= 0$&

$T_{36,6}= 0$&

$T_{36,7}= 0$\\

$T_{36,8}= 0$&

$T_{36,9}= 0$&

$T_{36,10}= 0$\\

$T_{36,11}= 0$&

$T_{36,12}= 0$&

$T_{36,13}= 0$\\

$T_{36,14}= 0$&

$T_{36,15}= 0$&

$T_{36,16}= 0$\\

$T_{36,17}= 0$&

$T_{36,18}= 0$&

$T_{36,19}= 0$\\

$T_{36,20}= 0$&

$T_{36,21}= 0$&

$T_{36,22}= 0$\\

$T_{36,23}= 0$&

$T_{36,24}= 0$&

$T_{36,25}= 0$\\

$T_{36,26}= 0$&

$T_{36,27}= 0$&

$T_{36,28}= 0$\\

$T_{36,29}= 0$&

$T_{36,30}= 0$&

$T_{36,31}= 0$\\

$T_{36,32}= 0$&

$T_{36,33}= 0$&

$T_{36,34}= 0$\\

$T_{36,35}= 0$&

$T_{36,36}= 0$&

$T_{36,37}= yz^2wx^2$\\

$T_{37,1}= 0$&

$T_{37,2}= 0$&

$T_{37,3}= 0$\\

$T_{37,4}= 0$&

$T_{37,5}= 0$&

$T_{37,6}= 0$\\

$T_{37,7}= 0$&

$T_{37,8}= 0$&

$T_{37,9}= 0$\\

$T_{37,10}= 0$&

$T_{37,11}= 0$&

$T_{37,12}= 0$\\

$T_{37,13}= 0$&

$T_{37,14}= 0$&

$T_{37,15}= 0$\\

$T_{37,16}= 0$&

$T_{37,17}= 0$&

$T_{37,18}= 0$\\

$T_{37,19}= 0$&

$T_{37,20}= 0$&

$T_{37,21}= 0$\\

$T_{37,22}= 0$&

$T_{37,23}= 0$&

$T_{37,24}= 0$\\

$T_{37,25}= 0$&

$T_{37,26}= 0$&

$T_{37,27}= 0$\\

$T_{37,28}= 0$&

$T_{37,29}= 0$&

$T_{37,30}= 0$\\

$T_{37,31}= 0$&

$T_{37,32}= 0$&

$T_{37,33}= 0$\\

$T_{37,34}= 0$&

$T_{37,35}= 0$&

$T_{37,36}= 0$\\

$T_{37,37}= ywx^2$& &\\
\hline
\caption{\label{tab7} Matrice $\mathcal{M}_{3}$.}
\end{longtable}
\normalsize
 \begin{spacing}{0.30}
\section*{Quelques exemples d'énumérations de polyominos inscrits dans le rectangle $3 \times 4$ }
 \end{spacing}
\addcontentsline{toc}{section}{Quelques exemples d'énumérations de polyominos inscrits dans le rectangle $3 \times 4$ }
Dans  cette section, nous allons présenter quelques résultats sur les polyominos inscrits dans le rectangle $3 \times 4$. 

On considère, pour cela, la matrice $\mathcal{M}_{3}$. On désigne par $\mathcal{M}_{3,3}$ le cube de $\mathcal{M}_{3}$. Les états initiaux de $\mathcal{A}_{3}$ sont $e_{1}$, $e_{3}$, $e_{5}$, $e_{7}$, $e_{11}$, $e_{16}$ et $e_{20}$. Ses états finaux sont $e_{15}$, $e_{21}$, $e_{22}$, $e_{23}$, $e_{24}$, $e_{25}$, $e_{26}$, $e_{27}$, $e_{33}$, $e_{34}$, $e_{35}$ et $e_{37}$. D'après la proposition \ref{inscr1}, tout polyomino résultant d’une suite de transitions à $4$ états partant d’un état initial à un état final, de $\mathcal{A}_{3}$, est inscrit dans le rectangle $3\times 4$. Ainsi pour un état initial $e_{i}$ et un état final $e_{j}$ de $\mathcal{A}_{3}$, l'entrée $\mathcal{M}_{3,3}(i,j)$ de la matrice $\mathcal{M}_{3,3}$ nous donne les informations nécessaires sur le nombre de polyominos de hauteur $4$ issus de ces deux états. Si $n_{f}$, $n_{a}$ et $n_{p}$ sont respectivement le nombre de feuilles, l'aire et le périmètre de $e_{i}$ et si $\mathcal{M}_{3,3}(i,j)=z^{n_{z}}w^{n_{w}}x^{n_{x}}y^{n_{y}}$ alors 
$n_{f} +n_{z}$, $n_{a}+n_{w}$, $n_{p} + n_{x}$, $1+n_{y}$ sont respectivement le nombre de feuilles, l'aire, le périmètre et la hauteur du polyomino résultant.
 \begin{spacing}{0.30}
\subsection*{Cas des états $e_{1}$ et $e_{15}$ }
 \end{spacing}
$\mathcal{M}_{3,3}(1,15)=z^3w^7x^{14}y^3+z^3w^6x^{12}y^3+z^3w^8x^{12}y^3+z^3w^7x^{12}y^3$\\
\begin{tabular}{|c|c|c|c|c|}
 \hline
  Nbre de polyominos & Nbre de cellules & Périmètre & hauteur &Nbre de feuilles\\
 \hline
 $1$ & $8$ & $18$ & $4$ &$3$\\
 \hline
 $1$ & $7$ & $16$ & $4$ &$3$\\
 \hline
  $1$ & $9$ & $16$ & $4$ &$3$\\
 \hline
 $1$ & $8$ & $16$ & $4$ &$3$\\
 \hline
\end{tabular}
\subsection*{Cas des états $e_{1}$ et $e_{21}$ }
$\mathcal{M}_{3,3}(1,21)=y^3z^2w^5x^{10}+y^3z^3w^6x^{12}+y^3z^3w^7x^{14}$\\
\begin{tabular}{|c|c|c|c|c|}
 \hline
  Nbre de polyominos & Nbre de cellules & Périmètre & hauteur &Nbre de feuilles\\
 \hline
 $1$ & $6$ & $14$ & $4$ &$2$\\
 \hline
 $1$ & $7$ & $16$ & $4$ &$3$\\
 \hline
  $1$ & $8$ & $18$ & $4$ &$3$\\
 \hline
\end{tabular}
\subsection*{Cas des états $e_{1}$ et $e_{22}$ }
$\mathcal{M}_{3,3}(1,22)=x^{14}y^3z^2w^7$\\
\begin{tabular}{|c|c|c|c|c|}
 \hline
  Nbre de polyominos & Nbre de cellules & Périmètre & hauteur &Nbre de feuilles\\
 \hline
 $1$ & $8$ & $18$ & $4$ &$2$\\
 \hline
\end{tabular}
\subsection*{Cas des états $e_{1}$ et $e_{23}$ }
$\mathcal{M}_{3,3}(1,23)=zw^8x^{14}y^3
$\\
\begin{tabular}{|c|c|c|c|c|}
 \hline
  Nbre de polyominos & Nbre de cellules & Périmètre & hauteur &Nbre de feuilles\\
 \hline
 $1$ & $9$ & $18$ & $4$ &$1$\\
 \hline
\end{tabular}
\subsection*{Cas des états $e_{1}$ et $e_{24}$ }
$\mathcal{M}_{3,3}(1,24)=y^3z^3w^6x^{12}+y^3z^4w^7x^{14}
$\\
\begin{tabular}{|c|c|c|c|c|}
 \hline
  Nbre de polyominos & Nbre de cellules & Périmètre & hauteur &Nbre de feuilles\\
 \hline
 $1$ & $7$ & $16$ & $4$ &$3$\\
 \hline
 $1$ & $8$ & $18$ & $4$ &$4$\\
 \hline
\end{tabular}
\subsection*{Cas des états $e_{1}$ et $e_{25}$ }
$\mathcal{M}_{3,3}(1,25)=x^{10}y^3z^2w^6+x^{10}y^3z^2w^7+x^{12}y^3z^3w^8
$\\
\begin{tabular}{|c|c|c|c|c|}
 \hline
  Nbre de polyominos & Nbre de cellules & Périmètre & hauteur &Nbre de feuilles\\
 \hline
 $1$ & $7$ & $14$ & $4$ &$2$\\
 \hline
 $1$ & $8$ & $14$ & $4$ &$2$\\
 \hline
 $1$ & $9$ & $16$ & $4$ &$3$\\
 \hline
\end{tabular}
\subsection*{Cas des états $e_{1}$ et $e_{26}$ }
$\mathcal{M}_{3,3}(1,26)=y^3z^2w^8x^{12}+y^3z^2w^7x^{12}$\\
\begin{tabular}{|c|c|c|c|c|}
 \hline
  Nbre de polyominos & Nbre de cellules & Périmètre & hauteur &Nbre de feuilles\\
 \hline
 $1$ & $9$ & $16$ & $4$ &$3$\\
 \hline
 $1$ & $8$ & $16$ & $4$ &$2$\\
 \hline
\end{tabular}
\subsection*{Cas des états $e_{1}$ et $e_{33}$ }
$\mathcal{M}_{3,3}(1,33)=2y^3z^3w^6x^{12}+y^3z^3w^7x^{14}+y^3z^3w^7x^{12}+y^3z^2w^6x^{10}+2y^3z^2w^5x^{10}+y^3z^2w^5x^{12}
$\\
\begin{tabular}{|c|c|c|c|c|}
 \hline
  Nbre de polyominos & Nbre de cellules & Périmètre & hauteur &Nbre de feuilles\\
 \hline
 $2$ & $7$ & $16$ & $4$ &$3$\\
 \hline
 $1$ & $8$ & $18$ & $4$ &$3$\\
 \hline
 $1$ & $8$ & $16$ & $4$ &$3$\\
 \hline
 $1$ & $7$ & $14$ & $4$ &$2$\\
 \hline
 $2$ & $6$ & $14$ & $4$ &$2$\\
 \hline
 $1$ & $6$ & $16$ & $4$ &$2$\\
 \hline
\end{tabular}
\subsection*{Cas des états $e_{1}$ et $e_{34}$ }
$\mathcal{M}_{3,3}(1,34)=y^3z^3w^6x^{12}+y^3z^3w^7x^{14}+y^3z^2w^5x^{12}+y^3z^2w^7x^{12}
$\\
\begin{tabular}{|c|c|c|c|c|}
 \hline
  Nbre de polyominos & Nbre de cellules & Périmètre & hauteur &Nbre de feuilles\\
 \hline
 $1$ & $7$ & $16$ & $4$ &$3$\\
 \hline
 $1$ & $8$ & $18$ & $4$ &$3$\\
 \hline
 $1$ & $6$ & $16$ & $4$ &$2$\\
 \hline
 $1$ & $8$ & $16$ & $4$ &$2$\\
 \hline
\end{tabular}
\subsection*{Cas des états $e_{1}$ et $e_{35}$ }
$\mathcal{M}_{3,3}(1,35)=2x^{10}y^3zw^8+x^{10}y^3z^2w^5+x^{10}y^3zw^6+x^{10}y^3w^{6}
$\\
\begin{tabular}{|c|c|c|c|c|}
 \hline
  Nbre de polyominos & Nbre de cellules & Périmètre & hauteur &Nbre de feuilles\\
 \hline
 $2$ & $9$ & $14$ & $4$ &$1$\\
 \hline
 $1$ & $6$ & $14$ & $4$ &$2$\\
 \hline
 $1$ & $7$ & $14$ & $4$ &$1$\\
 \hline
 $1$ & $7$ & $14$ & $4$ &$0$\\
 \hline
\end{tabular}
\subsection*{Cas des états $e_{1}$ et $e_{37}$ }
$\mathcal{M}_{3,3}(1,37)=2y^3z^3w^6x^{12}+2y^3zw^5x^{10}+2y^3z^2w^4x^{10}+3y^3z^2w^7x^{10}+2y^3z^4w^4x^{10
}$\\
\begin{tabular}{|c|c|c|c|c|}
 \hline
  Nbre de polyominos & Nbre de cellules & Périmètre & hauteur &Nbre de feuilles\\
 \hline
 $2$ & $7$ & $16$ & $4$ &$3$\\
 \hline
 $2$ & $6$ & $14$ & $4$ &$1$\\
 \hline
 $2$ & $5$ & $14$ & $4$ &$2$\\
 \hline
 $3$ & $8$ & $14$ & $4$ &$2$\\
 \hline
 $2$ & $5$ & $14$ & $4$ &$4$\\
 \hline
\end{tabular}
\subsection*{Cas des états $e_{20}$ et $e_{37}$ }
$e_{20}$ a $2$ feuilles, $3$ cellules et $8$ comme périmètre.

$\mathcal{M}_{3,3}(20,37)=3z^2w^7x^{10}y^3+2z^3w^6x^{12}y^3+2zw^5x^{10}y^3+2z^4w^4x^{10}y^3+2z^2w^4x^{10}y^3$\\
\begin{tabular}{|c|c|c|c|c|}
 \hline
  Nbre de polyominos & Nbre de cellules & Périmètre & hauteur &Nbre de feuilles\\
 \hline
 $3$ & $10$ & $18$ & $4$ &$4$\\
 \hline
 $2$ & $9$ & $20$ & $4$ &$5$\\
 \hline
 $2$ & $8$ & $18$ & $4$ &$3$\\
 \hline
 $2$ & $7$ & $18$ & $4$ &$6$\\
 \hline
 $2$ & $7$ & $18$ & $4$ &$4$\\
 \hline
\end{tabular}

On retrouve ci-dessous l'intégralité de la matrice $\mathcal{M}_{3,3}$.

\scriptsize
$\mathcal{M}_{3,3}(1,1)=0$\\
$\mathcal{M}_{3,3}(1,2)=2z^3w^5x^{10}y^3+z^2w^3x^6y^3+z^3w^6x^{12}y^3+2z^3w^6x^10y^3+z^2w^7x^{10}y^3+2z^3w^4x^8y^3+z^2w^5x^8y^3$\\
$\mathcal{M}_{3,3}(1,3)=0$\\
$\mathcal{M}_{3,3}(1,4)=3z^3w^5x^{10}y^3+2z^2w^4x^8y^3+2z^3w^6x^{10}y^3+z^2w^7x^{10}y^3+z^2w^6x^{10}y^3+z^2w^5x^8y^3$\\
$\mathcal{M}_{3,3}(1,5)=0$\\
$\mathcal{M}_{3,3}(1,6)=3z^2w^5x^{10}y^3+z^3w^6x^12y^3+z^2w^7x^{10}y^3+2z^2w^6x^{10}y^3$\\
$\mathcal{M}_{3,3}(1,7)=0$\\
$\mathcal{M}_{3,3}(1,8)=z^3w^5x^{10}y^3+z^3w^7x^{14}y^3+z^3w^6x^{12}y^3+z^2w^4x^8y^3$\\
$\mathcal{M}_{3,3}(1,9)=z^2w^5x^{10}y^3+2z^3w^6x^{12}y^3+z^2w^7x^{12}y^3$\\
$\mathcal{M}_{3,3}(1,10)=zw^8x^{10}y^3+zw^5x^8y^3+zw^6x^8y^3+2z^2w^7x^{10}y^3+z^2w^6x^{10}y^3$\\
$\mathcal{M}_{3,3}(1,11)=0$\\
$\mathcal{M}_{3,3}(1,12)=z^4w^6x^14y^3+z^3w^7x^{14}y^3+z^3w^6x^{12}y^3+z^5w^6x^{14}y^3+z^4w^5x^{12}y^3+z^3w^8x^{12}y^3+z^3w^7x^{12}y^3$\\
$\mathcal{M}_{3,3}(1,13)=2z^3w^5x^{12}y^3+z^2w^4x^{10}y^3+z^3w^7x^{14}y^3+z^2w^6x^{12}y^3+z^3w^6x^{14}y^3$\\
$\mathcal{M}_{3,3}(1,14)=2z^2w^6x^{14}y^3+z^2w^7x^{14}y^3$\\
$\mathcal{M}_{3,3}(1,15)=z^3w^7x^{14}y^3+z^3w^6x^{12}y^3+z^3w^8x^{12}y^3+z^3w^7x^{12}y^3$\\
$\mathcal{M}_{3,3}(1,16)=0$\\
$\mathcal{M}_{3,3}(1,17)=2z^2w^5x^{10}y^3+z^3w^6x^{12}y^3+z^2w^6x^{10}y^3+z^3w^7x^{12}y^3$\\
$\mathcal{M}_{3,3}(1,18)=z^3w^7x^{14}y^3+z^2w^6x^{12}y^3$\\
$\mathcal{M}_{3,3}(1,19)=zw^8x^{10}y^3+2zw^7x^{10}y^3+2zw^6x^{10}y^3$\\
$\mathcal{M}_{3,3}(1,20)=0$\\
$\mathcal{M}_{3,3}(1,21)=z^2w^5x^{10}y^3+z^3w^7x^{14}y^3+z^3w^6x^{12}y^3$\\
$\mathcal{M}_{3,3}(1,22)=z^2w^7x^{14}y^3$\\
$\mathcal{M}_{3,3}(1,23)=zw^8x^{14}y^3$\\
$\mathcal{M}_{3,3}(1,24)=z^4w^7x^{14}y^3+z^3w^6x^{12}y^3$\\
$\mathcal{M}_{3,3}(1,25)=z^2w^7x^{10}y^3+z^2w^6x^{10}y^3+z^3w^8x^{12}y^3$\\
$\mathcal{M}_{3,3}(1,26)=z^2w^7x^{12}y^3+z^2w^8x^{12}y^3$\\
$\mathcal{M}_{3,3}(1,27)=zw^8x^{10}y^3+zw^7x^{10}y^3+zw^9x^{10}y^3$\\
$\mathcal{M}_{3,3}(1,28)=0$\\
$\mathcal{M}_{3,3}(1,29)=z^2w^4x^{10}y^3+5z^3w^5x^{10}y^3+z^3w^6x^{12}y^3+zw^5x^{10}y^3+2z^2w^4x^8y^3+3z^3w^6x^{10}y^3+2z^2w^7x^{10}y^3+z^4w^4x^{10}y^3+z^2w^5x^8y^3$\\
$\mathcal{M}_{3,3}(1,30)=0$\\
$\mathcal{M}_{3,3}(1,31)=z^2w^5x^{10}y^3+z^2w^4x^{10}y^3+z^3w^5x^{10}y^3+z^3w^6x^{12}y^3+zw^5x^{10}y^3+2z^2w^7x^{10}y^3+z^4w^4x^{10}y^3+z^2w^6x^{10}y^3$\\
$\mathcal{M}_{3,3}(1,32)=z^4w^6x^{14}y^3+z^5w^5x^{12}y^3+z^5w^6x^{14}y^3+z^4w^4x^{10}y^3$\\
$\mathcal{M}_{3,3}(1,33)=2z^2w^5x^{10}y^3+z^2w^5x^{12}y^3+z^3w^7x^{14}y^3+2z^3w^6x^{12}y^3+z^2w^6x^{10}y^3+z^3w^7x^{12}y^3$\\
$\mathcal{M}_{3,3}(1,34)=z^2w^5x^{12}y^3+z^3w^7x^{14}y^3+z^3w^6x^{12}y^3+z^2w^7x^{12}y^3$\\
$\mathcal{M}_{3,3}(1,35)=z^2w^5x^{10}y^3+2zw^8x^{10}y^3+w^6x^{10}y^3+zw^6x^{10}y^3$\\
$\mathcal{M}_{3,3}(1,36)=0$\\
$\mathcal{M}_{3,3}(1,37)=2z^2w^4x^{10}y^3+2z^3w^6x^{12}y^3+2zw^5x^{10}y^3+3z^2w^7x^{10}y^3+2z^4w^4x^{10}y^3$\\
$\mathcal{M}_{3,3}(2,1)=0$\\
$\mathcal{M}_{3,3}(2,2)=2z^3w^5x^{10}y^3+z^2w^3x^6y^3+z^3w^6x^{12}y^3+2z^3w^6x^{10}y^3+z^2w^7x^{10}y^3+2z^3w^4x^8y^3+z^2w^5x^8y^3$\\
$\mathcal{M}_{3,3}(2,3)=0$\\
$\mathcal{M}_{3,3}(2,4)=3z^3w^5x^{10}y^3+2z^2w^4x^8y^3+2z^3w^6x^{10}y^3+z^2w^7x^{10}y^3+z^2w^6x^{10}y^3+z^2w^5x^8y^3$\\
$\mathcal{M}_{3,3}(2,5)=0$\\
$\mathcal{M}_{3,3}(2,6)=3z^2w^5x^{10}y^3+z^3w^6x^{12}y^3+z^2w^7x^{10}y^3+2z^2w^6x^{10}y^3$\\
$\mathcal{M}_{3,3}(2,7)=0$\\
$\mathcal{M}_{3,3}(2,8)=z^3w^5x^{10}y^3+z^3w^7x^{14}y^3+z^3w^6x^{12}y^3+z^2w^4x^8y^3$\\
$\mathcal{M}_{3,3}(2,9)=z^2w^5x^{10}y^3+2z^3w^6x^{12}y^3+z^2w^7x^{12}y^3$\\
$\mathcal{M}_{3,3}(2,10)=zw^8x^{10}y^3+zw^5x^8y^3+zw^6x^8y^3+2z^2w^7x^{10}y^3+z^2w^6x^{10}y^3$\\
$\mathcal{M}_{3,3}(2,11)=0$\\
$\mathcal{M}_{3,3}(2,12)=z^4w^6x^{14}y^3+z^3w^7x^{14}y^3+z^3w^6x^{12}y^3+z^5w^6x^{14}y^3+z^4w^5x^{12}y^3+z^3w^8x^{12}y^3+z^3w^7x^{12}y^3$\\
$\mathcal{M}_{3,3}(2,13)=2z^3w^5x^{12}y^3+z^2w^4x^{10}y^3+z^3w^7x^{14}y^3+z^2w^6x^{12}y^3+z^3w^6x^{14}y^3$\\
$\mathcal{M}_{3,3}(2,14)=2z^2w^6x^{14}y^3+z^2w^7x^{14}y^3$\\
$\mathcal{M}_{3,3}(2,15)=z^3w^7x^{14}y^3+z^3w^6x^{12}y^3+z^3w^8x^{12}y^3+z^3w^7x^{12}y^3$\\
$\mathcal{M}_{3,3}(2,16)=0$\\
$\mathcal{M}_{3,3}(2,17)=2z^2w^5x^{10}y^3+z^3w^6x^{12}y^3+z^2w^6x^{10}y^3+z^3w^7x^{12}y^3$\\
$\mathcal{M}_{3,3}(2,18)=z^3w^7x^{14}y^3+z^2w^6x^{12}y^3$\\
$\mathcal{M}_{3,3}(2,19)=zw^8x^{10}y^3+2zw^7x^{10}y^3+2zw^6x^{10}y^3$\\
$\mathcal{M}_{3,3}(2,20)=0$\\
$\mathcal{M}_{3,3}(2,21)=z^2w^5x^{10}y^3+z^3w^7x^{14}y^3+z^3w^6x^{12}y^3$\\
$\mathcal{M}_{3,3}(2,22)=z^2w^7x^{14}y^3$\\
$\mathcal{M}_{3,3}(2,23)=zw^8x^{14}y^3$\\
$\mathcal{M}_{3,3}(2,24)=z^4w^7x^{14}y^3+z^3w^6x^{12}y^3$\\
$\mathcal{M}_{3,3}(2,25)=z^2w^7x^{10}y^3+z^2w^6x^{10}y^3+z^3w^8x^{12}y^3$\\
$\mathcal{M}_{3,3}(2,26)=z^2w^7x^{12}y^3+z^2w^8x^{12}y^3$\\
$\mathcal{M}_{3,3}(2,27)=zw^8x^{10}y^3+zw^7x^{10}y^3+zw^9x^{10}y^3$\\
$\mathcal{M}_{3,3}(2,28)=0$\\
$\mathcal{M}_{3,3}(2,29)=z^2w^4x^{10}y^3+5z^3w^5x^{10}y^3+z^3w^6x^{12}y^3+zw^5x^{10}y^3+2z^2w^4x^8y^3+3z^3w^6x^{10}y^3+2z^2w^7x^{10}y^3+z^4w^4x^{10}y^3+z^2w^5x^8y^3$\\
$\mathcal{M}_{3,3}(2,30)=0$\\
$\mathcal{M}_{3,3}(2,31)=z^2w^5x^{10}y^3+z^2w^4x^{10}y^3+z^3w^5x^{10}y^3+z^3w^6x^{12}y^3+zw^5x^{10}y^3+2z^2w^7x^{10}y^3+z^4w^4x^{10}y^3+z^2w^6x^{10}y^3$\\
$\mathcal{M}_{3,3}(2,32)=z^4w^6x^{14}y^3+z^5w^5x^{12}y^3+z^5w^6x^{14}y^3+z^4w^4x^{10}y^3$\\
$\mathcal{M}_{3,3}(2,33)=2z^2w^5x^{10}y^3+z^2w^5x^{12}y^3+z^3w^7x^{14}y^3+2z^3w^6x^{12}y^3+z^2w^6x^{10}y^3+z^3w^7x^{12}y^3$\\
$\mathcal{M}_{3,3}(2,34)=z^2w^5x^{12}y^3+z^3w^7x^{14}y^3+z^3w^6x^{12}y^3+z^2w^7x^{12}y^3$\\
$\mathcal{M}_{3,3}(2,35)=z^2w^5x^{10}y^3+2zw^8x^{10}y^3+w^6x^{10}y^3+zw^6x^{10}y^3$\\
$\mathcal{M}_{3,3}(2,36)=0$\\
$\mathcal{M}_{3,3}(2,37)=2z^2w^4x^{10}y^3+2z^3w^6x^{12}y^3+2zw^5x^{10}y^3+3z^2w^7x^{10}y^3+2z^4w^4x^{10}y^3$\\
$\mathcal{M}_{3,3}(3,1)=0$\\
$\mathcal{M}_{3,3}(3,2)=2z^3w^5x^{10}y^3+z^2w^3x^6y^3+z^3w^6x^{12}y^3+2z^3w^6x^{10}y^3+z^2w^7x^{10}y^3+2z^3w^4x^8y^3+z^2w^5x^8y^3$\\
$\mathcal{M}_{3,3}(3,3)=0$\\
$\mathcal{M}_{3,3}(3,4)=3z^3w^5x^{10}y^3+2z^2w^4x^8y^3+2z^3w^6x^{10}y^3+z^2w^7x^{10}y^3+z^2w^6x^{10}y^3+z^2w^5x^8y^3$\\
$\mathcal{M}_{3,3}(3,5)=0$\\
$\mathcal{M}_{3,3}(3,6)=3z^2w^5x^{10}y^3+z^3w^6x^{12}y^3+z^2w^7x^{10}y^3+2z^2w^6x^{10}y^3$\\
$\mathcal{M}_{3,3}(3,7)=0$\\
$\mathcal{M}_{3,3}(3,8)=z^3w^5x^{10}y^3+z^3w^7x^{14}y^3+z^3w^6x^{12}y^3+z^2w^4x^8y^3$\\
$\mathcal{M}_{3,3}(3,9)=z^2w^5x^{10}y^3+2z^3w^6x^{12}y^3+z^2w^7x^{12}y^3$\\
$\mathcal{M}_{3,3}(3,10)=zw^8x^{10}y^3+zw^5x^8y^3+zw^6x^8y^3+2z^2w^7x^{10}y^3+z^2w^6x^{10}y^3$\\
$\mathcal{M}_{3,3}(3,11)=0$\\
$\mathcal{M}_{3,3}(3,12)=z^4w^6x^{14}y^3+z^3w^7x^{14}y^3+z^3w^6x^{12}y^3+z^5w^6x^{14}y^3+z^4w^5x^{12}y^3+z^3w^8x^{12}y^3+z^3w^7x^{12}y^3$\\
$\mathcal{M}_{3,3}(3,13)=2z^3w^5x^{12}y^3+z^2w^4x^{10}y^3+z^3w^7x^{14}y^3+z^2w^6x^{12}y^3+z^3w^6x^{14}y^3$\\
$\mathcal{M}_{3,3}(3,14)=2z^2w^6x^{14}y^3+z^2w^7x^{14}y^3$\\
$\mathcal{M}_{3,3}(3,15)=z^3w^7x^{14}y^3+z^3w^6x^{12}y^3+z^3w^8x^{12}y^3+z^3w^7x^{12}y^3$\\
$\mathcal{M}_{3,3}(3,16)=0$\\
$\mathcal{M}_{3,3}(3,17)=2z^2w^5x^{10}y^3+z^3w^6x^{12}y^3+z^2w^6x^{10}y^3+z^3w^7x^{12}y^3$\\
$\mathcal{M}_{3,3}(3,18)=z^3w^7x^{14}y^3+z^2w^6x^{12}y^3$\\
$\mathcal{M}_{3,3}(3,19)=zw^8x^{10}y^3+2zw^7x^{10}y^3+2zw^6x^{10}y^3$\\
$\mathcal{M}_{3,3}(3,20)=0$\\
$\mathcal{M}_{3,3}(3,21)=z^2w^5x^{10}y^3+z^3w^7x^{14}y^3+z^3w^6x^{12}y^3$\\
$\mathcal{M}_{3,3}(3,22)=z^2w^7x^{14}y^3$\\
$\mathcal{M}_{3,3}(3,23)=zw^8x^{14}y^3$\\
$\mathcal{M}_{3,3}(3,24)=z^4w^7x^{14}y^3+z^3w^6x^{12}y^3$\\
$\mathcal{M}_{3,3}(3,25)=z^2w^7x^{10}y^3+z^2w^6x^{10}y^3+z^3w^8x^{12}y^3$\\
$\mathcal{M}_{3,3}(3,26)=z^2w^7x^{12}y^3+z^2w^8x^{12}y^3$\\
$\mathcal{M}_{3,3}(3,27)=zw^8x^{10}y^3+zw^7x^{10}y^3+zw^9x^{10}y^3$\\
$\mathcal{M}_{3,3}(3,28)=0$\\
$\mathcal{M}_{3,3}(3,29)=z^2w^4x^{10}y^3+5z^3w^5x^{10}y^3+z^3w^6x^{12}y^3+zw^5x^{10}y^3+2z^2w^4x^8y^3+3z^3w^6x^{10}y^3+2z^2w^7x^{10}y^3+z^4w^4x^{10}y^3+z^2w^5x^8y^3$\\
$\mathcal{M}_{3,3}(3,30)=0$\\
$\mathcal{M}_{3,3}(3,31)=z^2w^5x^{10}y^3+z^2w^4x^{10}y^3+z^3w^5x^{10}y^3+z^3w^6x^{12}y^3+zw^5x^{10}y^3+2z^2w^7x^{10}y^3+z^4w^4x^{10}y^3+z^2w^6x^{10}y^3$\\
$\mathcal{M}_{3,3}(3,32)=z^4w^6x^{14}y^3+z^5w^5x^{12}y^3+z^5w^6x^{14}y^3+z^4w^4x^{10}y^3$\\
$\mathcal{M}_{3,3}(3,33)=2z^2w^5x^{10}y^3+z^2w^5x^{12}y^3+z^3w^7x^{14}y^3+2z^3w^6x^{12}y^3+z^2w^6x^{10}y^3+z^3w^7x^{12}y^3$\\
$\mathcal{M}_{3,3}(3,34)=z^2w^5x^{12}y^3+z^3w^7x^{14}y^3+z^3w^6x^{12}y^3+z^2w^7x^{12}y^3$\\
$\mathcal{M}_{3,3}(3,35)=z^2w^5x^{10}y^3+2zw^8x^{10}y^3+w^6x^{10}y^3+zw^6x^{10}y^3$\\
$\mathcal{M}_{3,3}(3,36)=0$\\
$\mathcal{M}_{3,3}(3,37)=2z^2w^4x^{10}y^3+2z^3w^6x^{12}y^3+2zw^5x^{10}y^3+3z^2w^7x^{10}y^3+2z^4w^4x^{10}y^3$\\
$\mathcal{M}_{3,3}(4,1)=0$\\
$\mathcal{M}_{3,3}(4,2)=2z^3w^5x^{10}y^3+z^2w^3x^6y^3+z^3w^6x^{12}y^3+2z^3w^6x^{10}y^3+z^2w^7x^{10}y^3+2z^3w^4x^8y^3+z^2w^5x^8y^3$\\
$\mathcal{M}_{3,3}(4,3)=0$\\
$\mathcal{M}_{3,3}(4,4)=3z^3w^5x^{10}y^3+2z^2w^4x^8y^3+2z^3w^6x^{10}y^3+z^2w^7x^{10}y^3+z^2w^6x^{10}y^3+z^2w^5x^8y^3$\\
$\mathcal{M}_{3,3}(4,5)=0$\\
$\mathcal{M}_{3,3}(4,6)=3z^2w^5x^{10}y^3+z^3w^6x^{12}y^3+z^2w^7x^{10}y^3+2z^2w^6x^{10}y^3$\\
$\mathcal{M}_{3,3}(4,7)=0$\\
$\mathcal{M}_{3,3}(4,8)=z^3w^5x^{10}y^3+z^3w^7x^{14}y^3+z^3w^6x^{12}y^3+z^2w^4x^8y^3$\\
$\mathcal{M}_{3,3}(4,9)=z^2w^5x^{10}y^3+2z^3w^6x^{12}y^3+z^2w^7x^{12}y^3$\\
$\mathcal{M}_{3,3}(4,10)=zw^8x^{10}y^3+zw^5x^8y^3+zw^6x^8y^3+2z^2w^7x^{10}y^3+z^2w^6x^{10}y^3$\\
$\mathcal{M}_{3,3}(4,11)=0$\\
$\mathcal{M}_{3,3}(4,12)=z^4w^6x^{14}y^3+z^3w^7x^{14}y^3+z^3w^6x^{12}y^3+z^5w^6x^{14}y^3+z^4w^5x^{12}y^3+z^3w^8x^{12}y^3+z^3w^7x^{12}y^3$\\
$\mathcal{M}_{3,3}(4,13)=2z^3w^5x^{12}y^3+z^2w^4x^{10}y^3+z^3w^7x^{14}y^3+z^2w^6x^{12}y^3+z^3w^6x^{14}y^3$\\
$\mathcal{M}_{3,3}(4,14)=2z^2w^6x^{14}y^3+z^2w^7x^{14}y^3$\\
$\mathcal{M}_{3,3}(4,15)=z^3w^7x^{14}y^3+z^3w^6x^{12}y^3+z^3w^8x^{12}y^3+z^3w^7x^{12}y^3$\\
$\mathcal{M}_{3,3}(4,16)=0$\\
$\mathcal{M}_{3,3}(4,17)=2z^2w^5x^{10}y^3+z^3w^6x^{12}y^3+z^2w^6x^{10}y^3+z^3w^7x^{12}y^3$\\
$\mathcal{M}_{3,3}(4,18)=z^3w^7x^{14}y^3+z^2w^6x^{12}y^3$\\
$\mathcal{M}_{3,3}(4,19)=zw^8x^{10}y^3+2zw^7x^{10}y^3+2zw^6x^{10}y^3$\\
$\mathcal{M}_{3,3}(4,20)=0$\\
$\mathcal{M}_{3,3}(4,21)=z^2w^5x^{10}y^3+z^3w^7x^{14}y^3+z^3w^6x^{12}y^3$\\
$\mathcal{M}_{3,3}(4,22)=z^2w^7x^{14}y^3$\\
$\mathcal{M}_{3,3}(4,23)=zw^8x^{14}y^3$\\
$\mathcal{M}_{3,3}(4,24)=z^4w^7x^{14}y^3+z^3w^6x^{12}y^3$\\
$\mathcal{M}_{3,3}(4,25)=z^2w^7x^{10}y^3+z^2w^6x^{10}y^3+z^3w^8x^{12}y^3$\\
$\mathcal{M}_{3,3}(4,26)=z^2w^7x^{12}y^3+z^2w^8x^{12}y^3$\\
$\mathcal{M}_{3,3}(4,27)=zw^8x^{10}y^3+zw^7x^{10}y^3+zw^9x^{10}y^3$\\
$\mathcal{M}_{3,3}(4,28)=0$\\
$\mathcal{M}_{3,3}(4,29)=z^2w^4x^{10}y^3+5z^3w^5x^{10}y^3+z^3w^6x^{12}y^3+zw^5x^{10}y^3+2z^2w^4x^8y^3+3z^3w^6x^{10}y^3+2z^2w^7x^{10}y^3+z^4w^4x^{10}y^3+z^2w^5x^8y^3$\\
$\mathcal{M}_{3,3}(4,30)=0$\\
$\mathcal{M}_{3,3}(4,31)=z^2w^5x^{10}y^3+z^2w^4x^{10}y^3+z^3w^5x^{10}y^3+z^3w^6x^{12}y^3+zw^5x^{10}y^3+2z^2w^7x^{10}y^3+z^4w^4x^{10}y^3+z^2w^6x^{10}y^3$\\
$\mathcal{M}_{3,3}(4,32)=z^4w^6x^{14}y^3+z^5w^5x^{12}y^3+z^5w^6x^{14}y^3+z^4w^4x^{10}y^3$\\
$\mathcal{M}_{3,3}(4,33)=2z^2w^5x^{10}y^3+z^2w^5x^{12}y^3+z^3w^7x^{14}y^3+2z^3w^6x^{12}y^3+z^2w^6x^{10}y^3+z^3w^7x^{12}y^3$\\
$\mathcal{M}_{3,3}(4,34)=z^2w^5x^{12}y^3+z^3w^7x^{14}y^3+z^3w^6x^{12}y^3+z^2w^7x^{12}y^3$\\
$\mathcal{M}_{3,3}(4,35)=z^2w^5x^{10}y^3+2zw^8x^{10}y^3+w^6x^{10}y^3+zw^6x^{10}y^3$\\
$\mathcal{M}_{3,3}(4,36)=0$\\
$\mathcal{M}_{3,3}(4,37)=2z^2w^4x^{10}y^3+2z^3w^6x^{12}y^3+2zw^5x^{10}y^3+3z^2w^7x^{10}y^3+2z^4w^4x^{10}y^3$\\
$\mathcal{M}_{3,3}(5,1)=0$\\
$\mathcal{M}_{3,3}(5,2)=2z^3w^5x^{10}y^3+z^2w^3x^6y^3+z^3w^6x^{12}y^3+2z^3w^6x^{10}y^3+z^2w^7x^{10}y^3+2z^3w^4x^8y^3+z^2w^5x^8y^3$\\
$\mathcal{M}_{3,3}(5,3)=0$\\
$\mathcal{M}_{3,3}(5,4)=3z^3w^5x^{10}y^3+2z^2w^4x^8y^3+2z^3w^6x^{10}y^3+z^2w^7x^{10}y^3+z^2w^6x^{10}y^3+z^2w^5x^8y^3$\\
$\mathcal{M}_{3,3}(5,5)=0$\\
$\mathcal{M}_{3,3}(5,6)=3z^2w^5x^{10}y^3+z^3w^6x^{12}y^3+z^2w^7x^{10}y^3+2z^2w^6x^{10}y^3$\\
$\mathcal{M}_{3,3}(5,7)=0$\\
$\mathcal{M}_{3,3}(5,8)=z^3w^5x^{10}y^3+z^3w^7x^{14}y^3+z^3w^6x^{12}y^3+z^2w^4x^8y^3$\\
$\mathcal{M}_{3,3}(5,9)=z^2w^5x^{10}y^3+2z^3w^6x^{12}y^3+z^2w^7x^{12}y^3$\\
$\mathcal{M}_{3,3}(5,10)=zw^8x^{10}y^3+zw^5x^8y^3+zw^6x^8y^3+2z^2w^7x^{10}y^3+z^2w^6x^{10}y^3$\\
$\mathcal{M}_{3,3}(5,11)=0$\\
$\mathcal{M}_{3,3}(5,12)=z^4w^6x^{14}y^3+z^3w^7x^{14}y^3+z^3w^6x^{12}y^3+z^5w^6x^{14}y^3+z^4w^5x^{12}y^3+z^3w^8x^{12}y^3+z^3w^7x^{12}y^3$\\
$\mathcal{M}_{3,3}(5,13)=2z^3w^5x^{12}y^3+z^2w^4x^{10}y^3+z^3w^7x^{14}y^3+z^2w^6x^{12}y^3+z^3w^6x^{14}y^3$\\
$\mathcal{M}_{3,3}(5,14)=2z^2w^6x^{14}y^3+z^2w^7x^{14}y^3$\\
$\mathcal{M}_{3,3}(5,15)=z^3w^7x^{14}y^3+z^3w^6x^{12}y^3+z^3w^8x^{12}y^3+z^3w^7x^{12}y^3$\\
$\mathcal{M}_{3,3}(5,16)=0$\\
$\mathcal{M}_{3,3}(5,17)=2z^2w^5x^{10}y^3+z^3w^6x^{12}y^3+z^2w^6x^{10}y^3+z^3w^7x^{12}y^3$\\
$\mathcal{M}_{3,3}(5,18)=z^3w^7x^{14}y^3+z^2w^6x^{12}y^3$\\
$\mathcal{M}_{3,3}(5,19)=zw^8x^{10}y^3+2zw^7x^{10}y^3+2zw^6x^{10}y^3$\\
$\mathcal{M}_{3,3}(5,20)=0$\\
$\mathcal{M}_{3,3}(5,21)=z^2w^5x^{10}y^3+z^3w^7x^{14}y^3+z^3w^6x^{12}y^3$\\
$\mathcal{M}_{3,3}(5,22)=z^2w^7x^{14}y^3$\\
$\mathcal{M}_{3,3}(5,23)=zw^8x^{14}y^3$\\
$\mathcal{M}_{3,3}(5,24)=z^4w^7x^{14}y^3+z^3w^6x^{12}y^3$\\
$\mathcal{M}_{3,3}(5,25)=z^2w^7x^{10}y^3+z^2w^6x^{10}y^3+z^3w^8x^{12}y^3$\\
$\mathcal{M}_{3,3}(5,26)=z^2w^7x^{12}y^3+z^2w^8x^{12}y^3$\\
$\mathcal{M}_{3,3}(5,27)=zw^8x^{10}y^3+zw^7x^{10}y^3+zw^9x^{10}y^3$\\
$\mathcal{M}_{3,3}(5,28)=0$\\
$\mathcal{M}_{3,3}(5,29)=z^2w^4x^{10}y^3+5z^3w^5x^{10}y^3+z^3w^6x^{12}y^3+zw^5x^{10}y^3+2z^2w^4x^8y^3+3z^3w^6x^{10}y^3+2z^2w^7x^{10}y^3+z^4w^4x^{10}y^3+z^2w^5x^8y^3$\\
$\mathcal{M}_{3,3}(5,30)=0$\\
$\mathcal{M}_{3,3}(5,31)=z^2w^5x^{10}y^3+z^2w^4x^{10}y^3+z^3w^5x^{10}y^3+z^3w^6x^{12}y^3+zw^5x^{10}y^3+2z^2w^7x^{10}y^3+z^4w^4x^{10}y^3+z^2w^6x^{10}y^3$\\
$\mathcal{M}_{3,3}(5,32)=z^4w^6x^{14}y^3+z^5w^5x^{12}y^3+z^5w^6x^{14}y^3+z^4w^4x^{10}y^3$\\
$\mathcal{M}_{3,3}(5,33)=2z^2w^5x^{10}y^3+z^2w^5x^{12}y^3+z^3w^7x^{14}y^3+2z^3w^6x^{12}y^3+z^2w^6x^{10}y^3+z^3w^7x^{12}y^3$\\
$\mathcal{M}_{3,3}(5,34)=z^2w^5x^{12}y^3+z^3w^7x^{14}y^3+z^3w^6x^{12}y^3+z^2w^7x^{12}y^3$\\
$\mathcal{M}_{3,3}(5,35)=z^2w^5x^{10}y^3+2zw^8x^{10}y^3+w^6x^{10}y^3+zw^6x^{10}y^3$\\
$\mathcal{M}_{3,3}(5,36)=0$\\
$\mathcal{M}_{3,3}(5,37)=2z^2w^4x^{10}y^3+2z^3w^6x^{12}y^3+2zw^5x^{10}y^3+3z^2w^7x^{10}y^3+2z^4w^4x^{10}y^3$\\
$\mathcal{M}_{3,3}(6,1)=0$\\
$\mathcal{M}_{3,3}(6,2)=2z^3w^5x^{10}y^3+z^2w^3x^6y^3+z^3w^6x^{12}y^3+2z^3w^6x^{10}y^3+z^2w^7x^{10}y^3+2z^3w^4x^8y^3+z^2w^5x^8y^3$\\
$\mathcal{M}_{3,3}(6,3)=0$\\
$\mathcal{M}_{3,3}(6,4)=3z^3w^5x^{10}y^3+2z^2w^4x^8y^3+2z^3w^6x^{10}y^3+z^2w^7x^{10}y^3+z^2w^6x^{10}y^3+z^2w^5x^8y^3$\\
$\mathcal{M}_{3,3}(6,5)=0$\\
$\mathcal{M}_{3,3}(6,6)=3z^2w^5x^{10}y^3+z^3w^6x^{12}y^3+z^2w^7x^{10}y^3+2z^2w^6x^{10}y^3$\\
$\mathcal{M}_{3,3}(6,7)=0$\\
$\mathcal{M}_{3,3}(6,8)=z^3w^5x^{10}y^3+z^3w^7x^{14}y^3+z^3w^6x^{12}y^3+z^2w^4x^8y^3$\\
$\mathcal{M}_{3,3}(6,9)=z^2w^5x^{10}y^3+2z^3w^6x^{12}y^3+z^2w^7x^{12}y^3$\\
$\mathcal{M}_{3,3}(6,10)=zw^8x^{10}y^3+zw^5x^8y^3+zw^6x^8y^3+2z^2w^7x^{10}y^3+z^2w^6x^{10}y^3$\\
$\mathcal{M}_{3,3}(6,11)=0$\\
$\mathcal{M}_{3,3}(6,12)=z^4w^6x^{14}y^3+z^3w^7x^{14}y^3+z^3w^6x^{12}y^3+z^5w^6x^{14}y^3+z^4w^5x^{12}y^3+z^3w^8x^{12}y^3+z^3w^7x^{12}y^3$\\
$\mathcal{M}_{3,3}(6,13)=2z^3w^5x^{12}y^3+z^2w^4x^{10}y^3+z^3w^7x^{14}y^3+z^2w^6x^{12}y^3+z^3w^6x^{14}y^3$\\
$\mathcal{M}_{3,3}(6,14)=2z^2w^6x^{14}y^3+z^2w^7x^{14}y^3$\\
$\mathcal{M}_{3,3}(6,15)=z^3w^7x^{14}y^3+z^3w^6x^{12}y^3+z^3w^8x^{12}y^3+z^3w^7x^{12}y^3$\\
$\mathcal{M}_{3,3}(6,16)=0$\\
$\mathcal{M}_{3,3}(6,17)=2z^2w^5x^{10}y^3+z^3w^6x^{12}y^3+z^2w^6x^{10}y^3+z^3w^7x^{12}y^3$\\
$\mathcal{M}_{3,3}(6,18)=z^3w^7x^{14}y^3+z^2w^6x^{12}y^3$\\
$\mathcal{M}_{3,3}(6,19)=zw^8x^{10}y^3+2zw^7x^{10}y^3+2zw^6x^{10}y^3$\\
$\mathcal{M}_{3,3}(6,20)=0$\\
$\mathcal{M}_{3,3}(6,21)=z^2w^5x^{10}y^3+z^3w^7x^{14}y^3+z^3w^6x^{12}y^3$\\
$\mathcal{M}_{3,3}(6,22)=z^2w^7x^{14}y^3$\\
$\mathcal{M}_{3,3}(6,23)=zw^8x^{14}y^3$\\
$\mathcal{M}_{3,3}(6,24)=z^4w^7x^{14}y^3+z^3w^6x^{12}y^3$\\
$\mathcal{M}_{3,3}(6,25)=z^2w^7x^{10}y^3+z^2w^6x^{10}y^3+z^3w^8x^{12}y^3$\\
$\mathcal{M}_{3,3}(6,26)=z^2w^7x^{12}y^3+z^2w^8x^{12}y^3$\\
$\mathcal{M}_{3,3}(6,27)=zw^8x^{10}y^3+zw^7x^{10}y^3+zw^9x^{10}y^3$\\
$\mathcal{M}_{3,3}(6,28)=0$\\
$\mathcal{M}_{3,3}(6,29)=z^2w^4x^{10}y^3+5z^3w^5x^{10}y^3+z^3w^6x^{12}y^3+zw^5x^{10}y^3+2z^2w^4x^8y^3+3z^3w^6x^{10}y^3+2z^2w^7x^{10}y^3+z^4w^4x^{10}y^3+z^2w^5x^8y^3$\\
$\mathcal{M}_{3,3}(6,30)=0$\\
$\mathcal{M}_{3,3}(6,31)=z^2w^5x^{10}y^3+z^2w^4x^{10}y^3+z^3w^5x^{10}y^3+z^3w^6x^{12}y^3+zw^5x^{10}y^3+2z^2w^7x^{10}y^3+z^4w^4x^{10}y^3+z^2w^6x^{10}y^3$\\
$\mathcal{M}_{3,3}(6,32)=z^4w^6x^{14}y^3+z^5w^5x^{12}y^3+z^5w^6x^{14}y^3+z^4w^4x^{10}y^3$\\
$\mathcal{M}_{3,3}(6,33)=2z^2w^5x^{10}y^3+z^2w^5x^{12}y^3+z^3w^7x^{14}y^3+2z^3w^6x^{12}y^3+z^2w^6x^{10}y^3+z^3w^7x^{12}y^3$\\
$\mathcal{M}_{3,3}(6,34)=z^2w^5x^{12}y^3+z^3w^7x^{14}y^3+z^3w^6x^{12}y^3+z^2w^7x^{12}y^3$\\
$\mathcal{M}_{3,3}(6,35)=z^2w^5x^{10}y^3+2zw^8x^{10}y^3+w^6x^{10}y^3+zw^6x^{10}y^3$\\
$\mathcal{M}_{3,3}(6,36)=0$\\
$\mathcal{M}_{3,3}(6,37)=2z^2w^4x^{10}y^3+2z^3w^6x^{12}y^3+2zw^5x^{10}y^3+3z^2w^7x^{10}y^3+2z^4w^4x^{10}y^3$\\
$\mathcal{M}_{3,3}(7,1)=0$\\
$\mathcal{M}_{3,3}(7,2)=2z^3w^5x^{10}y^3+z^2w^3x^6y^3+z^3w^6x^{12}y^3+2z^3w^6x^{10}y^3+z^2w^7x^{10}y^3+2z^3w^4x^8y^3+z^2w^5x^8y^3$\\
$\mathcal{M}_{3,3}(7,3)=0$\\
$\mathcal{M}_{3,3}(7,4)=3z^3w^5x^{10}y^3+2z^2w^4x^8y^3+2z^3w^6x^{10}y^3+z^2w^7x^{10}y^3+z^2w^6x^{10}y^3+z^2w^5x^8y^3$\\
$\mathcal{M}_{3,3}(7,5)=0$\\
$\mathcal{M}_{3,3}(7,6)=3z^2w^5x^{10}y^3+z^3w^6x^{12}y^3+z^2w^7x^{10}y^3+2z^2w^6x^{10}y^3$\\
$\mathcal{M}_{3,3}(7,7)=0$\\
$\mathcal{M}_{3,3}(7,8)=z^3w^5x^{10}y^3+z^3w^7x^{14}y^3+z^3w^6x^{12}y^3+z^2w^4x^8y^3$\\
$\mathcal{M}_{3,3}(7,9)=z^2w^5x^{10}y^3+2z^3w^6x^{12}y^3+z^2w^7x^{12}y^3$\\
$\mathcal{M}_{3,3}(7,10)=zw^8x^{10}y^3+zw^5x^8y^3+zw^6x^8y^3+2z^2w^7x^{10}y^3+z^2w^6x^{10}y^3$\\
$\mathcal{M}_{3,3}(7,11)=0$\\
$\mathcal{M}_{3,3}(7,12)=z^4w^6x^{14}y^3+z^3w^7x^{14}y^3+z^3w^6x^{12}y^3+z^5w^6x^{14}y^3+z^4w^5x^{12}y^3+z^3w^8x^{12}y^3+z^3w^7x^{12}y^3$\\
$\mathcal{M}_{3,3}(7,13)=2z^3w^5x^{12}y^3+z^2w^4x^{10}y^3+z^3w^7x^{14}y^3+z^2w^6x^{12}y^3+z^3w^6x^{14}y^3$\\
$\mathcal{M}_{3,3}(7,14)=2z^2w^6x^{14}y^3+z^2w^7x^{14}y^3$\\
$\mathcal{M}_{3,3}(7,15)=z^3w^7x^{14}y^3+z^3w^6x^{12}y^3+z^3w^8x^{12}y^3+z^3w^7x^{12}y^3$\\
$\mathcal{M}_{3,3}(7,16)=0$\\
$\mathcal{M}_{3,3}(7,17)=2z^2w^5x^{10}y^3+z^3w^6x^{12}y^3+z^2w^6x^{10}y^3+z^3w^7x^{12}y^3$\\
$\mathcal{M}_{3,3}(7,18)=z^3w^7x^{14}y^3+z^2w^6x^{12}y^3$\\
$\mathcal{M}_{3,3}(7,19)=zw^8x^{10}y^3+2zw^7x^{10}y^3+2zw^6x^{10}y^3$\\
$\mathcal{M}_{3,3}(7,20)=0$\\
$\mathcal{M}_{3,3}(7,21)=z^2w^5x^{10}y^3+z^3w^7x^{14}y^3+z^3w^6x^{12}y^3$\\
$\mathcal{M}_{3,3}(7,22)=z^2w^7x^{14}y^3$\\
$\mathcal{M}_{3,3}(7,23)=zw^8x^{14}y^3$\\
$\mathcal{M}_{3,3}(7,24)=z^4w^7x^{14}y^3+z^3w^6x^{12}y^3$\\
$\mathcal{M}_{3,3}(7,25)=z^2w^7x^{10}y^3+z^2w^6x^{10}y^3+z^3w^8x^{12}y^3$\\
$\mathcal{M}_{3,3}(7,26)=z^2w^7x^{12}y^3+z^2w^8x^{12}y^3$\\
$\mathcal{M}_{3,3}(7,27)=zw^8x^{10}y^3+zw^7x^{10}y^3+zw^9x^{10}y^3$\\
$\mathcal{M}_{3,3}(7,28)=0$\\
$\mathcal{M}_{3,3}(7,29)=z^2w^4x^{10}y^3+5z^3w^5x^{10}y^3+z^3w^6x^{12}y^3+zw^5x^{10}y^3+2z^2w^4x^8y^3+3z^3w^6x^{10}y^3+2z^2w^7x^{10}y^3+z^4w^4x^{10}y^3+z^2w^5x^8y^3$\\
$\mathcal{M}_{3,3}(7,30)=0$\\
$\mathcal{M}_{3,3}(7,31)=z^2w^5x^{10}y^3+z^2w^4x^{10}y^3+z^3w^5x^{10}y^3+z^3w^6x^{12}y^3+zw^5x^{10}y^3+2z^2w^7x^{10}y^3+z^4w^4x^{10}y^3+z^2w^6x^{10}y^3$\\
$\mathcal{M}_{3,3}(7,32)=z^4w^6x^{14}y^3+z^5w^5x^{12}y^3+z^5w^6x^{14}y^3+z^4w^4x^{10}y^3$\\
$\mathcal{M}_{3,3}(7,33)=2z^2w^5x^{10}y^3+z^2w^5x^{12}y^3+z^3w^7x^{14}y^3+2z^3w^6x^{12}y^3+z^2w^6x^{10}y^3+z^3w^7x^{12}y^3$\\
$\mathcal{M}_{3,3}(7,34)=z^2w^5x^{12}y^3+z^3w^7x^{14}y^3+z^3w^6x^{12}y^3+z^2w^7x^{12}y^3$\\
$\mathcal{M}_{3,3}(7,35)=z^2w^5x^{10}y^3+2zw^8x^{10}y^3+w^6x^{10}y^3+zw^6x^{10}y^3$\\
$\mathcal{M}_{3,3}(7,36)=0$\\
$\mathcal{M}_{3,3}(7,37)=2z^2w^4x^{10}y^3+2z^3w^6x^{12}y^3+2zw^5x^{10}y^3+3z^2w^7x^{10}y^3+2z^4w^4x^{10}y^3$\\
$\mathcal{M}_{3,3}(8,1)=0$\\
$\mathcal{M}_{3,3}(8,2)=2z^3w^5x^{10}y^3+z^2w^3x^6y^3+z^3w^6x^{12}y^3+2z^3w^6x^{10}y^3+z^2w^7x^{10}y^3+2z^3w^4x^8y^3+z^2w^5x^8y^3$\\
$\mathcal{M}_{3,3}(8,3)=0$\\
$\mathcal{M}_{3,3}(8,4)=3z^3w^5x^{10}y^3+2z^2w^4x^8y^3+2z^3w^6x^{10}y^3+z^2w^7x^{10}y^3+z^2w^6x^{10}y^3+z^2w^5x^8y^3$\\
$\mathcal{M}_{3,3}(8,5)=0$\\
$\mathcal{M}_{3,3}(8,6)=3z^2w^5x^{10}y^3+z^3w^6x^{12}y^3+z^2w^7x^{10}y^3+2z^2w^6x^{10}y^3$\\
$\mathcal{M}_{3,3}(8,7)=0$\\
$\mathcal{M}_{3,3}(8,8)=z^3w^5x^{10}y^3+z^3w^7x^{14}y^3+z^3w^6x^{12}y^3+z^2w^4x^8y^3$\\
$\mathcal{M}_{3,3}(8,9)=z^2w^5x^{10}y^3+2z^3w^6x^{12}y^3+z^2w^7x^{12}y^3$\\
$\mathcal{M}_{3,3}(8,10)=zw^8x^{10}y^3+zw^5x^8y^3+zw^6x^8y^3+2z^2w^7x^{10}y^3+z^2w^6x^{10}y^3$\\
$\mathcal{M}_{3,3}(8,11)=0$\\
$\mathcal{M}_{3,3}(8,12)=z^4w^6x^{14}y^3+z^3w^7x^{14}y^3+z^3w^6x^{12}y^3+z^5w^6x^{14}y^3+z^4w^5x^{12}y^3+z^3w^8x^{12}y^3+z^3w^7x^{12}y^3$\\
$\mathcal{M}_{3,3}(8,13)=2z^3w^5x^{12}y^3+z^2w^4x^{10}y^3+z^3w^7x^{14}y^3+z^2w^6x^{12}y^3+z^3w^6x^{14}y^3$\\
$\mathcal{M}_{3,3}(8,14)=2z^2w^6x^{14}y^3+z^2w^7x^{14}y^3$\\
$\mathcal{M}_{3,3}(8,15)=z^3w^7x^{14}y^3+z^3w^6x^{12}y^3+z^3w^8x^{12}y^3+z^3w^7x^{12}y^3$\\
$\mathcal{M}_{3,3}(8,16)=0$\\
$\mathcal{M}_{3,3}(8,17)=2z^2w^5x^{10}y^3+z^3w^6x^{12}y^3+z^2w^6x^{10}y^3+z^3w^7x^{12}y^3$\\
$\mathcal{M}_{3,3}(8,18)=z^3w^7x^{14}y^3+z^2w^6x^{12}y^3$\\
$\mathcal{M}_{3,3}(8,19)=zw^8x^{10}y^3+2zw^7x^{10}y^3+2zw^6x^{10}y^3$\\
$\mathcal{M}_{3,3}(8,20)=0$\\
$\mathcal{M}_{3,3}(8,21)=z^2w^5x^{10}y^3+z^3w^7x^{14}y^3+z^3w^6x^{12}y^3$\\
$\mathcal{M}_{3,3}(8,22)=z^2w^7x^{14}y^3$\\
$\mathcal{M}_{3,3}(8,23)=zw^8x^{14}y^3$\\
$\mathcal{M}_{3,3}(8,24)=z^4w^7x^{14}y^3+z^3w^6x^{12}y^3$\\
$\mathcal{M}_{3,3}(8,25)=z^2w^7x^{10}y^3+z^2w^6x^{10}y^3+z^3w^8x^{12}y^3$\\
$\mathcal{M}_{3,3}(8,26)=z^2w^7x^{12}y^3+z^2w^8x^{12}y^3$\\
$\mathcal{M}_{3,3}(8,27)=zw^8x^{10}y^3+zw^7x^{10}y^3+zw^9x^{10}y^3$\\
$\mathcal{M}_{3,3}(8,28)=0$\\
$\mathcal{M}_{3,3}(8,29)=z^2w^4x^{10}y^3+5z^3w^5x^{10}y^3+z^3w^6x^{12}y^3+zw^5x^{10}y^3+2z^2w^4x^8y^3+3z^3w^6x^{10}y^3+2z^2w^7x^{10}y^3+z^4w^4x^{10}y^3+z^2w^5x^8y^3$\\
$\mathcal{M}_{3,3}(8,30)=0$\\
$\mathcal{M}_{3,3}(8,31)=z^2w^5x^{10}y^3+z^2w^4x^{10}y^3+z^3w^5x^{10}y^3+z^3w^6x^{12}y^3+zw^5x^{10}y^3+2z^2w^7x^{10}y^3+z^4w^4x^{10}y^3+z^2w^6x^{10}y^3$\\
$\mathcal{M}_{3,3}(8,32)=z^4w^6x^{14}y^3+z^5w^5x^{12}y^3+z^5w^6x^{14}y^3+z^4w^4x^{10}y^3$\\
$\mathcal{M}_{3,3}(8,33)=2z^2w^5x^{10}y^3+z^2w^5x^{12}y^3+z^3w^7x^{14}y^3+2z^3w^6x^{12}y^3+z^2w^6x^{10}y^3+z^3w^7x^{12}y^3$\\
$\mathcal{M}_{3,3}(8,34)=z^2w^5x^{12}y^3+z^3w^7x^{14}y^3+z^3w^6x^{12}y^3+z^2w^7x^{12}y^3$\\
$\mathcal{M}_{3,3}(8,35)=z^2w^5x^{10}y^3+2zw^8x^{10}y^3+w^6x^{10}y^3+zw^6x^{10}y^3$\\
$\mathcal{M}_{3,3}(8,36)=0$\\
$\mathcal{M}_{3,3}(8,37)=2z^2w^4x^{10}y^3+2z^3w^6x^{12}y^3+2zw^5x^{10}y^3+3z^2w^7x^{10}y^3+2z^4w^4x^{10}y^3$\\
$\mathcal{M}_{3,3}(9,1)=0$\\
$\mathcal{M}_{3,3}(9,2)=2z^3w^5x^{10}y^3+z^2w^3x^6y^3+z^3w^6x^{12}y^3+2z^3w^6x^{10}y^3+z^2w^7x^{10}y^3+2z^3w^4x^8y^3+z^2w^5x^8y^3$\\
$\mathcal{M}_{3,3}(9,3)=0$\\
$\mathcal{M}_{3,3}(9,4)=3z^3w^5x^{10}y^3+2z^2w^4x^8y^3+2z^3w^6x^{10}y^3+z^2w^7x^{10}y^3+z^2w^6x^{10}y^3+z^2w^5x^8y^3$\\
$\mathcal{M}_{3,3}(9,5)=0$\\
$\mathcal{M}_{3,3}(9,6)=3z^2w^5x^{10}y^3+z^3w^6x^{12}y^3+z^2w^7x^{10}y^3+2z^2w^6x^{10}y^3$\\
$\mathcal{M}_{3,3}(9,7)=0$\\
$\mathcal{M}_{3,3}(9,8)=z^3w^5x^{10}y^3+z^3w^7x^{14}y^3+z^3w^6x^{12}y^3+z^2w^4x^8y^3$\\
$\mathcal{M}_{3,3}(9,9)=z^2w^5x^{10}y^3+2z^3w^6x^{12}y^3+z^2w^7x^{12}y^3$\\
$\mathcal{M}_{3,3}(9,10)=zw^8x^{10}y^3+zw^5x^8y^3+zw^6x^8y^3+2z^2w^7x^{10}y^3+z^2w^6x^{10}y^3$\\
$\mathcal{M}_{3,3}(9,11)=0$\\
$\mathcal{M}_{3,3}(9,12)=z^4w^6x^{14}y^3+z^3w^7x^{14}y^3+z^3w^6x^{12}y^3+z^5w^6x^{14}y^3+z^4w^5x^{12}y^3+z^3w^8x^{12}y^3+z^3w^7x^{12}y^3$\\
$\mathcal{M}_{3,3}(9,13)=2z^3w^5x^{12}y^3+z^2w^4x^{10}y^3+z^3w^7x^{14}y^3+z^2w^6x^{12}y^3+z^3w^6x^{14}y^3$\\
$\mathcal{M}_{3,3}(9,14)=2z^2w^6x^{14}y^3+z^2w^7x^{14}y^3$\\
$\mathcal{M}_{3,3}(9,15)=z^3w^7x^{14}y^3+z^3w^6x^{12}y^3+z^3w^8x^{12}y^3+z^3w^7x^{12}y^3$\\
$\mathcal{M}_{3,3}(9,16)=0$\\
$\mathcal{M}_{3,3}(9,17)=2z^2w^5x^{10}y^3+z^3w^6x^{12}y^3+z^2w^6x^{10}y^3+z^3w^7x^{12}y^3$\\
$\mathcal{M}_{3,3}(9,18)=z^3w^7x^{14}y^3+z^2w^6x^{12}y^3$\\
$\mathcal{M}_{3,3}(9,19)=zw^8x^{10}y^3+2zw^7x^{10}y^3+2zw^6x^{10}y^3$\\
$\mathcal{M}_{3,3}(9,20)=0$\\
$\mathcal{M}_{3,3}(9,21)=z^2w^5x^{10}y^3+z^3w^7x^{14}y^3+z^3w^6x^{12}y^3$\\
$\mathcal{M}_{3,3}(9,22)=z^2w^7x^{14}y^3$\\
$\mathcal{M}_{3,3}(9,23)=zw^8x^{14}y^3$\\
$\mathcal{M}_{3,3}(9,24)=z^4w^7x^{14}y^3+z^3w^6x^{12}y^3$\\
$\mathcal{M}_{3,3}(9,25)=z^2w^7x^{10}y^3+z^2w^6x^{10}y^3+z^3w^8x^{12}y^3$\\
$\mathcal{M}_{3,3}(9,26)=z^2w^7x^{12}y^3+z^2w^8x^{12}y^3$\\
$\mathcal{M}_{3,3}(9,27)=zw^8x^{10}y^3+zw^7x^{10}y^3+zw^9x^{10}y^3$\\
$\mathcal{M}_{3,3}(9,28)=0$\\
$\mathcal{M}_{3,3}(9,29)=z^2w^4x^{10}y^3+5z^3w^5x^{10}y^3+z^3w^6x^{12}y^3+zw^5x^{10}y^3+2z^2w^4x^8y^3+3z^3w^6x^{10}y^3+2z^2w^7x^{10}y^3+z^4w^4x^{10}y^3+z^2w^5x^8y^3$\\
$\mathcal{M}_{3,3}(9,30)=0$\\
$\mathcal{M}_{3,3}(9,31)=z^2w^5x^{10}y^3+z^2w^4x^{10}y^3+z^3w^5x^{10}y^3+z^3w^6x^{12}y^3+zw^5x^{10}y^3+2z^2w^7x^{10}y^3+z^4w^4x^{10}y^3+z^2w^6x^{10}y^3$\\
$\mathcal{M}_{3,3}(9,32)=z^4w^6x^{14}y^3+z^5w^5x^{12}y^3+z^5w^6x^{14}y^3+z^4w^4x^{10}y^3$\\
$\mathcal{M}_{3,3}(9,33)=2z^2w^5x^{10}y^3+z^2w^5x^{12}y^3+z^3w^7x^{14}y^3+2z^3w^6x^{12}y^3+z^2w^6x^{10}y^3+z^3w^7x^{12}y^3$\\
$\mathcal{M}_{3,3}(9,34)=z^2w^5x^{12}y^3+z^3w^7x^{14}y^3+z^3w^6x^{12}y^3+z^2w^7x^{12}y^3$\\
$\mathcal{M}_{3,3}(9,35)=z^2w^5x^{10}y^3+2zw^8x^{10}y^3+w^6x^{10}y^3+zw^6x^{10}y^3$\\
$\mathcal{M}_{3,3}(9,36)=0$\\
$\mathcal{M}_{3,3}(9,37)=2z^2w^4x^{10}y^3+2z^3w^6x^{12}y^3+2zw^5x^{10}y^3+3z^2w^7x^{10}y^3+2z^4w^4x^{10}y^3$\\
$\mathcal{M}_{3,3}(10,1)=0$\\
$\mathcal{M}_{3,3}(10,2)=2z^3w^5x^{10}y^3+z^2w^3x^6y^3+z^3w^6x^{12}y^3+2z^3w^6x^{10}y^3+z^2w^7x^{10}y^3+2z^3w^4x^8y^3+z^2w^5x^8y^3$\\
$\mathcal{M}_{3,3}(10,3)=0$\\
$\mathcal{M}_{3,3}(10,4)=3z^3w^5x^{10}y^3+2z^2w^4x^8y^3+2z^3w^6x^{10}y^3+z^2w^7x^{10}y^3+z^2w^6x^{10}y^3+z^2w^5x^8y^3$\\
$\mathcal{M}_{3,3}(10,5)=0$\\
$\mathcal{M}_{3,3}(10,6)=3z^2w^5x^{10}y^3+z^3w^6x^{12}y^3+z^2w^7x^{10}y^3+2z^2w^6x^{10}y^3$\\
$\mathcal{M}_{3,3}(10,7)=0$\\
$\mathcal{M}_{3,3}(10,8)=z^3w^5x^{10}y^3+z^3w^7x^{14}y^3+z^3w^6x^{12}y^3+z^2w^4x^8y^3$\\
$\mathcal{M}_{3,3}(10,9)=z^2w^5x^{10}y^3+2z^3w^6x^{12}y^3+z^2w^7x^{12}y^3$\\
$\mathcal{M}_{3,3}(10,10)=zw^8x^{10}y^3+zw^5x^8y^3+zw^6x^8y^3+2z^2w^7x^{10}y^3+z^2w^6x^{10}y^3$\\
$\mathcal{M}_{3,3}(10,11)=0$\\
$\mathcal{M}_{3,3}(10,12)=z^4w^6x^{14}y^3+z^3w^7x^{14}y^3+z^3w^6x^{12}y^3+z^5w^6x^{14}y^3+z^4w^5x^{12}y^3+z^3w^8x^{12}y^3+z^3w^7x^{12}y^3$\\
$\mathcal{M}_{3,3}(10,13)=2z^3w^5x^{12}y^3+z^2w^4x^{10}y^3+z^3w^7x^{14}y^3+z^2w^6x^{12}y^3+z^3w^6x^{14}y^3$\\
$\mathcal{M}_{3,3}(10,14)=2z^2w^6x^{14}y^3+z^2w^7x^{14}y^3$\\
$\mathcal{M}_{3,3}(10,15)=z^3w^7x^{14}y^3+z^3w^6x^{12}y^3+z^3w^8x^{12}y^3+z^3w^7x^{12}y^3$\\
$\mathcal{M}_{3,3}(10,16)=0$\\
$\mathcal{M}_{3,3}(10,17)=2z^2w^5x^{10}y^3+z^3w^6x^{12}y^3+z^2w^6x^{10}y^3+z^3w^7x^{12}y^3$\\
$\mathcal{M}_{3,3}(10,18)=z^3w^7x^{14}y^3+z^2w^6x^{12}y^3$\\
$\mathcal{M}_{3,3}(10,19)=zw^8x^{10}y^3+2zw^7x^{10}y^3+2zw^6x^{10}y^3$\\
$\mathcal{M}_{3,3}(10,20)=0$\\
$\mathcal{M}_{3,3}(10,21)=z^2w^5x^{10}y^3+z^3w^7x^{14}y^3+z^3w^6x^{12}y^3$\\
$\mathcal{M}_{3,3}(10,22)=z^2w^7x^{14}y^3$\\
$\mathcal{M}_{3,3}(10,23)=zw^8x^{14}y^3$\\
$\mathcal{M}_{3,3}(10,24)=z^4w^7x^{14}y^3+z^3w^6x^{12}y^3$\\
$\mathcal{M}_{3,3}(10,25)=z^2w^7x^{10}y^3+z^2w^6x^{10}y^3+z^3w^8x^{12}y^3$\\
$\mathcal{M}_{3,3}(10,26)=z^2w^7x^{12}y^3+z^2w^8x^{12}y^3$\\
$\mathcal{M}_{3,3}(10,27)=zw^8x^{10}y^3+zw^7x^{10}y^3+zw^9x^{10}y^3$\\
$\mathcal{M}_{3,3}(10,28)=0$\\
$\mathcal{M}_{3,3}(10,29)=z^2w^4x^{10}y^3+5z^3w^5x^{10}y^3+z^3w^6x^{12}y^3+zw^5x^{10}y^3+2z^2w^4x^8y^3+3z^3w^6x^{10}y^3+2z^2w^7x^{10}y^3+z^4w^4x^{10}y^3+z^2w^5x^8y^3$\\
$\mathcal{M}_{3,3}(10,30)=0$\\
$\mathcal{M}_{3,3}(10,31)=z^2w^5x^{10}y^3+z^2w^4x^{10}y^3+z^3w^5x^{10}y^3+z^3w^6x^{12}y^3+zw^5x^{10}y^3+2z^2w^7x^{10}y^3+z^4w^4x^{10}y^3+z^2w^6x^{10}y^3$\\
$\mathcal{M}_{3,3}(10,32)=z^4w^6x^{14}y^3+z^5w^5x^{12}y^3+z^5w^6x^{14}y^3+z^4w^4x^{10}y^3$\\
$\mathcal{M}_{3,3}(10,33)=2z^2w^5x^{10}y^3+z^2w^5x^{12}y^3+z^3w^7x^{14}y^3+2z^3w^6x^{12}y^3+z^2w^6x^{10}y^3+z^3w^7x^{12}y^3$\\
$\mathcal{M}_{3,3}(10,34)=z^2w^5x^{12}y^3+z^3w^7x^{14}y^3+z^3w^6x^{12}y^3+z^2w^7x^{12}y^3$\\
$\mathcal{M}_{3,3}(10,35)=z^2w^5x^{10}y^3+2zw^8x^{10}y^3+w^6x^{10}y^3+zw^6x^{10}y^3$\\
$\mathcal{M}_{3,3}(10,36)=0$\\
$\mathcal{M}_{3,3}(10,37)=2z^2w^4x^{10}y^3+2z^3w^6x^{12}y^3+2zw^5x^{10}y^3+3z^2w^7x^{10}y^3+2z^4w^4x^{10}y^3$\\
$\mathcal{M}_{3,3}(11,1)=0$\\
$\mathcal{M}_{3,3}(11,2)=2z^3w^5x^{10}y^3+z^2w^3x^6y^3+z^3w^6x^{12}y^3+2z^3w^6x^{10}y^3+z^2w^7x^{10}y^3+2z^3w^4x^8y^3+z^2w^5x^8y^3$\\
$\mathcal{M}_{3,3}(11,3)=0$\\
$\mathcal{M}_{3,3}(11,4)=3z^3w^5x^{10}y^3+2z^2w^4x^8y^3+2z^3w^6x^{10}y^3+z^2w^7x^{10}y^3+z^2w^6x^{10}y^3+z^2w^5x^8y^3$\\
$\mathcal{M}_{3,3}(11,5)=0$\\
$\mathcal{M}_{3,3}(11,6)=3z^2w^5x^{10}y^3+z^3w^6x^{12}y^3+z^2w^7x^{10}y^3+2z^2w^6x^{10}y^3$\\
$\mathcal{M}_{3,3}(11,7)=0$\\
$\mathcal{M}_{3,3}(11,8)=z^3w^5x^{10}y^3+z^3w^7x^{14}y^3+z^3w^6x^{12}y^3+z^2w^4x^8y^3$\\
$\mathcal{M}_{3,3}(11,9)=z^2w^5x^{10}y^3+2z^3w^6x^{12}y^3+z^2w^7x^{12}y^3$\\
$\mathcal{M}_{3,3}(11,10)=zw^8x^{10}y^3+zw^5x^8y^3+zw^6x^8y^3+2z^2w^7x^{10}y^3+z^2w^6x^{10}y^3$\\
$\mathcal{M}_{3,3}(11,11)=0$\\
$\mathcal{M}_{3,3}(11,12)=z^4w^6x^{14}y^3+z^3w^7x^{14}y^3+z^3w^6x^{12}y^3+z^5w^6x^{14}y^3+z^4w^5x^{12}y^3+z^3w^8x^{12}y^3+z^3w^7x^{12}y^3$\\
$\mathcal{M}_{3,3}(11,13)=2z^3w^5x^{12}y^3+z^2w^4x^{10}y^3+z^3w^7x^{14}y^3+z^2w^6x^{12}y^3+z^3w^6x^{14}y^3$\\
$\mathcal{M}_{3,3}(11,14)=2z^2w^6x^{14}y^3+z^2w^7x^{14}y^3$\\
$\mathcal{M}_{3,3}(11,15)=z^3w^7x^{14}y^3+z^3w^6x^{12}y^3+z^3w^8x^{12}y^3+z^3w^7x^{12}y^3$\\
$\mathcal{M}_{3,3}(11,16)=0$\\
$\mathcal{M}_{3,3}(11,17)=2z^2w^5x^{10}y^3+z^3w^6x^{12}y^3+z^2w^6x^{10}y^3+z^3w^7x^{12}y^3$\\
$\mathcal{M}_{3,3}(11,18)=z^3w^7x^{14}y^3+z^2w^6x^{12}y^3$\\
$\mathcal{M}_{3,3}(11,19)=zw^8x^{10}y^3+2zw^7x^{10}y^3+2zw^6x^{10}y^3$\\
$\mathcal{M}_{3,3}(11,20)=0$\\
$\mathcal{M}_{3,3}(11,21)=z^2w^5x^{10}y^3+z^3w^7x^{14}y^3+z^3w^6x^{12}y^3$\\
$\mathcal{M}_{3,3}(11,22)=z^2w^7x^{14}y^3$\\
$\mathcal{M}_{3,3}(11,23)=zw^8x^{14}y^3$\\
$\mathcal{M}_{3,3}(11,24)=z^4w^7x^{14}y^3+z^3w^6x^{12}y^3$\\
$\mathcal{M}_{3,3}(11,25)=z^2w^7x^{10}y^3+z^2w^6x^{10}y^3+z^3w^8x^{12}y^3$\\
$\mathcal{M}_{3,3}(11,26)=z^2w^7x^{12}y^3+z^2w^8x^{12}y^3$\\
$\mathcal{M}_{3,3}(11,27)=zw^8x^{10}y^3+zw^7x^{10}y^3+zw^9x^{10}y^3$\\
$\mathcal{M}_{3,3}(11,28)=0$\\
$\mathcal{M}_{3,3}(11,29)=z^2w^4x^{10}y^3+5z^3w^5x^{10}y^3+z^3w^6x^{12}y^3+zw^5x^{10}y^3+2z^2w^4x^8y^3+3z^3w^6x^{10}y^3+2z^2w^7x^{10}y^3+z^4w^4x^{10}y^3+z^2w^5x^8y^3$\\
$\mathcal{M}_{3,3}(11,30)=0$\\
$\mathcal{M}_{3,3}(11,31)=z^2w^5x^{10}y^3+z^2w^4x^{10}y^3+z^3w^5x^{10}y^3+z^3w^6x^{12}y^3+zw^5x^{10}y^3+2z^2w^7x^{10}y^3+z^4w^4x^{10}y^3+z^2w^6x^{10}y^3$\\
$\mathcal{M}_{3,3}(11,32)=z^4w^6x^{14}y^3+z^5w^5x^{12}y^3+z^5w^6x^{14}y^3+z^4w^4x^{10}y^3$\\
$\mathcal{M}_{3,3}(11,33)=2z^2w^5x^{10}y^3+z^2w^5x^{12}y^3+z^3w^7x^{14}y^3+2z^3w^6x^{12}y^3+z^2w^6x^{10}y^3+z^3w^7x^{12}y^3$\\
$\mathcal{M}_{3,3}(11,34)=z^2w^5x^{12}y^3+z^3w^7x^{14}y^3+z^3w^6x^{12}y^3+z^2w^7x^{12}y^3$\\
$\mathcal{M}_{3,3}(11,35)=z^2w^5x^{10}y^3+2zw^8x^{10}y^3+w^6x^{10}y^3+zw^6x^{10}y^3$\\
$\mathcal{M}_{3,3}(11,36)=0$\\
$\mathcal{M}_{3,3}(11,37)=2z^2w^4x^{10}y^3+2z^3w^6x^{12}y^3+2zw^5x^{10}y^3+3z^2w^7x^{10}y^3+2z^4w^4x^{10}y^3$\\
$\mathcal{M}_{3,3}(12,1)=0$\\
$\mathcal{M}_{3,3}(12,2)=2z^3w^5x^{10}y^3+z^2w^3x^6y^3+z^3w^6x^{12}y^3+2z^3w^6x^{10}y^3+z^2w^7x^{10}y^3+2z^3w^4x^8y^3+z^2w^5x^8y^3$\\
$\mathcal{M}_{3,3}(12,3)=0$\\
$\mathcal{M}_{3,3}(12,4)=3z^3w^5x^{10}y^3+2z^2w^4x^8y^3+2z^3w^6x^{10}y^3+z^2w^7x^{10}y^3+z^2w^6x^{10}y^3+z^2w^5x^8y^3$\\
$\mathcal{M}_{3,3}(12,5)=0$\\
$\mathcal{M}_{3,3}(12,6)=3z^2w^5x^{10}y^3+z^3w^6x^{12}y^3+z^2w^7x^{10}y^3+2z^2w^6x^{10}y^3$\\
$\mathcal{M}_{3,3}(12,7)=0$\\
$\mathcal{M}_{3,3}(12,8)=z^3w^5x^{10}y^3+z^3w^7x^{14}y^3+z^3w^6x^{12}y^3+z^2w^4x^8y^3$\\
$\mathcal{M}_{3,3}(12,9)=z^2w^5x^{10}y^3+2z^3w^6x^{12}y^3+z^2w^7x^{12}y^3$\\
$\mathcal{M}_{3,3}(12,10)=zw^8x^{10}y^3+zw^5x^8y^3+zw^6x^8y^3+2z^2w^7x^{10}y^3+z^2w^6x^{10}y^3$\\
$\mathcal{M}_{3,3}(12,11)=0$\\
$\mathcal{M}_{3,3}(12,12)=z^4w^6x^{14}y^3+z^3w^7x^{14}y^3+z^3w^6x^{12}y^3+z^5w^6x^{14}y^3+z^4w^5x^{12}y^3+z^3w^8x^{12}y^3+z^3w^7x^{12}y^3$\\
$\mathcal{M}_{3,3}(12,13)=2z^3w^5x^{12}y^3+z^2w^4x^{10}y^3+z^3w^7x^{14}y^3+z^2w^6x^{12}y^3+z^3w^6x^{14}y^3$\\
$\mathcal{M}_{3,3}(12,14)=2z^2w^6x^{14}y^3+z^2w^7x^{14}y^3$\\
$\mathcal{M}_{3,3}(12,15)=z^3w^7x^{14}y^3+z^3w^6x^{12}y^3+z^3w^8x^{12}y^3+z^3w^7x^{12}y^3$\\
$\mathcal{M}_{3,3}(12,16)=0$\\
$\mathcal{M}_{3,3}(12,17)=2z^2w^5x^{10}y^3+z^3w^6x^{12}y^3+z^2w^6x^{10}y^3+z^3w^7x^{12}y^3$\\
$\mathcal{M}_{3,3}(12,18)=z^3w^7x^{14}y^3+z^2w^6x^{12}y^3$\\
$\mathcal{M}_{3,3}(12,19)=zw^8x^{10}y^3+2zw^7x^{10}y^3+2zw^6x^{10}y^3$\\
$\mathcal{M}_{3,3}(12,20)=0$\\
$\mathcal{M}_{3,3}(12,21)=z^2w^5x^{10}y^3+z^3w^7x^{14}y^3+z^3w^6x^{12}y^3$\\
$\mathcal{M}_{3,3}(12,22)=z^2w^7x^{14}y^3$\\
$\mathcal{M}_{3,3}(12,23)=zw^8x^{14}y^3$\\
$\mathcal{M}_{3,3}(12,24)=z^4w^7x^{14}y^3+z^3w^6x^{12}y^3$\\
$\mathcal{M}_{3,3}(12,25)=z^2w^7x^{10}y^3+z^2w^6x^{10}y^3+z^3w^8x^{12}y^3$\\
$\mathcal{M}_{3,3}(12,26)=z^2w^7x^{12}y^3+z^2w^8x^{12}y^3$\\
$\mathcal{M}_{3,3}(12,27)=zw^8x^{10}y^3+zw^7x^{10}y^3+zw^9x^{10}y^3$\\
$\mathcal{M}_{3,3}(12,28)=0$\\
$\mathcal{M}_{3,3}(12,29)=z^2w^4x^{10}y^3+5z^3w^5x^{10}y^3+z^3w^6x^{12}y^3+zw^5x^{10}y^3+2z^2w^4x^8y^3+3z^3w^6x^{10}y^3+2z^2w^7x^{10}y^3+z^4w^4x^{10}y^3+z^2w^5x^8y^3$\\
$\mathcal{M}_{3,3}(12,30)=0$\\
$\mathcal{M}_{3,3}(12,31)=z^2w^5x^{10}y^3+z^2w^4x^{10}y^3+z^3w^5x^{10}y^3+z^3w^6x^{12}y^3+zw^5x^{10}y^3+2z^2w^7x^{10}y^3+z^4w^4x^{10}y^3+z^2w^6x^{10}y^3$\\
$\mathcal{M}_{3,3}(12,32)=z^4w^6x^{14}y^3+z^5w^5x^{12}y^3+z^5w^6x^{14}y^3+z^4w^4x^{10}y^3$\\
$\mathcal{M}_{3,3}(12,33)=2z^2w^5x^{10}y^3+z^2w^5x^{12}y^3+z^3w^7x^{14}y^3+2z^3w^6x^{12}y^3+z^2w^6x^{10}y^3+z^3w^7x^{12}y^3$\\
$\mathcal{M}_{3,3}(12,34)=z^2w^5x^{12}y^3+z^3w^7x^{14}y^3+z^3w^6x^{12}y^3+z^2w^7x^{12}y^3$\\
$\mathcal{M}_{3,3}(12,35)=z^2w^5x^{10}y^3+2zw^8x^{10}y^3+w^6x^{10}y^3+zw^6x^{10}y^3$\\
$\mathcal{M}_{3,3}(12,36)=0$\\
$\mathcal{M}_{3,3}(12,37)=2z^2w^4x^{10}y^3+2z^3w^6x^{12}y^3+2zw^5x^{10}y^3+3z^2w^7x^{10}y^3+2z^4w^4x^{10}y^3$\\
$\mathcal{M}_{3,3}(13,1)=0$\\
$\mathcal{M}_{3,3}(13,2)=2z^3w^5x^{10}y^3+z^2w^3x^6y^3+z^3w^6x^{12}y^3+2z^3w^6x^{10}y^3+z^2w^7x^{10}y^3+2z^3w^4x^8y^3+z^2w^5x^8y^3$\\
$\mathcal{M}_{3,3}(13,3)=0$\\
$\mathcal{M}_{3,3}(13,4)=3z^3w^5x^{10}y^3+2z^2w^4x^8y^3+2z^3w^6x^{10}y^3+z^2w^7x^{10}y^3+z^2w^6x^{10}y^3+z^2w^5x^8y^3$\\
$\mathcal{M}_{3,3}(13,5)=0$\\
$\mathcal{M}_{3,3}(13,6)=3z^2w^5x^{10}y^3+z^3w^6x^{12}y^3+z^2w^7x^{10}y^3+2z^2w^6x^{10}y^3$\\
$\mathcal{M}_{3,3}(13,7)=0$\\
$\mathcal{M}_{3,3}(13,8)=z^3w^5x^{10}y^3+z^3w^7x^{14}y^3+z^3w^6x^{12}y^3+z^2w^4x^8y^3$\\
$\mathcal{M}_{3,3}(13,9)=z^2w^5x^{10}y^3+2z^3w^6x^{12}y^3+z^2w^7x^{12}y^3$\\
$\mathcal{M}_{3,3}(13,10)=zw^8x^{10}y^3+zw^5x^8y^3+zw^6x^8y^3+2z^2w^7x^{10}y^3+z^2w^6x^{10}y^3$\\
$\mathcal{M}_{3,3}(13,11)=0$\\
$\mathcal{M}_{3,3}(13,12)=z^4w^6x^{14}y^3+z^3w^7x^{14}y^3+z^3w^6x^{12}y^3+z^5w^6x^{14}y^3+z^4w^5x^{12}y^3+z^3w^8x^{12}y^3+z^3w^7x^{12}y^3$\\
$\mathcal{M}_{3,3}(13,13)=2z^3w^5x^{12}y^3+z^2w^4x^{10}y^3+z^3w^7x^{14}y^3+z^2w^6x^{12}y^3+z^3w^6x^{14}y^3$\\
$\mathcal{M}_{3,3}(13,14)=2z^2w^6x^{14}y^3+z^2w^7x^{14}y^3$\\
$\mathcal{M}_{3,3}(13,15)=z^3w^7x^{14}y^3+z^3w^6x^{12}y^3+z^3w^8x^{12}y^3+z^3w^7x^{12}y^3$\\
$\mathcal{M}_{3,3}(13,16)=0$\\
$\mathcal{M}_{3,3}(13,17)=2z^2w^5x^{10}y^3+z^3w^6x^{12}y^3+z^2w^6x^{10}y^3+z^3w^7x^{12}y^3$\\
$\mathcal{M}_{3,3}(13,18)=z^3w^7x^{14}y^3+z^2w^6x^{12}y^3$\\
$\mathcal{M}_{3,3}(13,19)=zw^8x^{10}y^3+2zw^7x^{10}y^3+2zw^6x^{10}y^3$\\
$\mathcal{M}_{3,3}(13,20)=0$\\
$\mathcal{M}_{3,3}(13,21)=z^2w^5x^{10}y^3+z^3w^7x^{14}y^3+z^3w^6x^{12}y^3$\\
$\mathcal{M}_{3,3}(13,22)=z^2w^7x^{14}y^3$\\
$\mathcal{M}_{3,3}(13,23)=zw^8x^{14}y^3$\\
$\mathcal{M}_{3,3}(13,24)=z^4w^7x^{14}y^3+z^3w^6x^{12}y^3$\\
$\mathcal{M}_{3,3}(13,25)=z^2w^7x^{10}y^3+z^2w^6x^{10}y^3+z^3w^8x^{12}y^3$\\
$\mathcal{M}_{3,3}(13,26)=z^2w^7x^{12}y^3+z^2w^8x^{12}y^3$\\
$\mathcal{M}_{3,3}(13,27)=zw^8x^{10}y^3+zw^7x^{10}y^3+zw^9x^{10}y^3$\\
$\mathcal{M}_{3,3}(13,28)=0$\\
$\mathcal{M}_{3,3}(13,29)=z^2w^4x^{10}y^3+5z^3w^5x^{10}y^3+z^3w^6x^{12}y^3+zw^5x^{10}y^3+2z^2w^4x^8y^3+3z^3w^6x^{10}y^3+2z^2w^7x^{10}y^3+z^4w^4x^{10}y^3+z^2w^5x^8y^3$\\
$\mathcal{M}_{3,3}(13,30)=0$\\
$\mathcal{M}_{3,3}(13,31)=z^2w^5x^{10}y^3+z^2w^4x^{10}y^3+z^3w^5x^{10}y^3+z^3w^6x^{12}y^3+zw^5x^{10}y^3+2z^2w^7x^{10}y^3+z^4w^4x^{10}y^3+z^2w^6x^{10}y^3$\\
$\mathcal{M}_{3,3}(13,32)=z^4w^6x^{14}y^3+z^5w^5x^{12}y^3+z^5w^6x^{14}y^3+z^4w^4x^{10}y^3$\\
$\mathcal{M}_{3,3}(13,33)=2z^2w^5x^{10}y^3+z^2w^5x^{12}y^3+z^3w^7x^{14}y^3+2z^3w^6x^{12}y^3+z^2w^6x^{10}y^3+z^3w^7x^{12}y^3$\\
$\mathcal{M}_{3,3}(13,34)=z^2w^5x^{12}y^3+z^3w^7x^{14}y^3+z^3w^6x^{12}y^3+z^2w^7x^{12}y^3$\\
$\mathcal{M}_{3,3}(13,35)=z^2w^5x^{10}y^3+2zw^8x^{10}y^3+w^6x^{10}y^3+zw^6x^{10}y^3$\\
$\mathcal{M}_{3,3}(13,36)=0$\\
$\mathcal{M}_{3,3}(13,37)=2z^2w^4x^{10}y^3+2z^3w^6x^{12}y^3+2zw^5x^{10}y^3+3z^2w^7x^{10}y^3+2z^4w^4x^{10}y^3$\\
$\mathcal{M}_{3,3}(14,1)=0$\\
$\mathcal{M}_{3,3}(14,2)=2z^3w^5x^{10}y^3+z^2w^3x^6y^3+z^3w^6x^{12}y^3+2z^3w^6x^{10}y^3+z^2w^7x^{10}y^3+2z^3w^4x^8y^3+z^2w^5x^8y^3$\\
$\mathcal{M}_{3,3}(14,3)=0$\\
$\mathcal{M}_{3,3}(14,4)=3z^3w^5x^{10}y^3+2z^2w^4x^8y^3+2z^3w^6x^{10}y^3+z^2w^7x^{10}y^3+z^2w^6x^{10}y^3+z^2w^5x^8y^3$\\
$\mathcal{M}_{3,3}(14,5)=0$\\
$\mathcal{M}_{3,3}(14,6)=3z^2w^5x^{10}y^3+z^3w^6x^{12}y^3+z^2w^7x^{10}y^3+2z^2w^6x^{10}y^3$\\
$\mathcal{M}_{3,3}(14,7)=0$\\
$\mathcal{M}_{3,3}(14,8)=z^3w^5x^{10}y^3+z^3w^7x^{14}y^3+z^3w^6x^{12}y^3+z^2w^4x^8y^3$\\
$\mathcal{M}_{3,3}(14,9)=z^2w^5x^{10}y^3+2z^3w^6x^{12}y^3+z^2w^7x^{12}y^3$\\
$\mathcal{M}_{3,3}(14,10)=zw^8x^{10}y^3+zw^5x^8y^3+zw^6x^8y^3+2z^2w^7x^{10}y^3+z^2w^6x^{10}y^3$\\
$\mathcal{M}_{3,3}(14,11)=0$\\
$\mathcal{M}_{3,3}(14,12)=z^4w^6x^{14}y^3+z^3w^7x^{14}y^3+z^3w^6x^{12}y^3+z^5w^6x^{14}y^3+z^4w^5x^{12}y^3+z^3w^8x^{12}y^3+z^3w^7x^{12}y^3$\\
$\mathcal{M}_{3,3}(14,13)=2z^3w^5x^{12}y^3+z^2w^4x^{10}y^3+z^3w^7x^{14}y^3+z^2w^6x^{12}y^3+z^3w^6x^{14}y^3$\\
$\mathcal{M}_{3,3}(14,14)=2z^2w^6x^{14}y^3+z^2w^7x^{14}y^3$\\
$\mathcal{M}_{3,3}(14,15)=z^3w^7x^{14}y^3+z^3w^6x^{12}y^3+z^3w^8x^{12}y^3+z^3w^7x^{12}y^3$\\
$\mathcal{M}_{3,3}(14,16)=0$\\
$\mathcal{M}_{3,3}(14,17)=2z^2w^5x^{10}y^3+z^3w^6x^{12}y^3+z^2w^6x^{10}y^3+z^3w^7x^{12}y^3$\\
$\mathcal{M}_{3,3}(14,18)=z^3w^7x^{14}y^3+z^2w^6x^{12}y^3$\\
$\mathcal{M}_{3,3}(14,19)=zw^8x^{10}y^3+2zw^7x^{10}y^3+2zw^6x^{10}y^3$\\
$\mathcal{M}_{3,3}(14,20)=0$\\
$\mathcal{M}_{3,3}(14,21)=z^2w^5x^{10}y^3+z^3w^7x^{14}y^3+z^3w^6x^{12}y^3$\\
$\mathcal{M}_{3,3}(14,22)=z^2w^7x^{14}y^3$\\
$\mathcal{M}_{3,3}(14,23)=zw^8x^{14}y^3$\\
$\mathcal{M}_{3,3}(14,24)=z^4w^7x^{14}y^3+z^3w^6x^{12}y^3$\\
$\mathcal{M}_{3,3}(14,25)=z^2w^7x^{10}y^3+z^2w^6x^{10}y^3+z^3w^8x^{12}y^3$\\
$\mathcal{M}_{3,3}(14,26)=z^2w^7x^{12}y^3+z^2w^8x^{12}y^3$\\
$\mathcal{M}_{3,3}(14,27)=zw^8x^{10}y^3+zw^7x^{10}y^3+zw^9x^{10}y^3$\\
$\mathcal{M}_{3,3}(14,28)=0$\\
$\mathcal{M}_{3,3}(14,29)=z^2w^4x^{10}y^3+5z^3w^5x^{10}y^3+z^3w^6x^{12}y^3+zw^5x^{10}y^3+2z^2w^4x^8y^3+3z^3w^6x^{10}y^3+2z^2w^7x^{10}y^3+z^4w^4x^{10}y^3+z^2w^5x^8y^3$\\
$\mathcal{M}_{3,3}(14,30)=0$\\
$\mathcal{M}_{3,3}(14,31)=z^2w^5x^{10}y^3+z^2w^4x^{10}y^3+z^3w^5x^{10}y^3+z^3w^6x^{12}y^3+zw^5x^{10}y^3+2z^2w^7x^{10}y^3+z^4w^4x^{10}y^3+z^2w^6x^{10}y^3$\\
$\mathcal{M}_{3,3}(14,32)=z^4w^6x^{14}y^3+z^5w^5x^{12}y^3+z^5w^6x^{14}y^3+z^4w^4x^{10}y^3$\\
$\mathcal{M}_{3,3}(14,33)=2z^2w^5x^{10}y^3+z^2w^5x^{12}y^3+z^3w^7x^{14}y^3+2z^3w^6x^{12}y^3+z^2w^6x^{10}y^3+z^3w^7x^{12}y^3$\\
$\mathcal{M}_{3,3}(14,34)=z^2w^5x^{12}y^3+z^3w^7x^{14}y^3+z^3w^6x^{12}y^3+z^2w^7x^{12}y^3$\\
$\mathcal{M}_{3,3}(14,35)=z^2w^5x^{10}y^3+2zw^8x^{10}y^3+w^6x^{10}y^3+zw^6x^{10}y^3$\\
$\mathcal{M}_{3,3}(14,36)=0$\\
$\mathcal{M}_{3,3}(14,37)=2z^2w^4x^{10}y^3+2z^3w^6x^{12}y^3+2zw^5x^{10}y^3+3z^2w^7x^{10}y^3+2z^4w^4x^{10}y^3$\\
$\mathcal{M}_{3,3}(15,1)=0$\\
$\mathcal{M}_{3,3}(15,2)=2z^3w^5x^{10}y^3+z^2w^3x^6y^3+z^3w^6x^{12}y^3+2z^3w^6x^{10}y^3+z^2w^7x^{10}y^3+2z^3w^4x^8y^3+z^2w^5x^8y^3$\\
$\mathcal{M}_{3,3}(15,3)=0$\\
$\mathcal{M}_{3,3}(15,4)=3z^3w^5x^{10}y^3+2z^2w^4x^8y^3+2z^3w^6x^{10}y^3+z^2w^7x^{10}y^3+z^2w^6x^{10}y^3+z^2w^5x^8y^3$\\
$\mathcal{M}_{3,3}(15,5)=0$\\
$\mathcal{M}_{3,3}(15,6)=3z^2w^5x^{10}y^3+z^3w^6x^{12}y^3+z^2w^7x^{10}y^3+2z^2w^6x^{10}y^3$\\
$\mathcal{M}_{3,3}(15,7)=0$\\
$\mathcal{M}_{3,3}(15,8)=z^3w^5x^{10}y^3+z^3w^7x^{14}y^3+z^3w^6x^{12}y^3+z^2w^4x^8y^3$\\
$\mathcal{M}_{3,3}(15,9)=z^2w^5x^{10}y^3+2z^3w^6x^{12}y^3+z^2w^7x^{12}y^3$\\
$\mathcal{M}_{3,3}(15,10)=zw^8x^{10}y^3+zw^5x^8y^3+zw^6x^8y^3+2z^2w^7x^{10}y^3+z^2w^6x^{10}y^3$\\
$\mathcal{M}_{3,3}(15,11)=0$\\
$\mathcal{M}_{3,3}(15,12)=z^4w^6x^{14}y^3+z^3w^7x^{14}y^3+z^3w^6x^{12}y^3+z^5w^6x^{14}y^3+z^4w^5x^{12}y^3+z^3w^8x^{12}y^3+z^3w^7x^{12}y^3$\\
$\mathcal{M}_{3,3}(15,13)=2z^3w^5x^{12}y^3+z^2w^4x^{10}y^3+z^3w^7x^{14}y^3+z^2w^6x^{12}y^3+z^3w^6x^{14}y^3$\\
$\mathcal{M}_{3,3}(15,14)=2z^2w^6x^{14}y^3+z^2w^7x^{14}y^3$\\
$\mathcal{M}_{3,3}(15,15)=z^3w^7x^{14}y^3+z^3w^6x^{12}y^3+z^3w^8x^{12}y^3+z^3w^7x^{12}y^3$\\
$\mathcal{M}_{3,3}(15,16)=0$\\
$\mathcal{M}_{3,3}(15,17)=2z^2w^5x^{10}y^3+z^3w^6x^{12}y^3+z^2w^6x^{10}y^3+z^3w^7x^{12}y^3$\\
$\mathcal{M}_{3,3}(15,18)=z^3w^7x^{14}y^3+z^2w^6x^{12}y^3$\\
$\mathcal{M}_{3,3}(15,19)=zw^8x^{10}y^3+2zw^7x^{10}y^3+2zw^6x^{10}y^3$\\
$\mathcal{M}_{3,3}(15,20)=0$\\
$\mathcal{M}_{3,3}(15,21)=z^2w^5x^{10}y^3+z^3w^7x^{14}y^3+z^3w^6x^{12}y^3$\\
$\mathcal{M}_{3,3}(15,22)=z^2w^7x^{14}y^3$\\
$\mathcal{M}_{3,3}(15,23)=zw^8x^{14}y^3$\\
$\mathcal{M}_{3,3}(15,24)=z^4w^7x^{14}y^3+z^3w^6x^{12}y^3$\\
$\mathcal{M}_{3,3}(15,25)=z^2w^7x^{10}y^3+z^2w^6x^{10}y^3+z^3w^8x^{12}y^3$\\
$\mathcal{M}_{3,3}(15,26)=z^2w^7x^{12}y^3+z^2w^8x^{12}y^3$\\
$\mathcal{M}_{3,3}(15,27)=zw^8x^{10}y^3+zw^7x^{10}y^3+zw^9x^{10}y^3$\\
$\mathcal{M}_{3,3}(15,28)=0$\\
$\mathcal{M}_{3,3}(15,29)=z^2w^4x^{10}y^3+5z^3w^5x^{10}y^3+z^3w^6x^{12}y^3+zw^5x^{10}y^3+2z^2w^4x^8y^3+3z^3w^6x^{10}y^3+2z^2w^7x^{10}y^3+z^4w^4x^{10}y^3+z^2w^5x^8y^3$\\
$\mathcal{M}_{3,3}(15,30)=0$\\
$\mathcal{M}_{3,3}(15,31)=z^2w^5x^{10}y^3+z^2w^4x^{10}y^3+z^3w^5x^{10}y^3+z^3w^6x^{12}y^3+zw^5x^{10}y^3+2z^2w^7x^{10}y^3+z^4w^4x^{10}y^3+z^2w^6x^{10}y^3$\\
$\mathcal{M}_{3,3}(15,32)=z^4w^6x^{14}y^3+z^5w^5x^{12}y^3+z^5w^6x^{14}y^3+z^4w^4x^{10}y^3$\\
$\mathcal{M}_{3,3}(15,33)=2z^2w^5x^{10}y^3+z^2w^5x^{12}y^3+z^3w^7x^{14}y^3+2z^3w^6x^{12}y^3+z^2w^6x^{10}y^3+z^3w^7x^{12}y^3$\\
$\mathcal{M}_{3,3}(15,34)=z^2w^5x^{12}y^3+z^3w^7x^{14}y^3+z^3w^6x^{12}y^3+z^2w^7x^{12}y^3$\\
$\mathcal{M}_{3,3}(15,35)=z^2w^5x^{10}y^3+2zw^8x^{10}y^3+w^6x^{10}y^3+zw^6x^{10}y^3$\\
$\mathcal{M}_{3,3}(15,36)=0$\\
$\mathcal{M}_{3,3}(15,37)=2z^2w^4x^{10}y^3+2z^3w^6x^{12}y^3+2zw^5x^{10}y^3+3z^2w^7x^{10}y^3+2z^4w^4x^{10}y^3$\\
$\mathcal{M}_{3,3}(16,1)=0$\\
$\mathcal{M}_{3,3}(16,2)=2z^3w^5x^{10}y^3+z^2w^3x^6y^3+z^3w^6x^{12}y^3+2z^3w^6x^{10}y^3+z^2w^7x^{10}y^3+2z^3w^4x^8y^3+z^2w^5x^8y^3$\\
$\mathcal{M}_{3,3}(16,3)=0$\\
$\mathcal{M}_{3,3}(16,4)=3z^3w^5x^{10}y^3+2z^2w^4x^8y^3+2z^3w^6x^{10}y^3+z^2w^7x^{10}y^3+z^2w^6x^{10}y^3+z^2w^5x^8y^3$\\
$\mathcal{M}_{3,3}(16,5)=0$\\
$\mathcal{M}_{3,3}(16,6)=3z^2w^5x^{10}y^3+z^3w^6x^{12}y^3+z^2w^7x^{10}y^3+2z^2w^6x^{10}y^3$\\
$\mathcal{M}_{3,3}(16,7)=0$\\
$\mathcal{M}_{3,3}(16,8)=z^3w^5x^{10}y^3+z^3w^7x^{14}y^3+z^3w^6x^{12}y^3+z^2w^4x^8y^3$\\
$\mathcal{M}_{3,3}(16,9)=z^2w^5x^{10}y^3+2z^3w^6x^{12}y^3+z^2w^7x^{12}y^3$\\
$\mathcal{M}_{3,3}(16,10)=zw^8x^{10}y^3+zw^5x^8y^3+zw^6x^8y^3+2z^2w^7x^{10}y^3+z^2w^6x^{10}y^3$\\
$\mathcal{M}_{3,3}(16,11)=0$\\
$\mathcal{M}_{3,3}(16,12)=z^4w^6x^{14}y^3+z^3w^7x^{14}y^3+z^3w^6x^{12}y^3+z^5w^6x^{14}y^3+z^4w^5x^{12}y^3+z^3w^8x^{12}y^3+z^3w^7x^{12}y^3$\\
$\mathcal{M}_{3,3}(16,13)=2z^3w^5x^{12}y^3+z^2w^4x^{10}y^3+z^3w^7x^{14}y^3+z^2w^6x^{12}y^3+z^3w^6x^{14}y^3$\\
$\mathcal{M}_{3,3}(16,14)=2z^2w^6x^{14}y^3+z^2w^7x^{14}y^3$\\
$\mathcal{M}_{3,3}(16,15)=z^3w^7x^{14}y^3+z^3w^6x^{12}y^3+z^3w^8x^{12}y^3+z^3w^7x^{12}y^3$\\
$\mathcal{M}_{3,3}(16,16)=0$\\
$\mathcal{M}_{3,3}(16,17)=2z^2w^5x^{10}y^3+z^3w^6x^{12}y^3+z^2w^6x^{10}y^3+z^3w^7x^{12}y^3$\\
$\mathcal{M}_{3,3}(16,18)=z^3w^7x^{14}y^3+z^2w^6x^{12}y^3$\\
$\mathcal{M}_{3,3}(16,19)=zw^8x^{10}y^3+2zw^7x^{10}y^3+2zw^6x^{10}y^3$\\
$\mathcal{M}_{3,3}(16,20)=0$\\
$\mathcal{M}_{3,3}(16,21)=z^2w^5x^{10}y^3+z^3w^7x^{14}y^3+z^3w^6x^{12}y^3$\\
$\mathcal{M}_{3,3}(16,22)=z^2w^7x^{14}y^3$\\
$\mathcal{M}_{3,3}(16,23)=zw^8x^{14}y^3$\\
$\mathcal{M}_{3,3}(16,24)=z^4w^7x^{14}y^3+z^3w^6x^{12}y^3$\\
$\mathcal{M}_{3,3}(16,25)=z^2w^7x^{10}y^3+z^2w^6x^{10}y^3+z^3w^8x^{12}y^3$\\
$\mathcal{M}_{3,3}(16,26)=z^2w^7x^{12}y^3+z^2w^8x^{12}y^3$\\
$\mathcal{M}_{3,3}(16,27)=zw^8x^{10}y^3+zw^7x^{10}y^3+zw^9x^{10}y^3$\\
$\mathcal{M}_{3,3}(16,28)=0$\\
$\mathcal{M}_{3,3}(16,29)=z^2w^4x^{10}y^3+5z^3w^5x^{10}y^3+z^3w^6x^{12}y^3+zw^5x^{10}y^3+2z^2w^4x^8y^3+3z^3w^6x^{10}y^3+2z^2w^7x^{10}y^3+z^4w^4x^{10}y^3+z^2w^5x^8y^3$\\
$\mathcal{M}_{3,3}(16,30)=0$\\
$\mathcal{M}_{3,3}(16,31)=z^2w^5x^{10}y^3+z^2w^4x^{10}y^3+z^3w^5x^{10}y^3+z^3w^6x^{12}y^3+zw^5x^{10}y^3+2z^2w^7x^{10}y^3+z^4w^4x^{10}y^3+z^2w^6x^{10}y^3$\\
$\mathcal{M}_{3,3}(16,32)=z^4w^6x^{14}y^3+z^5w^5x^{12}y^3+z^5w^6x^{14}y^3+z^4w^4x^{10}y^3$\\
$\mathcal{M}_{3,3}(16,33)=2z^2w^5x^{10}y^3+z^2w^5x^{12}y^3+z^3w^7x^{14}y^3+2z^3w^6x^{12}y^3+z^2w^6x^{10}y^3+z^3w^7x^{12}y^3$\\
$\mathcal{M}_{3,3}(16,34)=z^2w^5x^{12}y^3+z^3w^7x^{14}y^3+z^3w^6x^{12}y^3+z^2w^7x^{12}y^3$\\
$\mathcal{M}_{3,3}(16,35)=z^2w^5x^{10}y^3+2zw^8x^{10}y^3+w^6x^{10}y^3+zw^6x^{10}y^3$\\
$\mathcal{M}_{3,3}(16,36)=0$\\
$\mathcal{M}_{3,3}(16,37)=2z^2w^4x^{10}y^3+2z^3w^6x^{12}y^3+2zw^5x^{10}y^3+3z^2w^7x^{10}y^3+2z^4w^4x^{10}y^3$\\
$\mathcal{M}_{3,3}(17,1)=0$\\
$\mathcal{M}_{3,3}(17,2)=2z^3w^5x^{10}y^3+z^2w^3x^6y^3+z^3w^6x^{12}y^3+2z^3w^6x^{10}y^3+z^2w^7x^{10}y^3+2z^3w^4x^8y^3+z^2w^5x^8y^3$\\
$\mathcal{M}_{3,3}(17,3)=0$\\
$\mathcal{M}_{3,3}(17,4)=3z^3w^5x^{10}y^3+2z^2w^4x^8y^3+2z^3w^6x^{10}y^3+z^2w^7x^{10}y^3+z^2w^6x^{10}y^3+z^2w^5x^8y^3$\\
$\mathcal{M}_{3,3}(17,5)=0$\\
$\mathcal{M}_{3,3}(17,6)=3z^2w^5x^{10}y^3+z^3w^6x^{12}y^3+z^2w^7x^{10}y^3+2z^2w^6x^{10}y^3$\\
$\mathcal{M}_{3,3}(17,7)=0$\\
$\mathcal{M}_{3,3}(17,8)=z^3w^5x^{10}y^3+z^3w^7x^{14}y^3+z^3w^6x^{12}y^3+z^2w^4x^8y^3$\\
$\mathcal{M}_{3,3}(17,9)=z^2w^5x^{10}y^3+2z^3w^6x^{12}y^3+z^2w^7x^{12}y^3$\\
$\mathcal{M}_{3,3}(17,10)=zw^8x^{10}y^3+zw^5x^8y^3+zw^6x^8y^3+2z^2w^7x^{10}y^3+z^2w^6x^{10}y^3$\\
$\mathcal{M}_{3,3}(17,11)=0$\\
$\mathcal{M}_{3,3}(17,12)=z^4w^6x^{14}y^3+z^3w^7x^{14}y^3+z^3w^6x^{12}y^3+z^5w^6x^{14}y^3+z^4w^5x^{12}y^3+z^3w^8x^{12}y^3+z^3w^7x^{12}y^3$\\
$\mathcal{M}_{3,3}(17,13)=2z^3w^5x^{12}y^3+z^2w^4x^{10}y^3+z^3w^7x^{14}y^3+z^2w^6x^{12}y^3+z^3w^6x^{14}y^3$\\
$\mathcal{M}_{3,3}(17,14)=2z^2w^6x^{14}y^3+z^2w^7x^{14}y^3$\\
$\mathcal{M}_{3,3}(17,15)=z^3w^7x^{14}y^3+z^3w^6x^{12}y^3+z^3w^8x^{12}y^3+z^3w^7x^{12}y^3$\\
$\mathcal{M}_{3,3}(17,16)=0$\\
$\mathcal{M}_{3,3}(17,17)=2z^2w^5x^{10}y^3+z^3w^6x^{12}y^3+z^2w^6x^{10}y^3+z^3w^7x^{12}y^3$\\
$\mathcal{M}_{3,3}(17,18)=z^3w^7x^{14}y^3+z^2w^6x^{12}y^3$\\
$\mathcal{M}_{3,3}(17,19)=zw^8x^{10}y^3+2zw^7x^{10}y^3+2zw^6x^{10}y^3$\\
$\mathcal{M}_{3,3}(17,20)=0$\\
$\mathcal{M}_{3,3}(17,21)=z^2w^5x^{10}y^3+z^3w^7x^{14}y^3+z^3w^6x^{12}y^3$\\
$\mathcal{M}_{3,3}(17,22)=z^2w^7x^{14}y^3$\\
$\mathcal{M}_{3,3}(17,23)=zw^8x^{14}y^3$\\
$\mathcal{M}_{3,3}(17,24)=z^4w^7x^{14}y^3+z^3w^6x^{12}y^3$\\
$\mathcal{M}_{3,3}(17,25)=z^2w^7x^{10}y^3+z^2w^6x^{10}y^3+z^3w^8x^{12}y^3$\\
$\mathcal{M}_{3,3}(17,26)=z^2w^7x^{12}y^3+z^2w^8x^{12}y^3$\\
$\mathcal{M}_{3,3}(17,27)=zw^8x^{10}y^3+zw^7x^{10}y^3+zw^9x^{10}y^3$\\
$\mathcal{M}_{3,3}(17,28)=0$\\
$\mathcal{M}_{3,3}(17,29)=z^2w^4x^{10}y^3+5z^3w^5x^{10}y^3+z^3w^6x^{12}y^3+zw^5x^{10}y^3+2z^2w^4x^8y^3+3z^3w^6x^{10}y^3+2z^2w^7x^{10}y^3+z^4w^4x^{10}y^3+z^2w^5x^8y^3$\\
$\mathcal{M}_{3,3}(17,30)=0$\\
$\mathcal{M}_{3,3}(17,31)=z^2w^5x^{10}y^3+z^2w^4x^{10}y^3+z^3w^5x^{10}y^3+z^3w^6x^{12}y^3+zw^5x^{10}y^3+2z^2w^7x^{10}y^3+z^4w^4x^{10}y^3+z^2w^6x^{10}y^3$\\
$\mathcal{M}_{3,3}(17,32)=z^4w^6x^{14}y^3+z^5w^5x^{12}y^3+z^5w^6x^{14}y^3+z^4w^4x^{10}y^3$\\
$\mathcal{M}_{3,3}(17,33)=2z^2w^5x^{10}y^3+z^2w^5x^{12}y^3+z^3w^7x^{14}y^3+2z^3w^6x^{12}y^3+z^2w^6x^{10}y^3+z^3w^7x^{12}y^3$\\
$\mathcal{M}_{3,3}(17,34)=z^2w^5x^{12}y^3+z^3w^7x^{14}y^3+z^3w^6x^{12}y^3+z^2w^7x^{12}y^3$\\
$\mathcal{M}_{3,3}(17,35)=z^2w^5x^{10}y^3+2zw^8x^{10}y^3+w^6x^{10}y^3+zw^6x^{10}y^3$\\
$\mathcal{M}_{3,3}(17,36)=0$\\
$\mathcal{M}_{3,3}(17,37)=2z^2w^4x^{10}y^3+2z^3w^6x^{12}y^3+2zw^5x^{10}y^3+3z^2w^7x^{10}y^3+2z^4w^4x^{10}y^3$\\
$\mathcal{M}_{3,3}(18,1)=0$\\
$\mathcal{M}_{3,3}(18,2)=2z^3w^5x^{10}y^3+z^2w^3x^6y^3+z^3w^6x^{12}y^3+2z^3w^6x^{10}y^3+z^2w^7x^{10}y^3+2z^3w^4x^8y^3+z^2w^5x^8y^3$\\
$\mathcal{M}_{3,3}(18,3)=0$\\
$\mathcal{M}_{3,3}(18,4)=3z^3w^5x^{10}y^3+2z^2w^4x^8y^3+2z^3w^6x^{10}y^3+z^2w^7x^{10}y^3+z^2w^6x^{10}y^3+z^2w^5x^8y^3$\\
$\mathcal{M}_{3,3}(18,5)=0$\\
$\mathcal{M}_{3,3}(18,6)=3z^2w^5x^{10}y^3+z^3w^6x^{12}y^3+z^2w^7x^{10}y^3+2z^2w^6x^{10}y^3$\\
$\mathcal{M}_{3,3}(18,7)=0$\\
$\mathcal{M}_{3,3}(18,8)=z^3w^5x^{10}y^3+z^3w^7x^{14}y^3+z^3w^6x^{12}y^3+z^2w^4x^8y^3$\\
$\mathcal{M}_{3,3}(18,9)=z^2w^5x^{10}y^3+2z^3w^6x^{12}y^3+z^2w^7x^{12}y^3$\\
$\mathcal{M}_{3,3}(18,10)=zw^8x^{10}y^3+zw^5x^8y^3+zw^6x^8y^3+2z^2w^7x^{10}y^3+z^2w^6x^{10}y^3$\\
$\mathcal{M}_{3,3}(18,11)=0$\\
$\mathcal{M}_{3,3}(18,12)=z^4w^6x^{14}y^3+z^3w^7x^{14}y^3+z^3w^6x^{12}y^3+z^5w^6x^{14}y^3+z^4w^5x^{12}y^3+z^3w^8x^{12}y^3+z^3w^7x^{12}y^3$\\
$\mathcal{M}_{3,3}(18,13)=2z^3w^5x^{12}y^3+z^2w^4x^{10}y^3+z^3w^7x^{14}y^3+z^2w^6x^{12}y^3+z^3w^6x^{14}y^3$\\
$\mathcal{M}_{3,3}(18,14)=2z^2w^6x^{14}y^3+z^2w^7x^{14}y^3$\\
$\mathcal{M}_{3,3}(18,15)=z^3w^7x^{14}y^3+z^3w^6x^{12}y^3+z^3w^8x^{12}y^3+z^3w^7x^{12}y^3$\\
$\mathcal{M}_{3,3}(18,16)=0$\\
$\mathcal{M}_{3,3}(18,17)=2z^2w^5x^{10}y^3+z^3w^6x^{12}y^3+z^2w^6x^{10}y^3+z^3w^7x^{12}y^3$\\
$\mathcal{M}_{3,3}(18,18)=z^3w^7x^{14}y^3+z^2w^6x^{12}y^3$\\
$\mathcal{M}_{3,3}(18,19)=zw^8x^{10}y^3+2zw^7x^{10}y^3+2zw^6x^{10}y^3$\\
$\mathcal{M}_{3,3}(18,20)=0$\\
$\mathcal{M}_{3,3}(18,21)=z^2w^5x^{10}y^3+z^3w^7x^{14}y^3+z^3w^6x^{12}y^3$\\
$\mathcal{M}_{3,3}(18,22)=z^2w^7x^{14}y^3$\\
$\mathcal{M}_{3,3}(18,23)=zw^8x^{14}y^3$\\
$\mathcal{M}_{3,3}(18,24)=z^4w^7x^{14}y^3+z^3w^6x^{12}y^3$\\
$\mathcal{M}_{3,3}(18,25)=z^2w^7x^{10}y^3+z^2w^6x^{10}y^3+z^3w^8x^{12}y^3$\\
$\mathcal{M}_{3,3}(18,26)=z^2w^7x^{12}y^3+z^2w^8x^{12}y^3$\\
$\mathcal{M}_{3,3}(18,27)=zw^8x^{10}y^3+zw^7x^{10}y^3+zw^9x^{10}y^3$\\
$\mathcal{M}_{3,3}(18,28)=0$\\
$\mathcal{M}_{3,3}(18,29)=z^2w^4x^{10}y^3+5z^3w^5x^{10}y^3+z^3w^6x^{12}y^3+zw^5x^{10}y^3+2z^2w^4x^8y^3+3z^3w^6x^{10}y^3+2z^2w^7x^{10}y^3+z^4w^4x^{10}y^3+z^2w^5x^8y^3$\\
$\mathcal{M}_{3,3}(18,30)=0$\\
$\mathcal{M}_{3,3}(18,31)=z^2w^5x^{10}y^3+z^2w^4x^{10}y^3+z^3w^5x^{10}y^3+z^3w^6x^{12}y^3+zw^5x^{10}y^3+2z^2w^7x^{10}y^3+z^4w^4x^{10}y^3+z^2w^6x^{10}y^3$\\
$\mathcal{M}_{3,3}(18,32)=z^4w^6x^{14}y^3+z^5w^5x^{12}y^3+z^5w^6x^{14}y^3+z^4w^4x^{10}y^3$\\
$\mathcal{M}_{3,3}(18,33)=2z^2w^5x^{10}y^3+z^2w^5x^{12}y^3+z^3w^7x^{14}y^3+2z^3w^6x^{12}y^3+z^2w^6x^{10}y^3+z^3w^7x^{12}y^3$\\
$\mathcal{M}_{3,3}(18,34)=z^2w^5x^{12}y^3+z^3w^7x^{14}y^3+z^3w^6x^{12}y^3+z^2w^7x^{12}y^3$\\
$\mathcal{M}_{3,3}(18,35)=z^2w^5x^{10}y^3+2zw^8x^{10}y^3+w^6x^{10}y^3+zw^6x^{10}y^3$\\
$\mathcal{M}_{3,3}(18,36)=0$\\
$\mathcal{M}_{3,3}(18,37)=2z^2w^4x^{10}y^3+2z^3w^6x^{12}y^3+2zw^5x^{10}y^3+3z^2w^7x^{10}y^3+2z^4w^4x^{10}y^3$\\
$\mathcal{M}_{3,3}(19,1)=0$\\
$\mathcal{M}_{3,3}(19,2)=2z^3w^5x^{10}y^3+z^2w^3x^6y^3+z^3w^6x^{12}y^3+2z^3w^6x^{10}y^3+z^2w^7x^{10}y^3+2z^3w^4x^8y^3+z^2w^5x^8y^3$\\
$\mathcal{M}_{3,3}(19,3)=0$\\
$\mathcal{M}_{3,3}(19,4)=3z^3w^5x^{10}y^3+2z^2w^4x^8y^3+2z^3w^6x^{10}y^3+z^2w^7x^{10}y^3+z^2w^6x^{10}y^3+z^2w^5x^8y^3$\\
$\mathcal{M}_{3,3}(19,5)=0$\\
$\mathcal{M}_{3,3}(19,6)=3z^2w^5x^{10}y^3+z^3w^6x^{12}y^3+z^2w^7x^{10}y^3+2z^2w^6x^{10}y^3$\\
$\mathcal{M}_{3,3}(19,7)=0$\\
$\mathcal{M}_{3,3}(19,8)=z^3w^5x^{10}y^3+z^3w^7x^{14}y^3+z^3w^6x^{12}y^3+z^2w^4x^8y^3$\\
$\mathcal{M}_{3,3}(19,9)=z^2w^5x^{10}y^3+2z^3w^6x^{12}y^3+z^2w^7x^{12}y^3$\\
$\mathcal{M}_{3,3}(19,10)=zw^8x^{10}y^3+zw^5x^8y^3+zw^6x^8y^3+2z^2w^7x^{10}y^3+z^2w^6x^{10}y^3$\\
$\mathcal{M}_{3,3}(19,11)=0$\\
$\mathcal{M}_{3,3}(19,12)=z^4w^6x^{14}y^3+z^3w^7x^{14}y^3+z^3w^6x^{12}y^3+z^5w^6x^{14}y^3+z^4w^5x^{12}y^3+z^3w^8x^{12}y^3+z^3w^7x^{12}y^3$\\
$\mathcal{M}_{3,3}(19,13)=2z^3w^5x^{12}y^3+z^2w^4x^{10}y^3+z^3w^7x^{14}y^3+z^2w^6x^{12}y^3+z^3w^6x^{14}y^3$\\
$\mathcal{M}_{3,3}(19,14)=2z^2w^6x^{14}y^3+z^2w^7x^{14}y^3$\\
$\mathcal{M}_{3,3}(19,15)=z^3w^7x^{14}y^3+z^3w^6x^{12}y^3+z^3w^8x^{12}y^3+z^3w^7x^{12}y^3$\\
$\mathcal{M}_{3,3}(19,16)=0$\\
$\mathcal{M}_{3,3}(19,17)=2z^2w^5x^{10}y^3+z^3w^6x^{12}y^3+z^2w^6x^{10}y^3+z^3w^7x^{12}y^3$\\
$\mathcal{M}_{3,3}(19,18)=z^3w^7x^{14}y^3+z^2w^6x^{12}y^3$\\
$\mathcal{M}_{3,3}(19,19)=zw^8x^{10}y^3+2zw^7x^{10}y^3+2zw^6x^{10}y^3$\\
$\mathcal{M}_{3,3}(19,20)=0$\\
$\mathcal{M}_{3,3}(19,21)=z^2w^5x^{10}y^3+z^3w^7x^{14}y^3+z^3w^6x^{12}y^3$\\
$\mathcal{M}_{3,3}(19,22)=z^2w^7x^{14}y^3$\\
$\mathcal{M}_{3,3}(19,23)=zw^8x^{14}y^3$\\
$\mathcal{M}_{3,3}(19,24)=z^4w^7x^{14}y^3+z^3w^6x^{12}y^3$\\
$\mathcal{M}_{3,3}(19,25)=z^2w^7x^{10}y^3+z^2w^6x^{10}y^3+z^3w^8x^{12}y^3$\\
$\mathcal{M}_{3,3}(19,26)=z^2w^7x^{12}y^3+z^2w^8x^{12}y^3$\\
$\mathcal{M}_{3,3}(19,27)=zw^8x^{10}y^3+zw^7x^{10}y^3+zw^9x^{10}y^3$\\
$\mathcal{M}_{3,3}(19,28)=0$\\
$\mathcal{M}_{3,3}(19,29)=z^2w^4x^{10}y^3+5z^3w^5x^{10}y^3+z^3w^6x^{12}y^3+zw^5x^{10}y^3+2z^2w^4x^8y^3+3z^3w^6x^{10}y^3+2z^2w^7x^{10}y^3+z^4w^4x^{10}y^3+z^2w^5x^8y^3$\\
$\mathcal{M}_{3,3}(19,30)=0$\\
$\mathcal{M}_{3,3}(19,31)=z^2w^5x^{10}y^3+z^2w^4x^{10}y^3+z^3w^5x^{10}y^3+z^3w^6x^{12}y^3+zw^5x^{10}y^3+2z^2w^7x^{10}y^3+z^4w^4x^{10}y^3+z^2w^6x^{10}y^3$\\
$\mathcal{M}_{3,3}(19,32)=z^4w^6x^{14}y^3+z^5w^5x^{12}y^3+z^5w^6x^{14}y^3+z^4w^4x^{10}y^3$\\
$\mathcal{M}_{3,3}(19,33)=2z^2w^5x^{10}y^3+z^2w^5x^{12}y^3+z^3w^7x^{14}y^3+2z^3w^6x^{12}y^3+z^2w^6x^{10}y^3+z^3w^7x^{12}y^3$\\
$\mathcal{M}_{3,3}(19,34)=z^2w^5x^{12}y^3+z^3w^7x^{14}y^3+z^3w^6x^{12}y^3+z^2w^7x^{12}y^3$\\
$\mathcal{M}_{3,3}(19,35)=z^2w^5x^{10}y^3+2zw^8x^{10}y^3+w^6x^{10}y^3+zw^6x^{10}y^3$\\
$\mathcal{M}_{3,3}(19,36)=0$\\
$\mathcal{M}_{3,3}(19,37)=2z^2w^4x^{10}y^3+2z^3w^6x^{12}y^3+2zw^5x^{10}y^3+3z^2w^7x^{10}y^3+2z^4w^4x^{10}y^3$\\
$\mathcal{M}_{3,3}(20,1)=0$\\
$\mathcal{M}_{3,3}(20,2)=2z^3w^5x^{10}y^3+z^2w^3x^6y^3+z^3w^6x^{12}y^3+2z^3w^6x^{10}y^3+z^2w^7x^{10}y^3+2z^3w^4x^8y^3+z^2w^5x^8y^3$\\
$\mathcal{M}_{3,3}(20,3)=0$\\
$\mathcal{M}_{3,3}(20,4)=3z^3w^5x^{10}y^3+2z^2w^4x^8y^3+2z^3w^6x^{10}y^3+z^2w^7x^{10}y^3+z^2w^6x^{10}y^3+z^2w^5x^8y^3$\\
$\mathcal{M}_{3,3}(20,5)=0$\\
$\mathcal{M}_{3,3}(20,6)=3z^2w^5x^{10}y^3+z^3w^6x^{12}y^3+z^2w^7x^{10}y^3+2z^2w^6x^{10}y^3$\\
$\mathcal{M}_{3,3}(20,7)=0$\\
$\mathcal{M}_{3,3}(20,8)=z^3w^5x^{10}y^3+z^3w^7x^{14}y^3+z^3w^6x^{12}y^3+z^2w^4x^8y^3$\\
$\mathcal{M}_{3,3}(20,9)=z^2w^5x^{10}y^3+2z^3w^6x^{12}y^3+z^2w^7x^{12}y^3$\\
$\mathcal{M}_{3,3}(20,10)=zw^8x^{10}y^3+zw^5x^8y^3+zw^6x^8y^3+2z^2w^7x^{10}y^3+z^2w^6x^{10}y^3$\\
$\mathcal{M}_{3,3}(20,11)=0$\\
$\mathcal{M}_{3,3}(20,12)=z^4w^6x^{14}y^3+z^3w^7x^{14}y^3+z^3w^6x^{12}y^3+z^5w^6x^{14}y^3+z^4w^5x^{12}y^3+z^3w^8x^{12}y^3+z^3w^7x^{12}y^3$\\
$\mathcal{M}_{3,3}(20,13)=2z^3w^5x^{12}y^3+z^2w^4x^{10}y^3+z^3w^7x^{14}y^3+z^2w^6x^{12}y^3+z^3w^6x^{14}y^3$\\
$\mathcal{M}_{3,3}(20,14)=2z^2w^6x^{14}y^3+z^2w^7x^{14}y^3$\\
$\mathcal{M}_{3,3}(20,15)=z^3w^7x^{14}y^3+z^3w^6x^{12}y^3+z^3w^8x^{12}y^3+z^3w^7x^{12}y^3$\\
$\mathcal{M}_{3,3}(20,16)=0$\\
$\mathcal{M}_{3,3}(20,17)=2z^2w^5x^{10}y^3+z^3w^6x^{12}y^3+z^2w^6x^{10}y^3+z^3w^7x^{12}y^3$\\
$\mathcal{M}_{3,3}(20,18)=z^3w^7x^{14}y^3+z^2w^6x^{12}y^3$\\
$\mathcal{M}_{3,3}(20,19)=zw^8x^{10}y^3+2zw^7x^{10}y^3+2zw^6x^{10}y^3$\\
$\mathcal{M}_{3,3}(20,20)=0$\\
$\mathcal{M}_{3,3}(20,21)=z^2w^5x^{10}y^3+z^3w^7x^{14}y^3+z^3w^6x^{12}y^3$\\
$\mathcal{M}_{3,3}(20,22)=z^2w^7x^{14}y^3$\\
$\mathcal{M}_{3,3}(20,23)=zw^8x^{14}y^3$\\
$\mathcal{M}_{3,3}(20,24)=z^4w^7x^{14}y^3+z^3w^6x^{12}y^3$\\
$\mathcal{M}_{3,3}(20,25)=z^2w^7x^{10}y^3+z^2w^6x^{10}y^3+z^3w^8x^{12}y^3$\\
$\mathcal{M}_{3,3}(20,26)=z^2w^7x^{12}y^3+z^2w^8x^{12}y^3$\\
$\mathcal{M}_{3,3}(20,27)=zw^8x^{10}y^3+zw^7x^{10}y^3+zw^9x^{10}y^3$\\
$\mathcal{M}_{3,3}(20,28)=0$\\
$\mathcal{M}_{3,3}(20,29)=z^2w^4x^{10}y^3+5z^3w^5x^{10}y^3+z^3w^6x^{12}y^3+zw^5x^{10}y^3+2z^2w^4x^8y^3+3z^3w^6x^{10}y^3+2z^2w^7x^{10}y^3+z^4w^4x^{10}y^3+z^2w^5x^8y^3$\\
$\mathcal{M}_{3,3}(20,30)=0$\\
$\mathcal{M}_{3,3}(20,31)=z^2w^5x^{10}y^3+z^2w^4x^{10}y^3+z^3w^5x^{10}y^3+z^3w^6x^{12}y^3+zw^5x^{10}y^3+2z^2w^7x^{10}y^3+z^4w^4x^{10}y^3+z^2w^6x^{10}y^3$\\
$\mathcal{M}_{3,3}(20,32)=z^4w^6x^{14}y^3+z^5w^5x^{12}y^3+z^5w^6x^{14}y^3+z^4w^4x^{10}y^3$\\
$\mathcal{M}_{3,3}(20,33)=2z^2w^5x^{10}y^3+z^2w^5x^{12}y^3+z^3w^7x^{14}y^3+2z^3w^6x^{12}y^3+z^2w^6x^{10}y^3+z^3w^7x^{12}y^3$\\
$\mathcal{M}_{3,3}(20,34)=z^2w^5x^{12}y^3+z^3w^7x^{14}y^3+z^3w^6x^{12}y^3+z^2w^7x^{12}y^3$\\
$\mathcal{M}_{3,3}(20,35)=z^2w^5x^{10}y^3+2zw^8x^{10}y^3+w^6x^{10}y^3+zw^6x^{10}y^3$\\
$\mathcal{M}_{3,3}(20,36)=0$\\
$\mathcal{M}_{3,3}(20,37)=2z^2w^4x^{10}y^3+2z^3w^6x^{12}y^3+2zw^5x^{10}y^3+3z^2w^7x^{10}y^3+2z^4w^4x^{10}y^3$\\
$\mathcal{M}_{3,3}(21,1)=0$\\
$\mathcal{M}_{3,3}(21,2)=2z^3w^5x^{10}y^3+z^2w^3x^6y^3+z^3w^6x^{12}y^3+2z^3w^6x^{10}y^3+z^2w^7x^{10}y^3+2z^3w^4x^8y^3+z^2w^5x^8y^3$\\
$\mathcal{M}_{3,3}(21,3)=0$\\
$\mathcal{M}_{3,3}(21,4)=3z^3w^5x^{10}y^3+2z^2w^4x^8y^3+2z^3w^6x^{10}y^3+z^2w^7x^{10}y^3+z^2w^6x^{10}y^3+z^2w^5x^8y^3$\\
$\mathcal{M}_{3,3}(21,5)=0$\\
$\mathcal{M}_{3,3}(21,6)=3z^2w^5x^{10}y^3+z^3w^6x^{12}y^3+z^2w^7x^{10}y^3+2z^2w^6x^{10}y^3$\\
$\mathcal{M}_{3,3}(21,7)=0$\\
$\mathcal{M}_{3,3}(21,8)=z^3w^5x^{10}y^3+z^3w^7x^{14}y^3+z^3w^6x^{12}y^3+z^2w^4x^8y^3$\\
$\mathcal{M}_{3,3}(21,9)=z^2w^5x^{10}y^3+2z^3w^6x^{12}y^3+z^2w^7x^{12}y^3$\\
$\mathcal{M}_{3,3}(21,10)=zw^8x^{10}y^3+zw^5x^8y^3+zw^6x^8y^3+2z^2w^7x^{10}y^3+z^2w^6x^{10}y^3$\\
$\mathcal{M}_{3,3}(21,11)=0$\\
$\mathcal{M}_{3,3}(21,12)=z^4w^6x^{14}y^3+z^3w^7x^{14}y^3+z^3w^6x^{12}y^3+z^5w^6x^{14}y^3+z^4w^5x^{12}y^3+z^3w^8x^{12}y^3+z^3w^7x^{12}y^3$\\
$\mathcal{M}_{3,3}(21,13)=2z^3w^5x^{12}y^3+z^2w^4x^{10}y^3+z^3w^7x^{14}y^3+z^2w^6x^{12}y^3+z^3w^6x^{14}y^3$\\
$\mathcal{M}_{3,3}(21,14)=2z^2w^6x^{14}y^3+z^2w^7x^{14}y^3$\\
$\mathcal{M}_{3,3}(21,15)=z^3w^7x^{14}y^3+z^3w^6x^{12}y^3+z^3w^8x^{12}y^3+z^3w^7x^{12}y^3$\\
$\mathcal{M}_{3,3}(21,16)=0$\\
$\mathcal{M}_{3,3}(21,17)=2z^2w^5x^{10}y^3+z^3w^6x^{12}y^3+z^2w^6x^{10}y^3+z^3w^7x^{12}y^3$\\
$\mathcal{M}_{3,3}(21,18)=z^3w^7x^{14}y^3+z^2w^6x^{12}y^3$\\
$\mathcal{M}_{3,3}(21,19)=zw^8x^{10}y^3+2zw^7x^{10}y^3+2zw^6x^{10}y^3$\\
$\mathcal{M}_{3,3}(21,20)=0$\\
$\mathcal{M}_{3,3}(21,21)=z^2w^5x^{10}y^3+z^3w^7x^{14}y^3+z^3w^6x^{12}y^3$\\
$\mathcal{M}_{3,3}(21,22)=z^2w^7x^{14}y^3$\\
$\mathcal{M}_{3,3}(21,23)=zw^8x^{14}y^3$\\
$\mathcal{M}_{3,3}(21,24)=z^4w^7x^{14}y^3+z^3w^6x^{12}y^3$\\
$\mathcal{M}_{3,3}(21,25)=z^2w^7x^{10}y^3+z^2w^6x^{10}y^3+z^3w^8x^{12}y^3$\\
$\mathcal{M}_{3,3}(21,26)=z^2w^7x^{12}y^3+z^2w^8x^{12}y^3$\\
$\mathcal{M}_{3,3}(21,27)=zw^8x^{10}y^3+zw^7x^{10}y^3+zw^9x^{10}y^3$\\
$\mathcal{M}_{3,3}(21,28)=0$\\
$\mathcal{M}_{3,3}(21,29)=z^2w^4x^{10}y^3+5z^3w^5x^{10}y^3+z^3w^6x^{12}y^3+zw^5x^{10}y^3+2z^2w^4x^8y^3+3z^3w^6x^{10}y^3+2z^2w^7x^{10}y^3+z^4w^4x^{10}y^3+z^2w^5x^8y^3$\\
$\mathcal{M}_{3,3}(21,30)=0$\\
$\mathcal{M}_{3,3}(21,31)=z^2w^5x^{10}y^3+z^2w^4x^{10}y^3+z^3w^5x^{10}y^3+z^3w^6x^{12}y^3+zw^5x^{10}y^3+2z^2w^7x^{10}y^3+z^4w^4x^{10}y^3+z^2w^6x^{10}y^3$\\
$\mathcal{M}_{3,3}(21,32)=z^4w^6x^{14}y^3+z^5w^5x^{12}y^3+z^5w^6x^{14}y^3+z^4w^4x^{10}y^3$\\
$\mathcal{M}_{3,3}(21,33)=2z^2w^5x^{10}y^3+z^2w^5x^{12}y^3+z^3w^7x^{14}y^3+2z^3w^6x^{12}y^3+z^2w^6x^{10}y^3+z^3w^7x^{12}y^3$\\
$\mathcal{M}_{3,3}(21,34)=z^2w^5x^{12}y^3+z^3w^7x^{14}y^3+z^3w^6x^{12}y^3+z^2w^7x^{12}y^3$\\
$\mathcal{M}_{3,3}(21,35)=z^2w^5x^{10}y^3+2zw^8x^{10}y^3+w^6x^{10}y^3+zw^6x^{10}y^3$\\
$\mathcal{M}_{3,3}(21,36)=0$\\
$\mathcal{M}_{3,3}(21,37)=2z^2w^4x^{10}y^3+2z^3w^6x^{12}y^3+2zw^5x^{10}y^3+3z^2w^7x^{10}y^3+2z^4w^4x^{10}y^3$\\
$\mathcal{M}_{3,3}(22,1)=0$\\
$\mathcal{M}_{3,3}(22,2)=2z^3w^5x^{10}y^3+z^2w^3x^6y^3+z^3w^6x^{12}y^3+2z^3w^6x^{10}y^3+z^2w^7x^{10}y^3+2z^3w^4x^8y^3+z^2w^5x^8y^3$\\
$\mathcal{M}_{3,3}(22,3)=0$\\
$\mathcal{M}_{3,3}(22,4)=3z^3w^5x^{10}y^3+2z^2w^4x^8y^3+2z^3w^6x^{10}y^3+z^2w^7x^{10}y^3+z^2w^6x^{10}y^3+z^2w^5x^8y^3$\\
$\mathcal{M}_{3,3}(22,5)=0$\\
$\mathcal{M}_{3,3}(22,6)=3z^2w^5x^{10}y^3+z^3w^6x^{12}y^3+z^2w^7x^{10}y^3+2z^2w^6x^{10}y^3$\\
$\mathcal{M}_{3,3}(22,7)=0$\\
$\mathcal{M}_{3,3}(22,8)=z^3w^5x^{10}y^3+z^3w^7x^{14}y^3+z^3w^6x^{12}y^3+z^2w^4x^8y^3$\\
$\mathcal{M}_{3,3}(22,9)=z^2w^5x^{10}y^3+2z^3w^6x^{12}y^3+z^2w^7x^{12}y^3$\\
$\mathcal{M}_{3,3}(22,10)=zw^8x^{10}y^3+zw^5x^8y^3+zw^6x^8y^3+2z^2w^7x^{10}y^3+z^2w^6x^{10}y^3$\\
$\mathcal{M}_{3,3}(22,11)=0$\\
$\mathcal{M}_{3,3}(22,12)=z^4w^6x^{14}y^3+z^3w^7x^{14}y^3+z^3w^6x^{12}y^3+z^5w^6x^{14}y^3+z^4w^5x^{12}y^3+z^3w^8x^{12}y^3+z^3w^7x^{12}y^3$\\
$\mathcal{M}_{3,3}(22,13)=2z^3w^5x^{12}y^3+z^2w^4x^{10}y^3+z^3w^7x^{14}y^3+z^2w^6x^{12}y^3+z^3w^6x^{14}y^3$\\
$\mathcal{M}_{3,3}(22,14)=2z^2w^6x^{14}y^3+z^2w^7x^{14}y^3$\\
$\mathcal{M}_{3,3}(22,15)=z^3w^7x^{14}y^3+z^3w^6x^{12}y^3+z^3w^8x^{12}y^3+z^3w^7x^{12}y^3$\\
$\mathcal{M}_{3,3}(22,16)=0$\\
$\mathcal{M}_{3,3}(22,17)=2z^2w^5x^{10}y^3+z^3w^6x^{12}y^3+z^2w^6x^{10}y^3+z^3w^7x^{12}y^3$\\
$\mathcal{M}_{3,3}(22,18)=z^3w^7x^{14}y^3+z^2w^6x^{12}y^3$\\
$\mathcal{M}_{3,3}(22,19)=zw^8x^{10}y^3+2zw^7x^{10}y^3+2zw^6x^{10}y^3$\\
$\mathcal{M}_{3,3}(22,20)=0$\\
$\mathcal{M}_{3,3}(22,21)=z^2w^5x^{10}y^3+z^3w^7x^{14}y^3+z^3w^6x^{12}y^3$\\
$\mathcal{M}_{3,3}(22,22)=z^2w^7x^{14}y^3$\\
$\mathcal{M}_{3,3}(22,23)=zw^8x^{14}y^3$\\
$\mathcal{M}_{3,3}(22,24)=z^4w^7x^{14}y^3+z^3w^6x^{12}y^3$\\
$\mathcal{M}_{3,3}(22,25)=z^2w^7x^{10}y^3+z^2w^6x^{10}y^3+z^3w^8x^{12}y^3$\\
$\mathcal{M}_{3,3}(22,26)=z^2w^7x^{12}y^3+z^2w^8x^{12}y^3$\\
$\mathcal{M}_{3,3}(22,27)=zw^8x^{10}y^3+zw^7x^{10}y^3+zw^9x^{10}y^3$\\
$\mathcal{M}_{3,3}(22,28)=0$\\
$\mathcal{M}_{3,3}(22,29)=z^2w^4x^{10}y^3+5z^3w^5x^{10}y^3+z^3w^6x^{12}y^3+zw^5x^{10}y^3+2z^2w^4x^8y^3+3z^3w^6x^{10}y^3+2z^2w^7x^{10}y^3+z^4w^4x^{10}y^3+z^2w^5x^8y^3$\\
$\mathcal{M}_{3,3}(22,30)=0$\\
$\mathcal{M}_{3,3}(22,31)=z^2w^5x^{10}y^3+z^2w^4x^{10}y^3+z^3w^5x^{10}y^3+z^3w^6x^{12}y^3+zw^5x^{10}y^3+2z^2w^7x^{10}y^3+z^4w^4x^{10}y^3+z^2w^6x^{10}y^3$\\
$\mathcal{M}_{3,3}(22,32)=z^4w^6x^{14}y^3+z^5w^5x^{12}y^3+z^5w^6x^{14}y^3+z^4w^4x^{10}y^3$\\
$\mathcal{M}_{3,3}(22,33)=2z^2w^5x^{10}y^3+z^2w^5x^{12}y^3+z^3w^7x^{14}y^3+2z^3w^6x^{12}y^3+z^2w^6x^{10}y^3+z^3w^7x^{12}y^3$\\
$\mathcal{M}_{3,3}(22,34)=z^2w^5x^{12}y^3+z^3w^7x^{14}y^3+z^3w^6x^{12}y^3+z^2w^7x^{12}y^3$\\
$\mathcal{M}_{3,3}(22,35)=z^2w^5x^{10}y^3+2zw^8x^{10}y^3+w^6x^{10}y^3+zw^6x^{10}y^3$\\
$\mathcal{M}_{3,3}(22,36)=0$\\
$\mathcal{M}_{3,3}(22,37)=2z^2w^4x^{10}y^3+2z^3w^6x^{12}y^3+2zw^5x^{10}y^3+3z^2w^7x^{10}y^3+2z^4w^4x^{10}y^3$\\
$\mathcal{M}_{3,3}(23,1)=0$\\
$\mathcal{M}_{3,3}(23,2)=2z^3w^5x^{10}y^3+z^2w^3x^6y^3+z^3w^6x^{12}y^3+2z^3w^6x^{10}y^3+z^2w^7x^{10}y^3+2z^3w^4x^8y^3+z^2w^5x^8y^3$\\
$\mathcal{M}_{3,3}(23,3)=0$\\
$\mathcal{M}_{3,3}(23,4)=3z^3w^5x^{10}y^3+2z^2w^4x^8y^3+2z^3w^6x^{10}y^3+z^2w^7x^{10}y^3+z^2w^6x^{10}y^3+z^2w^5x^8y^3$\\
$\mathcal{M}_{3,3}(23,5)=0$\\
$\mathcal{M}_{3,3}(23,6)=3z^2w^5x^{10}y^3+z^3w^6x^{12}y^3+z^2w^7x^{10}y^3+2z^2w^6x^{10}y^3$\\
$\mathcal{M}_{3,3}(23,7)=0$\\
$\mathcal{M}_{3,3}(23,8)=z^3w^5x^{10}y^3+z^3w^7x^{14}y^3+z^3w^6x^{12}y^3+z^2w^4x^8y^3$\\
$\mathcal{M}_{3,3}(23,9)=z^2w^5x^{10}y^3+2z^3w^6x^{12}y^3+z^2w^7x^{12}y^3$\\
$\mathcal{M}_{3,3}(23,10)=zw^8x^{10}y^3+zw^5x^8y^3+zw^6x^8y^3+2z^2w^7x^{10}y^3+z^2w^6x^{10}y^3$\\
$\mathcal{M}_{3,3}(23,11)=0$\\
$\mathcal{M}_{3,3}(23,12)=z^4w^6x^{14}y^3+z^3w^7x^{14}y^3+z^3w^6x^{12}y^3+z^5w^6x^{14}y^3+z^4w^5x^{12}y^3+z^3w^8x^{12}y^3+z^3w^7x^{12}y^3$\\
$\mathcal{M}_{3,3}(23,13)=2z^3w^5x^{12}y^3+z^2w^4x^{10}y^3+z^3w^7x^{14}y^3+z^2w^6x^{12}y^3+z^3w^6x^{14}y^3$\\
$\mathcal{M}_{3,3}(23,14)=2z^2w^6x^{14}y^3+z^2w^7x^{14}y^3$\\
$\mathcal{M}_{3,3}(23,15)=z^3w^7x^{14}y^3+z^3w^6x^{12}y^3+z^3w^8x^{12}y^3+z^3w^7x^{12}y^3$\\
$\mathcal{M}_{3,3}(23,16)=0$\\
$\mathcal{M}_{3,3}(23,17)=2z^2w^5x^{10}y^3+z^3w^6x^{12}y^3+z^2w^6x^{10}y^3+z^3w^7x^{12}y^3$\\
$\mathcal{M}_{3,3}(23,18)=z^3w^7x^{14}y^3+z^2w^6x^{12}y^3$\\
$\mathcal{M}_{3,3}(23,19)=zw^8x^{10}y^3+2zw^7x^{10}y^3+2zw^6x^{10}y^3$\\
$\mathcal{M}_{3,3}(23,20)=0$\\
$\mathcal{M}_{3,3}(23,21)=z^2w^5x^{10}y^3+z^3w^7x^{14}y^3+z^3w^6x^{12}y^3$\\
$\mathcal{M}_{3,3}(23,22)=z^2w^7x^{14}y^3$\\
$\mathcal{M}_{3,3}(23,23)=zw^8x^{14}y^3$\\
$\mathcal{M}_{3,3}(23,24)=z^4w^7x^{14}y^3+z^3w^6x^{12}y^3$\\
$\mathcal{M}_{3,3}(23,25)=z^2w^7x^{10}y^3+z^2w^6x^{10}y^3+z^3w^8x^{12}y^3$\\
$\mathcal{M}_{3,3}(23,26)=z^2w^7x^{12}y^3+z^2w^8x^{12}y^3$\\
$\mathcal{M}_{3,3}(23,27)=zw^8x^{10}y^3+zw^7x^{10}y^3+zw^9x^{10}y^3$\\
$\mathcal{M}_{3,3}(23,28)=0$\\
$\mathcal{M}_{3,3}(23,29)=z^2w^4x^{10}y^3+5z^3w^5x^{10}y^3+z^3w^6x^{12}y^3+zw^5x^{10}y^3+2z^2w^4x^8y^3+3z^3w^6x^{10}y^3+2z^2w^7x^{10}y^3+z^4w^4x^{10}y^3+z^2w^5x^8y^3$\\
$\mathcal{M}_{3,3}(23,30)=0$\\
$\mathcal{M}_{3,3}(23,31)=z^2w^5x^{10}y^3+z^2w^4x^{10}y^3+z^3w^5x^{10}y^3+z^3w^6x^{12}y^3+zw^5x^{10}y^3+2z^2w^7x^{10}y^3+z^4w^4x^{10}y^3+z^2w^6x^{10}y^3$\\
$\mathcal{M}_{3,3}(23,32)=z^4w^6x^{14}y^3+z^5w^5x^{12}y^3+z^5w^6x^{14}y^3+z^4w^4x^{10}y^3$\\
$\mathcal{M}_{3,3}(23,33)=2z^2w^5x^{10}y^3+z^2w^5x^{12}y^3+z^3w^7x^{14}y^3+2z^3w^6x^{12}y^3+z^2w^6x^{10}y^3+z^3w^7x^{12}y^3$\\
$\mathcal{M}_{3,3}(23,34)=z^2w^5x^{12}y^3+z^3w^7x^{14}y^3+z^3w^6x^{12}y^3+z^2w^7x^{12}y^3$\\
$\mathcal{M}_{3,3}(23,35)=z^2w^5x^{10}y^3+2zw^8x^{10}y^3+w^6x^{10}y^3+zw^6x^{10}y^3$\\
$\mathcal{M}_{3,3}(23,36)=0$\\
$\mathcal{M}_{3,3}(23,37)=2z^2w^4x^{10}y^3+2z^3w^6x^{12}y^3+2zw^5x^{10}y^3+3z^2w^7x^{10}y^3+2z^4w^4x^{10}y^3$\\
$\mathcal{M}_{3,3}(24,1)=0$\\
$\mathcal{M}_{3,3}(24,2)=2z^3w^5x^{10}y^3+z^2w^3x^6y^3+z^3w^6x^{12}y^3+2z^3w^6x^{10}y^3+z^2w^7x^{10}y^3+2z^3w^4x^8y^3+z^2w^5x^8y^3$\\
$\mathcal{M}_{3,3}(24,3)=0$\\
$\mathcal{M}_{3,3}(24,4)=3z^3w^5x^{10}y^3+2z^2w^4x^8y^3+2z^3w^6x^{10}y^3+z^2w^7x^{10}y^3+z^2w^6x^{10}y^3+z^2w^5x^8y^3$\\
$\mathcal{M}_{3,3}(24,5)=0$\\
$\mathcal{M}_{3,3}(24,6)=3z^2w^5x^{10}y^3+z^3w^6x^{12}y^3+z^2w^7x^{10}y^3+2z^2w^6x^{10}y^3$\\
$\mathcal{M}_{3,3}(24,7)=0$\\
$\mathcal{M}_{3,3}(24,8)=z^3w^5x^{10}y^3+z^3w^7x^{14}y^3+z^3w^6x^{12}y^3+z^2w^4x^8y^3$\\
$\mathcal{M}_{3,3}(24,9)=z^2w^5x^{10}y^3+2z^3w^6x^{12}y^3+z^2w^7x^{12}y^3$\\
$\mathcal{M}_{3,3}(24,10)=zw^8x^{10}y^3+zw^5x^8y^3+zw^6x^8y^3+2z^2w^7x^{10}y^3+z^2w^6x^{10}y^3$\\
$\mathcal{M}_{3,3}(24,11)=0$\\
$\mathcal{M}_{3,3}(24,12)=z^4w^6x^{14}y^3+z^3w^7x^{14}y^3+z^3w^6x^{12}y^3+z^5w^6x^{14}y^3+z^4w^5x^{12}y^3+z^3w^8x^{12}y^3+z^3w^7x^{12}y^3$\\
$\mathcal{M}_{3,3}(24,13)=2z^3w^5x^{12}y^3+z^2w^4x^{10}y^3+z^3w^7x^{14}y^3+z^2w^6x^{12}y^3+z^3w^6x^{14}y^3$\\
$\mathcal{M}_{3,3}(24,14)=2z^2w^6x^{14}y^3+z^2w^7x^{14}y^3$\\
$\mathcal{M}_{3,3}(24,15)=z^3w^7x^{14}y^3+z^3w^6x^{12}y^3+z^3w^8x^{12}y^3+z^3w^7x^{12}y^3$\\
$\mathcal{M}_{3,3}(24,16)=0$\\
$\mathcal{M}_{3,3}(24,17)=2z^2w^5x^{10}y^3+z^3w^6x^{12}y^3+z^2w^6x^{10}y^3+z^3w^7x^{12}y^3$\\
$\mathcal{M}_{3,3}(24,18)=z^3w^7x^{14}y^3+z^2w^6x^{12}y^3$\\
$\mathcal{M}_{3,3}(24,19)=zw^8x^{10}y^3+2zw^7x^{10}y^3+2zw^6x^{10}y^3$\\
$\mathcal{M}_{3,3}(24,20)=0$\\
$\mathcal{M}_{3,3}(24,21)=z^2w^5x^{10}y^3+z^3w^7x^{14}y^3+z^3w^6x^{12}y^3$\\
$\mathcal{M}_{3,3}(24,22)=z^2w^7x^{14}y^3$\\
$\mathcal{M}_{3,3}(24,23)=zw^8x^{14}y^3$\\
$\mathcal{M}_{3,3}(24,24)=z^4w^7x^{14}y^3+z^3w^6x^{12}y^3$\\
$\mathcal{M}_{3,3}(24,25)=z^2w^7x^{10}y^3+z^2w^6x^{10}y^3+z^3w^8x^{12}y^3$\\
$\mathcal{M}_{3,3}(24,26)=z^2w^7x^{12}y^3+z^2w^8x^{12}y^3$\\
$\mathcal{M}_{3,3}(24,27)=zw^8x^{10}y^3+zw^7x^{10}y^3+zw^9x^{10}y^3$\\
$\mathcal{M}_{3,3}(24,28)=0$\\
$\mathcal{M}_{3,3}(24,29)=z^2w^4x^{10}y^3+5z^3w^5x^{10}y^3+z^3w^6x^{12}y^3+zw^5x^{10}y^3+2z^2w^4x^8y^3+3z^3w^6x^{10}y^3+2z^2w^7x^{10}y^3+z^4w^4x^{10}y^3+z^2w^5x^8y^3$\\
$\mathcal{M}_{3,3}(24,30)=0$\\
$\mathcal{M}_{3,3}(24,31)=z^2w^5x^{10}y^3+z^2w^4x^{10}y^3+z^3w^5x^{10}y^3+z^3w^6x^{12}y^3+zw^5x^{10}y^3+2z^2w^7x^{10}y^3+z^4w^4x^{10}y^3+z^2w^6x^{10}y^3$\\
$\mathcal{M}_{3,3}(24,32)=z^4w^6x^{14}y^3+z^5w^5x^{12}y^3+z^5w^6x^{14}y^3+z^4w^4x^{10}y^3$\\
$\mathcal{M}_{3,3}(24,33)=2z^2w^5x^{10}y^3+z^2w^5x^{12}y^3+z^3w^7x^{14}y^3+2z^3w^6x^{12}y^3+z^2w^6x^{10}y^3+z^3w^7x^{12}y^3$\\
$\mathcal{M}_{3,3}(24,34)=z^2w^5x^{12}y^3+z^3w^7x^{14}y^3+z^3w^6x^{12}y^3+z^2w^7x^{12}y^3$\\
$\mathcal{M}_{3,3}(24,35)=z^2w^5x^{10}y^3+2zw^8x^{10}y^3+w^6x^{10}y^3+zw^6x^{10}y^3$\\
$\mathcal{M}_{3,3}(24,36)=0$\\
$\mathcal{M}_{3,3}(24,37)=2z^2w^4x^{10}y^3+2z^3w^6x^{12}y^3+2zw^5x^{10}y^3+3z^2w^7x^{10}y^3+2z^4w^4x^{10}y^3$\\
$\mathcal{M}_{3,3}(25,1)=0$\\
$\mathcal{M}_{3,3}(25,2)=2z^3w^5x^{10}y^3+z^2w^3x^6y^3+z^3w^6x^{12}y^3+2z^3w^6x^{10}y^3+z^2w^7x^{10}y^3+2z^3w^4x^8y^3+z^2w^5x^8y^3$\\
$\mathcal{M}_{3,3}(25,3)=0$\\
$\mathcal{M}_{3,3}(25,4)=3z^3w^5x^{10}y^3+2z^2w^4x^8y^3+2z^3w^6x^{10}y^3+z^2w^7x^{10}y^3+z^2w^6x^{10}y^3+z^2w^5x^8y^3$\\
$\mathcal{M}_{3,3}(25,5)=0$\\
$\mathcal{M}_{3,3}(25,6)=3z^2w^5x^{10}y^3+z^3w^6x^{12}y^3+z^2w^7x^{10}y^3+2z^2w^6x^{10}y^3$\\
$\mathcal{M}_{3,3}(25,7)=0$\\
$\mathcal{M}_{3,3}(25,8)=z^3w^5x^{10}y^3+z^3w^7x^{14}y^3+z^3w^6x^{12}y^3+z^2w^4x^8y^3$\\
$\mathcal{M}_{3,3}(25,9)=z^2w^5x^{10}y^3+2z^3w^6x^{12}y^3+z^2w^7x^{12}y^3$\\
$\mathcal{M}_{3,3}(25,10)=zw^8x^{10}y^3+zw^5x^8y^3+zw^6x^8y^3+2z^2w^7x^{10}y^3+z^2w^6x^{10}y^3$\\
$\mathcal{M}_{3,3}(25,11)=0$\\
$\mathcal{M}_{3,3}(25,12)=z^4w^6x^{14}y^3+z^3w^7x^{14}y^3+z^3w^6x^{12}y^3+z^5w^6x^{14}y^3+z^4w^5x^{12}y^3+z^3w^8x^{12}y^3+z^3w^7x^{12}y^3$\\
$\mathcal{M}_{3,3}(25,13)=2z^3w^5x^{12}y^3+z^2w^4x^{10}y^3+z^3w^7x^{14}y^3+z^2w^6x^{12}y^3+z^3w^6x^{14}y^3$\\
$\mathcal{M}_{3,3}(25,14)=2z^2w^6x^{14}y^3+z^2w^7x^{14}y^3$\\
$\mathcal{M}_{3,3}(25,15)=z^3w^7x^{14}y^3+z^3w^6x^{12}y^3+z^3w^8x^{12}y^3+z^3w^7x^{12}y^3$\\
$\mathcal{M}_{3,3}(25,16)=0$\\
$\mathcal{M}_{3,3}(25,17)=2z^2w^5x^{10}y^3+z^3w^6x^{12}y^3+z^2w^6x^{10}y^3+z^3w^7x^{12}y^3$\\
$\mathcal{M}_{3,3}(25,18)=z^3w^7x^{14}y^3+z^2w^6x^{12}y^3$\\
$\mathcal{M}_{3,3}(25,19)=zw^8x^{10}y^3+2zw^7x^{10}y^3+2zw^6x^{10}y^3$\\
$\mathcal{M}_{3,3}(25,20)=0$\\
$\mathcal{M}_{3,3}(25,21)=z^2w^5x^{10}y^3+z^3w^7x^{14}y^3+z^3w^6x^{12}y^3$\\
$\mathcal{M}_{3,3}(25,22)=z^2w^7x^{14}y^3$\\
$\mathcal{M}_{3,3}(25,23)=zw^8x^{14}y^3$\\
$\mathcal{M}_{3,3}(25,24)=z^4w^7x^{14}y^3+z^3w^6x^{12}y^3$\\
$\mathcal{M}_{3,3}(25,25)=z^2w^7x^{10}y^3+z^2w^6x^{10}y^3+z^3w^8x^{12}y^3$\\
$\mathcal{M}_{3,3}(25,26)=z^2w^7x^{12}y^3+z^2w^8x^{12}y^3$\\
$\mathcal{M}_{3,3}(25,27)=zw^8x^{10}y^3+zw^7x^{10}y^3+zw^9x^{10}y^3$\\
$\mathcal{M}_{3,3}(25,28)=0$\\
$\mathcal{M}_{3,3}(25,29)=z^2w^4x^{10}y^3+5z^3w^5x^{10}y^3+z^3w^6x^{12}y^3+zw^5x^{10}y^3+2z^2w^4x^8y^3+3z^3w^6x^{10}y^3+2z^2w^7x^{10}y^3+z^4w^4x^{10}y^3+z^2w^5x^8y^3$\\
$\mathcal{M}_{3,3}(25,30)=0$\\
$\mathcal{M}_{3,3}(25,31)=z^2w^5x^{10}y^3+z^2w^4x^{10}y^3+z^3w^5x^{10}y^3+z^3w^6x^{12}y^3+zw^5x^{10}y^3+2z^2w^7x^{10}y^3+z^4w^4x^{10}y^3+z^2w^6x^{10}y^3$\\
$\mathcal{M}_{3,3}(25,32)=z^4w^6x^{14}y^3+z^5w^5x^{12}y^3+z^5w^6x^{14}y^3+z^4w^4x^{10}y^3$\\
$\mathcal{M}_{3,3}(25,33)=2z^2w^5x^{10}y^3+z^2w^5x^{12}y^3+z^3w^7x^{14}y^3+2z^3w^6x^{12}y^3+z^2w^6x^{10}y^3+z^3w^7x^{12}y^3$\\
$\mathcal{M}_{3,3}(25,34)=z^2w^5x^{12}y^3+z^3w^7x^{14}y^3+z^3w^6x^{12}y^3+z^2w^7x^{12}y^3$\\
$\mathcal{M}_{3,3}(25,35)=z^2w^5x^{10}y^3+2zw^8x^{10}y^3+w^6x^{10}y^3+zw^6x^{10}y^3$\\
$\mathcal{M}_{3,3}(25,36)=0$\\
$\mathcal{M}_{3,3}(25,37)=2z^2w^4x^{10}y^3+2z^3w^6x^{12}y^3+2zw^5x^{10}y^3+3z^2w^7x^{10}y^3+2z^4w^4x^{10}y^3$\\
$\mathcal{M}_{3,3}(26,1)=0$\\
$\mathcal{M}_{3,3}(26,2)=2z^3w^5x^{10}y^3+z^2w^3x^6y^3+z^3w^6x^{12}y^3+2z^3w^6x^{10}y^3+z^2w^7x^{10}y^3+2z^3w^4x^8y^3+z^2w^5x^8y^3$\\
$\mathcal{M}_{3,3}(26,3)=0$\\
$\mathcal{M}_{3,3}(26,4)=3z^3w^5x^{10}y^3+2z^2w^4x^8y^3+2z^3w^6x^{10}y^3+z^2w^7x^{10}y^3+z^2w^6x^{10}y^3+z^2w^5x^8y^3$\\
$\mathcal{M}_{3,3}(26,5)=0$\\
$\mathcal{M}_{3,3}(26,6)=3z^2w^5x^{10}y^3+z^3w^6x^{12}y^3+z^2w^7x^{10}y^3+2z^2w^6x^{10}y^3$\\
$\mathcal{M}_{3,3}(26,7)=0$\\
$\mathcal{M}_{3,3}(26,8)=z^3w^5x^{10}y^3+z^3w^7x^{14}y^3+z^3w^6x^{12}y^3+z^2w^4x^8y^3$\\
$\mathcal{M}_{3,3}(26,9)=z^2w^5x^{10}y^3+2z^3w^6x^{12}y^3+z^2w^7x^{12}y^3$\\
$\mathcal{M}_{3,3}(26,10)=zw^8x^{10}y^3+zw^5x^8y^3+zw^6x^8y^3+2z^2w^7x^{10}y^3+z^2w^6x^{10}y^3$\\
$\mathcal{M}_{3,3}(26,11)=0$\\
$\mathcal{M}_{3,3}(26,12)=z^4w^6x^{14}y^3+z^3w^7x^{14}y^3+z^3w^6x^{12}y^3+z^5w^6x^{14}y^3+z^4w^5x^{12}y^3+z^3w^8x^{12}y^3+z^3w^7x^{12}y^3$\\
$\mathcal{M}_{3,3}(26,13)=2z^3w^5x^{12}y^3+z^2w^4x^{10}y^3+z^3w^7x^{14}y^3+z^2w^6x^{12}y^3+z^3w^6x^{14}y^3$\\
$\mathcal{M}_{3,3}(26,14)=2z^2w^6x^{14}y^3+z^2w^7x^{14}y^3$\\
$\mathcal{M}_{3,3}(26,15)=z^3w^7x^{14}y^3+z^3w^6x^{12}y^3+z^3w^8x^{12}y^3+z^3w^7x^{12}y^3$\\
$\mathcal{M}_{3,3}(26,16)=0$\\
$\mathcal{M}_{3,3}(26,17)=2z^2w^5x^{10}y^3+z^3w^6x^{12}y^3+z^2w^6x^{10}y^3+z^3w^7x^{12}y^3$\\
$\mathcal{M}_{3,3}(26,18)=z^3w^7x^{14}y^3+z^2w^6x^{12}y^3$\\
$\mathcal{M}_{3,3}(26,19)=zw^8x^{10}y^3+2zw^7x^{10}y^3+2zw^6x^{10}y^3$\\
$\mathcal{M}_{3,3}(26,20)=0$\\
$\mathcal{M}_{3,3}(26,21)=z^2w^5x^{10}y^3+z^3w^7x^{14}y^3+z^3w^6x^{12}y^3$\\
$\mathcal{M}_{3,3}(26,22)=z^2w^7x^{14}y^3$\\
$\mathcal{M}_{3,3}(26,23)=zw^8x^{14}y^3$\\
$\mathcal{M}_{3,3}(26,24)=z^4w^7x^{14}y^3+z^3w^6x^{12}y^3$\\
$\mathcal{M}_{3,3}(26,25)=z^2w^7x^{10}y^3+z^2w^6x^{10}y^3+z^3w^8x^{12}y^3$\\
$\mathcal{M}_{3,3}(26,26)=z^2w^7x^{12}y^3+z^2w^8x^{12}y^3$\\
$\mathcal{M}_{3,3}(26,27)=zw^8x^{10}y^3+zw^7x^{10}y^3+zw^9x^{10}y^3$\\
$\mathcal{M}_{3,3}(26,28)=0$\\
$\mathcal{M}_{3,3}(26,29)=z^2w^4x^{10}y^3+5z^3w^5x^{10}y^3+z^3w^6x^{12}y^3+zw^5x^{10}y^3+2z^2w^4x^8y^3+3z^3w^6x^{10}y^3+2z^2w^7x^{10}y^3+z^4w^4x^{10}y^3+z^2w^5x^8y^3$\\
$\mathcal{M}_{3,3}(26,30)=0$\\
$\mathcal{M}_{3,3}(26,31)=z^2w^5x^{10}y^3+z^2w^4x^{10}y^3+z^3w^5x^{10}y^3+z^3w^6x^{12}y^3+zw^5x^{10}y^3+2z^2w^7x^{10}y^3+z^4w^4x^{10}y^3+z^2w^6x^{10}y^3$\\
$\mathcal{M}_{3,3}(26,32)=z^4w^6x^{14}y^3+z^5w^5x^{12}y^3+z^5w^6x^{14}y^3+z^4w^4x^{10}y^3$\\
$\mathcal{M}_{3,3}(26,33)=2z^2w^5x^{10}y^3+z^2w^5x^{12}y^3+z^3w^7x^{14}y^3+2z^3w^6x^{12}y^3+z^2w^6x^{10}y^3+z^3w^7x^{12}y^3$\\
$\mathcal{M}_{3,3}(26,34)=z^2w^5x^{12}y^3+z^3w^7x^{14}y^3+z^3w^6x^{12}y^3+z^2w^7x^{12}y^3$\\
$\mathcal{M}_{3,3}(26,35)=z^2w^5x^{10}y^3+2zw^8x^{10}y^3+w^6x^{10}y^3+zw^6x^{10}y^3$\\
$\mathcal{M}_{3,3}(26,36)=0$\\
$\mathcal{M}_{3,3}(26,37)=2z^2w^4x^{10}y^3+2z^3w^6x^{12}y^3+2zw^5x^{10}y^3+3z^2w^7x^{10}y^3+2z^4w^4x^{10}y^3$\\
$\mathcal{M}_{3,3}(27,1)=0$\\
$\mathcal{M}_{3,3}(27,2)=2z^3w^5x^{10}y^3+z^2w^3x^6y^3+z^3w^6x^{12}y^3+2z^3w^6x^{10}y^3+z^2w^7x^{10}y^3+2z^3w^4x^8y^3+z^2w^5x^8y^3$\\
$\mathcal{M}_{3,3}(27,3)=0$\\
$\mathcal{M}_{3,3}(27,4)=3z^3w^5x^{10}y^3+2z^2w^4x^8y^3+2z^3w^6x^{10}y^3+z^2w^7x^{10}y^3+z^2w^6x^{10}y^3+z^2w^5x^8y^3$\\
$\mathcal{M}_{3,3}(27,5)=0$\\
$\mathcal{M}_{3,3}(27,6)=3z^2w^5x^{10}y^3+z^3w^6x^{12}y^3+z^2w^7x^{10}y^3+2z^2w^6x^{10}y^3$\\
$\mathcal{M}_{3,3}(27,7)=0$\\
$\mathcal{M}_{3,3}(27,8)=z^3w^5x^{10}y^3+z^3w^7x^{14}y^3+z^3w^6x^{12}y^3+z^2w^4x^8y^3$\\
$\mathcal{M}_{3,3}(27,9)=z^2w^5x^{10}y^3+2z^3w^6x^{12}y^3+z^2w^7x^{12}y^3$\\
$\mathcal{M}_{3,3}(27,10)=zw^8x^{10}y^3+zw^5x^8y^3+zw^6x^8y^3+2z^2w^7x^{10}y^3+z^2w^6x^{10}y^3$\\
$\mathcal{M}_{3,3}(27,11)=0$\\
$\mathcal{M}_{3,3}(27,12)=z^4w^6x^{14}y^3+z^3w^7x^{14}y^3+z^3w^6x^{12}y^3+z^5w^6x^{14}y^3+z^4w^5x^{12}y^3+z^3w^8x^{12}y^3+z^3w^7x^{12}y^3$\\
$\mathcal{M}_{3,3}(27,13)=2z^3w^5x^{12}y^3+z^2w^4x^{10}y^3+z^3w^7x^{14}y^3+z^2w^6x^{12}y^3+z^3w^6x^{14}y^3$\\
$\mathcal{M}_{3,3}(27,14)=2z^2w^6x^{14}y^3+z^2w^7x^{14}y^3$\\
$\mathcal{M}_{3,3}(27,15)=z^3w^7x^{14}y^3+z^3w^6x^{12}y^3+z^3w^8x^{12}y^3+z^3w^7x^{12}y^3$\\
$\mathcal{M}_{3,3}(27,16)=0$\\
$\mathcal{M}_{3,3}(27,17)=2z^2w^5x^{10}y^3+z^3w^6x^{12}y^3+z^2w^6x^{10}y^3+z^3w^7x^{12}y^3$\\
$\mathcal{M}_{3,3}(27,18)=z^3w^7x^{14}y^3+z^2w^6x^{12}y^3$\\
$\mathcal{M}_{3,3}(27,19)=zw^8x^{10}y^3+2zw^7x^{10}y^3+2zw^6x^{10}y^3$\\
$\mathcal{M}_{3,3}(27,20)=0$\\
$\mathcal{M}_{3,3}(27,21)=z^2w^5x^{10}y^3+z^3w^7x^{14}y^3+z^3w^6x^{12}y^3$\\
$\mathcal{M}_{3,3}(27,22)=z^2w^7x^{14}y^3$\\
$\mathcal{M}_{3,3}(27,23)=zw^8x^{14}y^3$\\
$\mathcal{M}_{3,3}(27,24)=z^4w^7x^{14}y^3+z^3w^6x^{12}y^3$\\
$\mathcal{M}_{3,3}(27,25)=z^2w^7x^{10}y^3+z^2w^6x^{10}y^3+z^3w^8x^{12}y^3$\\
$\mathcal{M}_{3,3}(27,26)=z^2w^7x^{12}y^3+z^2w^8x^{12}y^3$\\
$\mathcal{M}_{3,3}(27,27)=zw^8x^{10}y^3+zw^7x^{10}y^3+zw^9x^{10}y^3$\\
$\mathcal{M}_{3,3}(27,28)=0$\\
$\mathcal{M}_{3,3}(27,29)=z^2w^4x^{10}y^3+5z^3w^5x^{10}y^3+z^3w^6x^{12}y^3+zw^5x^{10}y^3+2z^2w^4x^8y^3+3z^3w^6x^{10}y^3+2z^2w^7x^{10}y^3+z^4w^4x^{10}y^3+z^2w^5x^8y^3$\\
$\mathcal{M}_{3,3}(27,30)=0$\\
$\mathcal{M}_{3,3}(27,31)=z^2w^5x^{10}y^3+z^2w^4x^{10}y^3+z^3w^5x^{10}y^3+z^3w^6x^{12}y^3+zw^5x^{10}y^3+2z^2w^7x^{10}y^3+z^4w^4x^{10}y^3+z^2w^6x^{10}y^3$\\
$\mathcal{M}_{3,3}(27,32)=z^4w^6x^{14}y^3+z^5w^5x^{12}y^3+z^5w^6x^{14}y^3+z^4w^4x^{10}y^3$\\
$\mathcal{M}_{3,3}(27,33)=2z^2w^5x^{10}y^3+z^2w^5x^{12}y^3+z^3w^7x^{14}y^3+2z^3w^6x^{12}y^3+z^2w^6x^{10}y^3+z^3w^7x^{12}y^3$\\
$\mathcal{M}_{3,3}(27,34)=z^2w^5x^{12}y^3+z^3w^7x^{14}y^3+z^3w^6x^{12}y^3+z^2w^7x^{12}y^3$\\
$\mathcal{M}_{3,3}(27,35)=z^2w^5x^{10}y^3+2zw^8x^{10}y^3+w^6x^{10}y^3+zw^6x^{10}y^3$\\
$\mathcal{M}_{3,3}(27,36)=0$\\
$\mathcal{M}_{3,3}(27,37)=2z^2w^4x^{10}y^3+2z^3w^6x^{12}y^3+2zw^5x^{10}y^3+3z^2w^7x^{10}y^3+2z^4w^4x^{10}y^3$\\
$\mathcal{M}_{3,3}(28,1)=0$\\
$\mathcal{M}_{3,3}(28,2)=2z^3w^5x^{10}y^3+z^2w^3x^6y^3+z^3w^6x^{12}y^3+2z^3w^6x^{10}y^3+z^2w^7x^{10}y^3+2z^3w^4x^8y^3+z^2w^5x^8y^3$\\
$\mathcal{M}_{3,3}(28,3)=0$\\
$\mathcal{M}_{3,3}(28,4)=3z^3w^5x^{10}y^3+2z^2w^4x^8y^3+2z^3w^6x^{10}y^3+z^2w^7x^{10}y^3+z^2w^6x^{10}y^3+z^2w^5x^8y^3$\\
$\mathcal{M}_{3,3}(28,5)=0$\\
$\mathcal{M}_{3,3}(28,6)=3z^2w^5x^{10}y^3+z^3w^6x^{12}y^3+z^2w^7x^{10}y^3+2z^2w^6x^{10}y^3$\\
$\mathcal{M}_{3,3}(28,7)=0$\\
$\mathcal{M}_{3,3}(28,8)=z^3w^5x^{10}y^3+z^3w^7x^{14}y^3+z^3w^6x^{12}y^3+z^2w^4x^8y^3$\\
$\mathcal{M}_{3,3}(28,9)=z^2w^5x^{10}y^3+2z^3w^6x^{12}y^3+z^2w^7x^{12}y^3$\\
$\mathcal{M}_{3,3}(28,10)=zw^8x^{10}y^3+zw^5x^8y^3+zw^6x^8y^3+2z^2w^7x^{10}y^3+z^2w^6x^{10}y^3$\\
$\mathcal{M}_{3,3}(28,11)=0$\\
$\mathcal{M}_{3,3}(28,12)=z^4w^6x^{14}y^3+z^3w^7x^{14}y^3+z^3w^6x^{12}y^3+z^5w^6x^{14}y^3+z^4w^5x^{12}y^3+z^3w^8x^{12}y^3+z^3w^7x^{12}y^3$\\
$\mathcal{M}_{3,3}(28,13)=2z^3w^5x^{12}y^3+z^2w^4x^{10}y^3+z^3w^7x^{14}y^3+z^2w^6x^{12}y^3+z^3w^6x^{14}y^3$\\
$\mathcal{M}_{3,3}(28,14)=2z^2w^6x^{14}y^3+z^2w^7x^{14}y^3$\\
$\mathcal{M}_{3,3}(28,15)=z^3w^7x^{14}y^3+z^3w^6x^{12}y^3+z^3w^8x^{12}y^3+z^3w^7x^{12}y^3$\\
$\mathcal{M}_{3,3}(28,16)=0$\\
$\mathcal{M}_{3,3}(28,17)=2z^2w^5x^{10}y^3+z^3w^6x^{12}y^3+z^2w^6x^{10}y^3+z^3w^7x^{12}y^3$\\
$\mathcal{M}_{3,3}(28,18)=z^3w^7x^{14}y^3+z^2w^6x^{12}y^3$\\
$\mathcal{M}_{3,3}(28,19)=zw^8x^{10}y^3+2zw^7x^{10}y^3+2zw^6x^{10}y^3$\\
$\mathcal{M}_{3,3}(28,20)=0$\\
$\mathcal{M}_{3,3}(28,21)=z^2w^5x^{10}y^3+z^3w^7x^{14}y^3+z^3w^6x^{12}y^3$\\
$\mathcal{M}_{3,3}(28,22)=z^2w^7x^{14}y^3$\\
$\mathcal{M}_{3,3}(28,23)=zw^8x^{14}y^3$\\
$\mathcal{M}_{3,3}(28,24)=z^4w^7x^{14}y^3+z^3w^6x^{12}y^3$\\
$\mathcal{M}_{3,3}(28,25)=z^2w^7x^{10}y^3+z^2w^6x^{10}y^3+z^3w^8x^{12}y^3$\\
$\mathcal{M}_{3,3}(28,26)=z^2w^7x^{12}y^3+z^2w^8x^{12}y^3$\\
$\mathcal{M}_{3,3}(28,27)=zw^8x^{10}y^3+zw^7x^{10}y^3+zw^9x^{10}y^3$\\
$\mathcal{M}_{3,3}(28,28)=0$\\
$\mathcal{M}_{3,3}(28,29)=z^2w^4x^{10}y^3+5z^3w^5x^{10}y^3+z^3w^6x^{12}y^3+zw^5x^{10}y^3+2z^2w^4x^8y^3+3z^3w^6x^{10}y^3+2z^2w^7x^{10}y^3+z^4w^4x^{10}y^3+z^2w^5x^8y^3$\\
$\mathcal{M}_{3,3}(28,30)=0$\\
$\mathcal{M}_{3,3}(28,31)=z^2w^5x^{10}y^3+z^2w^4x^{10}y^3+z^3w^5x^{10}y^3+z^3w^6x^{12}y^3+zw^5x^{10}y^3+2z^2w^7x^{10}y^3+z^4w^4x^{10}y^3+z^2w^6x^{10}y^3$\\
$\mathcal{M}_{3,3}(28,32)=z^4w^6x^{14}y^3+z^5w^5x^{12}y^3+z^5w^6x^{14}y^3+z^4w^4x^{10}y^3$\\
$\mathcal{M}_{3,3}(28,33)=2z^2w^5x^{10}y^3+z^2w^5x^{12}y^3+z^3w^7x^{14}y^3+2z^3w^6x^{12}y^3+z^2w^6x^{10}y^3+z^3w^7x^{12}y^3$\\
$\mathcal{M}_{3,3}(28,34)=z^2w^5x^{12}y^3+z^3w^7x^{14}y^3+z^3w^6x^{12}y^3+z^2w^7x^{12}y^3$\\
$\mathcal{M}_{3,3}(28,35)=z^2w^5x^{10}y^3+2zw^8x^{10}y^3+w^6x^{10}y^3+zw^6x^{10}y^3$\\
$\mathcal{M}_{3,3}(28,36)=0$\\
$\mathcal{M}_{3,3}(28,37)=2z^2w^4x^{10}y^3+2z^3w^6x^{12}y^3+2zw^5x^{10}y^3+3z^2w^7x^{10}y^3+2z^4w^4x^{10}y^3$\\
$\mathcal{M}_{3,3}(29,1)=0$\\
$\mathcal{M}_{3,3}(29,2)=2z^3w^5x^{10}y^3+z^2w^3x^6y^3+z^3w^6x^{12}y^3+2z^3w^6x^{10}y^3+z^2w^7x^{10}y^3+2z^3w^4x^8y^3+z^2w^5x^8y^3$\\
$\mathcal{M}_{3,3}(29,3)=0$\\
$\mathcal{M}_{3,3}(29,4)=3z^3w^5x^{10}y^3+2z^2w^4x^8y^3+2z^3w^6x^{10}y^3+z^2w^7x^{10}y^3+z^2w^6x^{10}y^3+z^2w^5x^8y^3$\\
$\mathcal{M}_{3,3}(29,5)=0$\\
$\mathcal{M}_{3,3}(29,6)=3z^2w^5x^{10}y^3+z^3w^6x^{12}y^3+z^2w^7x^{10}y^3+2z^2w^6x^{10}y^3$\\
$\mathcal{M}_{3,3}(29,7)=0$\\
$\mathcal{M}_{3,3}(29,8)=z^3w^5x^{10}y^3+z^3w^7x^{14}y^3+z^3w^6x^{12}y^3+z^2w^4x^8y^3$\\
$\mathcal{M}_{3,3}(29,9)=z^2w^5x^{10}y^3+2z^3w^6x^{12}y^3+z^2w^7x^{12}y^3$\\
$\mathcal{M}_{3,3}(29,10)=zw^8x^{10}y^3+zw^5x^8y^3+zw^6x^8y^3+2z^2w^7x^{10}y^3+z^2w^6x^{10}y^3$\\
$\mathcal{M}_{3,3}(29,11)=0$\\
$\mathcal{M}_{3,3}(29,12)=z^4w^6x^{14}y^3+z^3w^7x^{14}y^3+z^3w^6x^{12}y^3+z^5w^6x^{14}y^3+z^4w^5x^{12}y^3+z^3w^8x^{12}y^3+z^3w^7x^{12}y^3$\\
$\mathcal{M}_{3,3}(29,13)=2z^3w^5x^{12}y^3+z^2w^4x^{10}y^3+z^3w^7x^{14}y^3+z^2w^6x^{12}y^3+z^3w^6x^{14}y^3$\\
$\mathcal{M}_{3,3}(29,14)=2z^2w^6x^{14}y^3+z^2w^7x^{14}y^3$\\
$\mathcal{M}_{3,3}(29,15)=z^3w^7x^{14}y^3+z^3w^6x^{12}y^3+z^3w^8x^{12}y^3+z^3w^7x^{12}y^3$\\
$\mathcal{M}_{3,3}(29,16)=0$\\
$\mathcal{M}_{3,3}(29,17)=2z^2w^5x^{10}y^3+z^3w^6x^{12}y^3+z^2w^6x^{10}y^3+z^3w^7x^{12}y^3$\\
$\mathcal{M}_{3,3}(29,18)=z^3w^7x^{14}y^3+z^2w^6x^{12}y^3$\\
$\mathcal{M}_{3,3}(29,19)=zw^8x^{10}y^3+2zw^7x^{10}y^3+2zw^6x^{10}y^3$\\
$\mathcal{M}_{3,3}(29,20)=0$\\
$\mathcal{M}_{3,3}(29,21)=z^2w^5x^{10}y^3+z^3w^7x^{14}y^3+z^3w^6x^{12}y^3$\\
$\mathcal{M}_{3,3}(29,22)=z^2w^7x^{14}y^3$\\
$\mathcal{M}_{3,3}(29,23)=zw^8x^{14}y^3$\\
$\mathcal{M}_{3,3}(29,24)=z^4w^7x^{14}y^3+z^3w^6x^{12}y^3$\\
$\mathcal{M}_{3,3}(29,25)=z^2w^7x^{10}y^3+z^2w^6x^{10}y^3+z^3w^8x^{12}y^3$\\
$\mathcal{M}_{3,3}(29,26)=z^2w^7x^{12}y^3+z^2w^8x^{12}y^3$\\
$\mathcal{M}_{3,3}(29,27)=zw^8x^{10}y^3+zw^7x^{10}y^3+zw^9x^{10}y^3$\\
$\mathcal{M}_{3,3}(29,28)=0$\\
$\mathcal{M}_{3,3}(29,29)=z^2w^4x^{10}y^3+5z^3w^5x^{10}y^3+z^3w^6x^{12}y^3+zw^5x^{10}y^3+2z^2w^4x^8y^3+3z^3w^6x^{10}y^3+2z^2w^7x^{10}y^3+z^4w^4x^{10}y^3+z^2w^5x^8y^3$\\
$\mathcal{M}_{3,3}(29,30)=0$\\
$\mathcal{M}_{3,3}(29,31)=z^2w^5x^{10}y^3+z^2w^4x^{10}y^3+z^3w^5x^{10}y^3+z^3w^6x^{12}y^3+zw^5x^{10}y^3+2z^2w^7x^{10}y^3+z^4w^4x^{10}y^3+z^2w^6x^{10}y^3$\\
$\mathcal{M}_{3,3}(29,32)=z^4w^6x^{14}y^3+z^5w^5x^{12}y^3+z^5w^6x^{14}y^3+z^4w^4x^{10}y^3$\\
$\mathcal{M}_{3,3}(29,33)=2z^2w^5x^{10}y^3+z^2w^5x^{12}y^3+z^3w^7x^{14}y^3+2z^3w^6x^{12}y^3+z^2w^6x^{10}y^3+z^3w^7x^{12}y^3$\\
$\mathcal{M}_{3,3}(29,34)=z^2w^5x^{12}y^3+z^3w^7x^{14}y^3+z^3w^6x^{12}y^3+z^2w^7x^{12}y^3$\\
$\mathcal{M}_{3,3}(29,35)=z^2w^5x^{10}y^3+2zw^8x^{10}y^3+w^6x^{10}y^3+zw^6x^{10}y^3$\\
$\mathcal{M}_{3,3}(29,36)=0$\\
$\mathcal{M}_{3,3}(29,37)=2z^2w^4x^{10}y^3+2z^3w^6x^{12}y^3+2zw^5x^{10}y^3+3z^2w^7x^{10}y^3+2z^4w^4x^{10}y^3$\\
$\mathcal{M}_{3,3}(30,1)=0$\\
$\mathcal{M}_{3,3}(30,2)=2z^3w^5x^{10}y^3+z^2w^3x^6y^3+z^3w^6x^{12}y^3+2z^3w^6x^{10}y^3+z^2w^7x^{10}y^3+2z^3w^4x^8y^3+z^2w^5x^8y^3$\\
$\mathcal{M}_{3,3}(30,3)=0$\\
$\mathcal{M}_{3,3}(30,4)=3z^3w^5x^{10}y^3+2z^2w^4x^8y^3+2z^3w^6x^{10}y^3+z^2w^7x^{10}y^3+z^2w^6x^{10}y^3+z^2w^5x^8y^3$\\
$\mathcal{M}_{3,3}(30,5)=0$\\
$\mathcal{M}_{3,3}(30,6)=3z^2w^5x^{10}y^3+z^3w^6x^{12}y^3+z^2w^7x^{10}y^3+2z^2w^6x^{10}y^3$\\
$\mathcal{M}_{3,3}(30,7)=0$\\
$\mathcal{M}_{3,3}(30,8)=z^3w^5x^{10}y^3+z^3w^7x^{14}y^3+z^3w^6x^{12}y^3+z^2w^4x^8y^3$\\
$\mathcal{M}_{3,3}(30,9)=z^2w^5x^{10}y^3+2z^3w^6x^{12}y^3+z^2w^7x^{12}y^3$\\
$\mathcal{M}_{3,3}(30,10)=zw^8x^{10}y^3+zw^5x^8y^3+zw^6x^8y^3+2z^2w^7x^{10}y^3+z^2w^6x^{10}y^3$\\
$\mathcal{M}_{3,3}(30,11)=0$\\
$\mathcal{M}_{3,3}(30,12)=z^4w^6x^{14}y^3+z^3w^7x^{14}y^3+z^3w^6x^{12}y^3+z^5w^6x^{14}y^3+z^4w^5x^{12}y^3+z^3w^8x^{12}y^3+z^3w^7x^{12}y^3$\\
$\mathcal{M}_{3,3}(30,13)=2z^3w^5x^{12}y^3+z^2w^4x^{10}y^3+z^3w^7x^{14}y^3+z^2w^6x^{12}y^3+z^3w^6x^{14}y^3$\\
$\mathcal{M}_{3,3}(30,14)=2z^2w^6x^{14}y^3+z^2w^7x^{14}y^3$\\
$\mathcal{M}_{3,3}(30,15)=z^3w^7x^{14}y^3+z^3w^6x^{12}y^3+z^3w^8x^{12}y^3+z^3w^7x^{12}y^3$\\
$\mathcal{M}_{3,3}(30,16)=0$\\
$\mathcal{M}_{3,3}(30,17)=2z^2w^5x^{10}y^3+z^3w^6x^{12}y^3+z^2w^6x^{10}y^3+z^3w^7x^{12}y^3$\\
$\mathcal{M}_{3,3}(30,18)=z^3w^7x^{14}y^3+z^2w^6x^{12}y^3$\\
$\mathcal{M}_{3,3}(30,19)=zw^8x^{10}y^3+2zw^7x^{10}y^3+2zw^6x^{10}y^3$\\
$\mathcal{M}_{3,3}(30,20)=0$\\
$\mathcal{M}_{3,3}(30,21)=z^2w^5x^{10}y^3+z^3w^7x^{14}y^3+z^3w^6x^{12}y^3$\\
$\mathcal{M}_{3,3}(30,22)=z^2w^7x^{14}y^3$\\
$\mathcal{M}_{3,3}(30,23)=zw^8x^{14}y^3$\\
$\mathcal{M}_{3,3}(30,24)=z^4w^7x^{14}y^3+z^3w^6x^{12}y^3$\\
$\mathcal{M}_{3,3}(30,25)=z^2w^7x^{10}y^3+z^2w^6x^{10}y^3+z^3w^8x^{12}y^3$\\
$\mathcal{M}_{3,3}(30,26)=z^2w^7x^{12}y^3+z^2w^8x^{12}y^3$\\
$\mathcal{M}_{3,3}(30,27)=zw^8x^{10}y^3+zw^7x^{10}y^3+zw^9x^{10}y^3$\\
$\mathcal{M}_{3,3}(30,28)=0$\\
$\mathcal{M}_{3,3}(30,29)=z^2w^4x^{10}y^3+5z^3w^5x^{10}y^3+z^3w^6x^{12}y^3+zw^5x^{10}y^3+2z^2w^4x^8y^3+3z^3w^6x^{10}y^3+2z^2w^7x^{10}y^3+z^4w^4x^{10}y^3+z^2w^5x^8y^3$\\
$\mathcal{M}_{3,3}(30,30)=0$\\
$\mathcal{M}_{3,3}(30,31)=z^2w^5x^{10}y^3+z^2w^4x^{10}y^3+z^3w^5x^{10}y^3+z^3w^6x^{12}y^3+zw^5x^{10}y^3+2z^2w^7x^{10}y^3+z^4w^4x^{10}y^3+z^2w^6x^{10}y^3$\\
$\mathcal{M}_{3,3}(30,32)=z^4w^6x^{14}y^3+z^5w^5x^{12}y^3+z^5w^6x^{14}y^3+z^4w^4x^{10}y^3$\\
$\mathcal{M}_{3,3}(30,33)=2z^2w^5x^{10}y^3+z^2w^5x^{12}y^3+z^3w^7x^{14}y^3+2z^3w^6x^{12}y^3+z^2w^6x^{10}y^3+z^3w^7x^{12}y^3$\\
$\mathcal{M}_{3,3}(30,34)=z^2w^5x^{12}y^3+z^3w^7x^{14}y^3+z^3w^6x^{12}y^3+z^2w^7x^{12}y^3$\\
$\mathcal{M}_{3,3}(30,35)=z^2w^5x^{10}y^3+2zw^8x^{10}y^3+w^6x^{10}y^3+zw^6x^{10}y^3$\\
$\mathcal{M}_{3,3}(30,36)=0$\\
$\mathcal{M}_{3,3}(30,37)=2z^2w^4x^{10}y^3+2z^3w^6x^{12}y^3+2zw^5x^{10}y^3+3z^2w^7x^{10}y^3+2z^4w^4x^{10}y^3$\\
$\mathcal{M}_{3,3}(31,1)=0$\\
$\mathcal{M}_{3,3}(31,2)=2z^3w^5x^{10}y^3+z^2w^3x^6y^3+z^3w^6x^{12}y^3+2z^3w^6x^{10}y^3+z^2w^7x^{10}y^3+2z^3w^4x^8y^3+z^2w^5x^8y^3$\\
$\mathcal{M}_{3,3}(31,3)=0$\\
$\mathcal{M}_{3,3}(31,4)=3z^3w^5x^{10}y^3+2z^2w^4x^8y^3+2z^3w^6x^{10}y^3+z^2w^7x^{10}y^3+z^2w^6x^{10}y^3+z^2w^5x^8y^3$\\
$\mathcal{M}_{3,3}(31,5)=0$\\
$\mathcal{M}_{3,3}(31,6)=3z^2w^5x^{10}y^3+z^3w^6x^{12}y^3+z^2w^7x^{10}y^3+2z^2w^6x^{10}y^3$\\
$\mathcal{M}_{3,3}(31,7)=0$\\
$\mathcal{M}_{3,3}(31,8)=z^3w^5x^{10}y^3+z^3w^7x^{14}y^3+z^3w^6x^{12}y^3+z^2w^4x^8y^3$\\
$\mathcal{M}_{3,3}(31,9)=z^2w^5x^{10}y^3+2z^3w^6x^{12}y^3+z^2w^7x^{12}y^3$\\
$\mathcal{M}_{3,3}(31,10)=zw^8x^{10}y^3+zw^5x^8y^3+zw^6x^8y^3+2z^2w^7x^{10}y^3+z^2w^6x^{10}y^3$\\
$\mathcal{M}_{3,3}(31,11)=0$\\
$\mathcal{M}_{3,3}(31,12)=z^4w^6x^{14}y^3+z^3w^7x^{14}y^3+z^3w^6x^{12}y^3+z^5w^6x^{14}y^3+z^4w^5x^{12}y^3+z^3w^8x^{12}y^3+z^3w^7x^{12}y^3$\\
$\mathcal{M}_{3,3}(31,13)=2z^3w^5x^{12}y^3+z^2w^4x^{10}y^3+z^3w^7x^{14}y^3+z^2w^6x^{12}y^3+z^3w^6x^{14}y^3$\\
$\mathcal{M}_{3,3}(31,14)=2z^2w^6x^{14}y^3+z^2w^7x^{14}y^3$\\
$\mathcal{M}_{3,3}(31,15)=z^3w^7x^{14}y^3+z^3w^6x^{12}y^3+z^3w^8x^{12}y^3+z^3w^7x^{12}y^3$\\
$\mathcal{M}_{3,3}(31,16)=0$\\
$\mathcal{M}_{3,3}(31,17)=2z^2w^5x^{10}y^3+z^3w^6x^{12}y^3+z^2w^6x^{10}y^3+z^3w^7x^{12}y^3$\\
$\mathcal{M}_{3,3}(31,18)=z^3w^7x^{14}y^3+z^2w^6x^{12}y^3$\\
$\mathcal{M}_{3,3}(31,19)=zw^8x^{10}y^3+2zw^7x^{10}y^3+2zw^6x^{10}y^3$\\
$\mathcal{M}_{3,3}(31,20)=0$\\
$\mathcal{M}_{3,3}(31,21)=z^2w^5x^{10}y^3+z^3w^7x^{14}y^3+z^3w^6x^{12}y^3$\\
$\mathcal{M}_{3,3}(31,22)=z^2w^7x^{14}y^3$\\
$\mathcal{M}_{3,3}(31,23)=zw^8x^{14}y^3$\\
$\mathcal{M}_{3,3}(31,24)=z^4w^7x^{14}y^3+z^3w^6x^{12}y^3$\\
$\mathcal{M}_{3,3}(31,25)=z^2w^7x^{10}y^3+z^2w^6x^{10}y^3+z^3w^8x^{12}y^3$\\
$\mathcal{M}_{3,3}(31,26)=z^2w^7x^{12}y^3+z^2w^8x^{12}y^3$\\
$\mathcal{M}_{3,3}(31,27)=zw^8x^{10}y^3+zw^7x^{10}y^3+zw^9x^{10}y^3$\\
$\mathcal{M}_{3,3}(31,28)=0$\\
$\mathcal{M}_{3,3}(31,29)=z^2w^4x^{10}y^3+5z^3w^5x^{10}y^3+z^3w^6x^{12}y^3+zw^5x^{10}y^3+2z^2w^4x^8y^3+3z^3w^6x^{10}y^3+2z^2w^7x^{10}y^3+z^4w^4x^{10}y^3+z^2w^5x^8y^3$\\
$\mathcal{M}_{3,3}(31,30)=0$\\
$\mathcal{M}_{3,3}(31,31)=z^2w^5x^{10}y^3+z^2w^4x^{10}y^3+z^3w^5x^{10}y^3+z^3w^6x^{12}y^3+zw^5x^{10}y^3+2z^2w^7x^{10}y^3+z^4w^4x^{10}y^3+z^2w^6x^{10}y^3$\\
$\mathcal{M}_{3,3}(31,32)=z^4w^6x^{14}y^3+z^5w^5x^{12}y^3+z^5w^6x^{14}y^3+z^4w^4x^{10}y^3$\\
$\mathcal{M}_{3,3}(31,33)=2z^2w^5x^{10}y^3+z^2w^5x^{12}y^3+z^3w^7x^{14}y^3+2z^3w^6x^{12}y^3+z^2w^6x^{10}y^3+z^3w^7x^{12}y^3$\\
$\mathcal{M}_{3,3}(31,34)=z^2w^5x^{12}y^3+z^3w^7x^{14}y^3+z^3w^6x^{12}y^3+z^2w^7x^{12}y^3$\\
$\mathcal{M}_{3,3}(31,35)=z^2w^5x^{10}y^3+2zw^8x^{10}y^3+w^6x^{10}y^3+zw^6x^{10}y^3$\\
$\mathcal{M}_{3,3}(31,36)=0$\\
$\mathcal{M}_{3,3}(31,37)=2z^2w^4x^{10}y^3+2z^3w^6x^{12}y^3+2zw^5x^{10}y^3+3z^2w^7x^{10}y^3+2z^4w^4x^{10}y^3$\\
$\mathcal{M}_{3,3}(32,1)=0$\\
$\mathcal{M}_{3,3}(32,2)=2z^3w^5x^{10}y^3+z^2w^3x^6y^3+z^3w^6x^{12}y^3+2z^3w^6x^{10}y^3+z^2w^7x^{10}y^3+2z^3w^4x^8y^3+z^2w^5x^8y^3$\\
$\mathcal{M}_{3,3}(32,3)=0$\\
$\mathcal{M}_{3,3}(32,4)=3z^3w^5x^{10}y^3+2z^2w^4x^8y^3+2z^3w^6x^{10}y^3+z^2w^7x^{10}y^3+z^2w^6x^{10}y^3+z^2w^5x^8y^3$\\
$\mathcal{M}_{3,3}(32,5)=0$\\
$\mathcal{M}_{3,3}(32,6)=3z^2w^5x^{10}y^3+z^3w^6x^{12}y^3+z^2w^7x^{10}y^3+2z^2w^6x^{10}y^3$\\
$\mathcal{M}_{3,3}(32,7)=0$\\
$\mathcal{M}_{3,3}(32,8)=z^3w^5x^{10}y^3+z^3w^7x^{14}y^3+z^3w^6x^{12}y^3+z^2w^4x^8y^3$\\
$\mathcal{M}_{3,3}(32,9)=z^2w^5x^{10}y^3+2z^3w^6x^{12}y^3+z^2w^7x^{12}y^3$\\
$\mathcal{M}_{3,3}(32,10)=zw^8x^{10}y^3+zw^5x^8y^3+zw^6x^8y^3+2z^2w^7x^{10}y^3+z^2w^6x^{10}y^3$\\
$\mathcal{M}_{3,3}(32,11)=0$\\
$\mathcal{M}_{3,3}(32,12)=z^4w^6x^{14}y^3+z^3w^7x^{14}y^3+z^3w^6x^{12}y^3+z^5w^6x^{14}y^3+z^4w^5x^{12}y^3+z^3w^8x^{12}y^3+z^3w^7x^{12}y^3$\\
$\mathcal{M}_{3,3}(32,13)=2z^3w^5x^{12}y^3+z^2w^4x^{10}y^3+z^3w^7x^{14}y^3+z^2w^6x^{12}y^3+z^3w^6x^{14}y^3$\\
$\mathcal{M}_{3,3}(32,14)=2z^2w^6x^{14}y^3+z^2w^7x^{14}y^3$\\
$\mathcal{M}_{3,3}(32,15)=z^3w^7x^{14}y^3+z^3w^6x^{12}y^3+z^3w^8x^{12}y^3+z^3w^7x^{12}y^3$\\
$\mathcal{M}_{3,3}(32,16)=0$\\
$\mathcal{M}_{3,3}(32,17)=2z^2w^5x^{10}y^3+z^3w^6x^{12}y^3+z^2w^6x^{10}y^3+z^3w^7x^{12}y^3$\\
$\mathcal{M}_{3,3}(32,18)=z^3w^7x^{14}y^3+z^2w^6x^{12}y^3$\\
$\mathcal{M}_{3,3}(32,19)=zw^8x^{10}y^3+2zw^7x^{10}y^3+2zw^6x^{10}y^3$\\
$\mathcal{M}_{3,3}(32,20)=0$\\
$\mathcal{M}_{3,3}(32,21)=z^2w^5x^{10}y^3+z^3w^7x^{14}y^3+z^3w^6x^{12}y^3$\\
$\mathcal{M}_{3,3}(32,22)=z^2w^7x^{14}y^3$\\
$\mathcal{M}_{3,3}(32,23)=zw^8x^{14}y^3$\\
$\mathcal{M}_{3,3}(32,24)=z^4w^7x^{14}y^3+z^3w^6x^{12}y^3$\\
$\mathcal{M}_{3,3}(32,25)=z^2w^7x^{10}y^3+z^2w^6x^{10}y^3+z^3w^8x^{12}y^3$\\
$\mathcal{M}_{3,3}(32,26)=z^2w^7x^{12}y^3+z^2w^8x^{12}y^3$\\
$\mathcal{M}_{3,3}(32,27)=zw^8x^{10}y^3+zw^7x^{10}y^3+zw^9x^{10}y^3$\\
$\mathcal{M}_{3,3}(32,28)=0$\\
$\mathcal{M}_{3,3}(32,29)=z^2w^4x^{10}y^3+5z^3w^5x^{10}y^3+z^3w^6x^{12}y^3+zw^5x^{10}y^3+2z^2w^4x^8y^3+3z^3w^6x^{10}y^3+2z^2w^7x^{10}y^3+z^4w^4x^{10}y^3+z^2w^5x^8y^3$\\
$\mathcal{M}_{3,3}(32,30)=0$\\
$\mathcal{M}_{3,3}(32,31)=z^2w^5x^{10}y^3+z^2w^4x^{10}y^3+z^3w^5x^{10}y^3+z^3w^6x^{12}y^3+zw^5x^{10}y^3+2z^2w^7x^{10}y^3+z^4w^4x^{10}y^3+z^2w^6x^{10}y^3$\\
$\mathcal{M}_{3,3}(32,32)=z^4w^6x^{14}y^3+z^5w^5x^{12}y^3+z^5w^6x^{14}y^3+z^4w^4x^{10}y^3$\\
$\mathcal{M}_{3,3}(32,33)=2z^2w^5x^{10}y^3+z^2w^5x^{12}y^3+z^3w^7x^{14}y^3+2z^3w^6x^{12}y^3+z^2w^6x^{10}y^3+z^3w^7x^{12}y^3$\\
$\mathcal{M}_{3,3}(32,34)=z^2w^5x^{12}y^3+z^3w^7x^{14}y^3+z^3w^6x^{12}y^3+z^2w^7x^{12}y^3$\\
$\mathcal{M}_{3,3}(32,35)=z^2w^5x^{10}y^3+2zw^8x^{10}y^3+w^6x^{10}y^3+zw^6x^{10}y^3$\\
$\mathcal{M}_{3,3}(32,36)=0$\\
$\mathcal{M}_{3,3}(32,37)=2z^2w^4x^{10}y^3+2z^3w^6x^{12}y^3+2zw^5x^{10}y^3+3z^2w^7x^{10}y^3+2z^4w^4x^{10}y^3$\\
$\mathcal{M}_{3,3}(33,1)=0$\\
$\mathcal{M}_{3,3}(33,2)=2z^3w^5x^{10}y^3+z^2w^3x^6y^3+z^3w^6x^{12}y^3+2z^3w^6x^{10}y^3+z^2w^7x^{10}y^3+2z^3w^4x^8y^3+z^2w^5x^8y^3$\\
$\mathcal{M}_{3,3}(33,3)=0$\\
$\mathcal{M}_{3,3}(33,4)=3z^3w^5x^{10}y^3+2z^2w^4x^8y^3+2z^3w^6x^{10}y^3+z^2w^7x^{10}y^3+z^2w^6x^{10}y^3+z^2w^5x^8y^3$\\
$\mathcal{M}_{3,3}(33,5)=0$\\
$\mathcal{M}_{3,3}(33,6)=3z^2w^5x^{10}y^3+z^3w^6x^{12}y^3+z^2w^7x^{10}y^3+2z^2w^6x^{10}y^3$\\
$\mathcal{M}_{3,3}(33,7)=0$\\
$\mathcal{M}_{3,3}(33,8)=z^3w^5x^{10}y^3+z^3w^7x^{14}y^3+z^3w^6x^{12}y^3+z^2w^4x^8y^3$\\
$\mathcal{M}_{3,3}(33,9)=z^2w^5x^{10}y^3+2z^3w^6x^{12}y^3+z^2w^7x^{12}y^3$\\
$\mathcal{M}_{3,3}(33,10)=zw^8x^{10}y^3+zw^5x^8y^3+zw^6x^8y^3+2z^2w^7x^{10}y^3+z^2w^6x^{10}y^3$\\
$\mathcal{M}_{3,3}(33,11)=0$\\
$\mathcal{M}_{3,3}(33,12)=z^4w^6x^{14}y^3+z^3w^7x^{14}y^3+z^3w^6x^{12}y^3+z^5w^6x^{14}y^3+z^4w^5x^{12}y^3+z^3w^8x^{12}y^3+z^3w^7x^{12}y^3$\\
$\mathcal{M}_{3,3}(33,13)=2z^3w^5x^{12}y^3+z^2w^4x^{10}y^3+z^3w^7x^{14}y^3+z^2w^6x^{12}y^3+z^3w^6x^{14}y^3$\\
$\mathcal{M}_{3,3}(33,14)=2z^2w^6x^{14}y^3+z^2w^7x^{14}y^3$\\
$\mathcal{M}_{3,3}(33,15)=z^3w^7x^{14}y^3+z^3w^6x^{12}y^3+z^3w^8x^{12}y^3+z^3w^7x^{12}y^3$\\
$\mathcal{M}_{3,3}(33,16)=0$\\
$\mathcal{M}_{3,3}(33,17)=2z^2w^5x^{10}y^3+z^3w^6x^{12}y^3+z^2w^6x^{10}y^3+z^3w^7x^{12}y^3$\\
$\mathcal{M}_{3,3}(33,18)=z^3w^7x^{14}y^3+z^2w^6x^{12}y^3$\\
$\mathcal{M}_{3,3}(33,19)=zw^8x^{10}y^3+2zw^7x^{10}y^3+2zw^6x^{10}y^3$\\
$\mathcal{M}_{3,3}(33,20)=0$\\
$\mathcal{M}_{3,3}(33,21)=z^2w^5x^{10}y^3+z^3w^7x^{14}y^3+z^3w^6x^{12}y^3$\\
$\mathcal{M}_{3,3}(33,22)=z^2w^7x^{14}y^3$\\
$\mathcal{M}_{3,3}(33,23)=zw^8x^{14}y^3$\\
$\mathcal{M}_{3,3}(33,24)=z^4w^7x^{14}y^3+z^3w^6x^{12}y^3$\\
$\mathcal{M}_{3,3}(33,25)=z^2w^7x^{10}y^3+z^2w^6x^{10}y^3+z^3w^8x^{12}y^3$\\
$\mathcal{M}_{3,3}(33,26)=z^2w^7x^{12}y^3+z^2w^8x^{12}y^3$\\
$\mathcal{M}_{3,3}(33,27)=zw^8x^{10}y^3+zw^7x^{10}y^3+zw^9x^{10}y^3$\\
$\mathcal{M}_{3,3}(33,28)=0$\\
$\mathcal{M}_{3,3}(33,29)=z^2w^4x^{10}y^3+5z^3w^5x^{10}y^3+z^3w^6x^{12}y^3+zw^5x^{10}y^3+2z^2w^4x^8y^3+3z^3w^6x^{10}y^3+2z^2w^7x^{10}y^3+z^4w^4x^{10}y^3+z^2w^5x^8y^3$\\
$\mathcal{M}_{3,3}(33,30)=0$\\
$\mathcal{M}_{3,3}(33,31)=z^2w^5x^{10}y^3+z^2w^4x^{10}y^3+z^3w^5x^{10}y^3+z^3w^6x^{12}y^3+zw^5x^{10}y^3+2z^2w^7x^{10}y^3+z^4w^4x^{10}y^3+z^2w^6x^{10}y^3$\\
$\mathcal{M}_{3,3}(33,32)=z^4w^6x^{14}y^3+z^5w^5x^{12}y^3+z^5w^6x^{14}y^3+z^4w^4x^{10}y^3$\\
$\mathcal{M}_{3,3}(33,33)=2z^2w^5x^{10}y^3+z^2w^5x^{12}y^3+z^3w^7x^{14}y^3+2z^3w^6x^{12}y^3+z^2w^6x^{10}y^3+z^3w^7x^{12}y^3$\\
$\mathcal{M}_{3,3}(33,34)=z^2w^5x^{12}y^3+z^3w^7x^{14}y^3+z^3w^6x^{12}y^3+z^2w^7x^{12}y^3$\\
$\mathcal{M}_{3,3}(33,35)=z^2w^5x^{10}y^3+2zw^8x^{10}y^3+w^6x^{10}y^3+zw^6x^{10}y^3$\\
$\mathcal{M}_{3,3}(33,36)=0$\\
$\mathcal{M}_{3,3}(33,37)=2z^2w^4x^{10}y^3+2z^3w^6x^{12}y^3+2zw^5x^{10}y^3+3z^2w^7x^{10}y^3+2z^4w^4x^{10}y^3$\\
$\mathcal{M}_{3,3}(34,1)=0$\\
$\mathcal{M}_{3,3}(34,2)=2z^3w^5x^{10}y^3+z^2w^3x^6y^3+z^3w^6x^{12}y^3+2z^3w^6x^{10}y^3+z^2w^7x^{10}y^3+2z^3w^4x^8y^3+z^2w^5x^8y^3$\\
$\mathcal{M}_{3,3}(34,3)=0$\\
$\mathcal{M}_{3,3}(34,4)=3z^3w^5x^{10}y^3+2z^2w^4x^8y^3+2z^3w^6x^{10}y^3+z^2w^7x^{10}y^3+z^2w^6x^{10}y^3+z^2w^5x^8y^3$\\
$\mathcal{M}_{3,3}(34,5)=0$\\
$\mathcal{M}_{3,3}(34,6)=3z^2w^5x^{10}y^3+z^3w^6x^{12}y^3+z^2w^7x^{10}y^3+2z^2w^6x^{10}y^3$\\
$\mathcal{M}_{3,3}(34,7)=0$\\
$\mathcal{M}_{3,3}(34,8)=z^3w^5x^{10}y^3+z^3w^7x^{14}y^3+z^3w^6x^{12}y^3+z^2w^4x^8y^3$\\
$\mathcal{M}_{3,3}(34,9)=z^2w^5x^{10}y^3+2z^3w^6x^{12}y^3+z^2w^7x^{12}y^3$\\
$\mathcal{M}_{3,3}(34,10)=zw^8x^{10}y^3+zw^5x^8y^3+zw^6x^8y^3+2z^2w^7x^{10}y^3+z^2w^6x^{10}y^3$\\
$\mathcal{M}_{3,3}(34,11)=0$\\
$\mathcal{M}_{3,3}(34,12)=z^4w^6x^{14}y^3+z^3w^7x^{14}y^3+z^3w^6x^{12}y^3+z^5w^6x^{14}y^3+z^4w^5x^{12}y^3+z^3w^8x^{12}y^3+z^3w^7x^{12}y^3$\\
$\mathcal{M}_{3,3}(34,13)=2z^3w^5x^{12}y^3+z^2w^4x^{10}y^3+z^3w^7x^{14}y^3+z^2w^6x^{12}y^3+z^3w^6x^{14}y^3$\\
$\mathcal{M}_{3,3}(34,14)=2z^2w^6x^{14}y^3+z^2w^7x^{14}y^3$\\
$\mathcal{M}_{3,3}(34,15)=z^3w^7x^{14}y^3+z^3w^6x^{12}y^3+z^3w^8x^{12}y^3+z^3w^7x^{12}y^3$\\
$\mathcal{M}_{3,3}(34,16)=0$\\
$\mathcal{M}_{3,3}(34,17)=2z^2w^5x^{10}y^3+z^3w^6x^{12}y^3+z^2w^6x^{10}y^3+z^3w^7x^{12}y^3$\\
$\mathcal{M}_{3,3}(34,18)=z^3w^7x^{14}y^3+z^2w^6x^{12}y^3$\\
$\mathcal{M}_{3,3}(34,19)=zw^8x^{10}y^3+2zw^7x^{10}y^3+2zw^6x^{10}y^3$\\
$\mathcal{M}_{3,3}(34,20)=0$\\
$\mathcal{M}_{3,3}(34,21)=z^2w^5x^{10}y^3+z^3w^7x^{14}y^3+z^3w^6x^{12}y^3$\\
$\mathcal{M}_{3,3}(34,22)=z^2w^7x^{14}y^3$\\
$\mathcal{M}_{3,3}(34,23)=zw^8x^{14}y^3$\\
$\mathcal{M}_{3,3}(34,24)=z^4w^7x^{14}y^3+z^3w^6x^{12}y^3$\\
$\mathcal{M}_{3,3}(34,25)=z^2w^7x^{10}y^3+z^2w^6x^{10}y^3+z^3w^8x^{12}y^3$\\
$\mathcal{M}_{3,3}(34,26)=z^2w^7x^{12}y^3+z^2w^8x^{12}y^3$\\
$\mathcal{M}_{3,3}(34,27)=zw^8x^{10}y^3+zw^7x^{10}y^3+zw^9x^{10}y^3$\\
$\mathcal{M}_{3,3}(34,28)=0$\\
$\mathcal{M}_{3,3}(34,29)=z^2w^4x^{10}y^3+5z^3w^5x^{10}y^3+z^3w^6x^{12}y^3+zw^5x^{10}y^3+2z^2w^4x^8y^3+3z^3w^6x^{10}y^3+2z^2w^7x^{10}y^3+z^4w^4x^{10}y^3+z^2w^5x^8y^3$\\
$\mathcal{M}_{3,3}(34,30)=0$\\
$\mathcal{M}_{3,3}(34,31)=z^2w^5x^{10}y^3+z^2w^4x^{10}y^3+z^3w^5x^{10}y^3+z^3w^6x^{12}y^3+zw^5x^{10}y^3+2z^2w^7x^{10}y^3+z^4w^4x^{10}y^3+z^2w^6x^{10}y^3$\\
$\mathcal{M}_{3,3}(34,32)=z^4w^6x^{14}y^3+z^5w^5x^{12}y^3+z^5w^6x^{14}y^3+z^4w^4x^{10}y^3$\\
$\mathcal{M}_{3,3}(34,33)=2z^2w^5x^{10}y^3+z^2w^5x^{12}y^3+z^3w^7x^{14}y^3+2z^3w^6x^{12}y^3+z^2w^6x^{10}y^3+z^3w^7x^{12}y^3$\\
$\mathcal{M}_{3,3}(34,34)=z^2w^5x^{12}y^3+z^3w^7x^{14}y^3+z^3w^6x^{12}y^3+z^2w^7x^{12}y^3$\\
$\mathcal{M}_{3,3}(34,35)=z^2w^5x^{10}y^3+2zw^8x^{10}y^3+w^6x^{10}y^3+zw^6x^{10}y^3$\\
$\mathcal{M}_{3,3}(34,36)=0$\\
$\mathcal{M}_{3,3}(34,37)=2z^2w^4x^{10}y^3+2z^3w^6x^{12}y^3+2zw^5x^{10}y^3+3z^2w^7x^{10}y^3+2z^4w^4x^{10}y^3$\\
$\mathcal{M}_{3,3}(35,1)=0$\\
$\mathcal{M}_{3,3}(35,2)=2z^3w^5x^{10}y^3+z^2w^3x^6y^3+z^3w^6x^{12}y^3+2z^3w^6x^{10}y^3+z^2w^7x^{10}y^3+2z^3w^4x^8y^3+z^2w^5x^8y^3$\\
$\mathcal{M}_{3,3}(35,3)=0$\\
$\mathcal{M}_{3,3}(35,4)=3z^3w^5x^{10}y^3+2z^2w^4x^8y^3+2z^3w^6x^{10}y^3+z^2w^7x^{10}y^3+z^2w^6x^{10}y^3+z^2w^5x^8y^3$\\
$\mathcal{M}_{3,3}(35,5)=0$\\
$\mathcal{M}_{3,3}(35,6)=3z^2w^5x^{10}y^3+z^3w^6x^{12}y^3+z^2w^7x^{10}y^3+2z^2w^6x^{10}y^3$\\
$\mathcal{M}_{3,3}(35,7)=0$\\
$\mathcal{M}_{3,3}(35,8)=z^3w^5x^{10}y^3+z^3w^7x^{14}y^3+z^3w^6x^{12}y^3+z^2w^4x^8y^3$\\
$\mathcal{M}_{3,3}(35,9)=z^2w^5x^{10}y^3+2z^3w^6x^{12}y^3+z^2w^7x^{12}y^3$\\
$\mathcal{M}_{3,3}(35,10)=zw^8x^{10}y^3+zw^5x^8y^3+zw^6x^8y^3+2z^2w^7x^{10}y^3+z^2w^6x^{10}y^3$\\
$\mathcal{M}_{3,3}(35,11)=0$\\
$\mathcal{M}_{3,3}(35,12)=z^4w^6x^{14}y^3+z^3w^7x^{14}y^3+z^3w^6x^{12}y^3+z^5w^6x^{14}y^3+z^4w^5x^{12}y^3+z^3w^8x^{12}y^3+z^3w^7x^{12}y^3$\\
$\mathcal{M}_{3,3}(35,13)=2z^3w^5x^{12}y^3+z^2w^4x^{10}y^3+z^3w^7x^{14}y^3+z^2w^6x^{12}y^3+z^3w^6x^{14}y^3$\\
$\mathcal{M}_{3,3}(35,14)=2z^2w^6x^{14}y^3+z^2w^7x^{14}y^3$\\
$\mathcal{M}_{3,3}(35,15)=z^3w^7x^{14}y^3+z^3w^6x^{12}y^3+z^3w^8x^{12}y^3+z^3w^7x^{12}y^3$\\
$\mathcal{M}_{3,3}(35,16)=0$\\
$\mathcal{M}_{3,3}(35,17)=2z^2w^5x^{10}y^3+z^3w^6x^{12}y^3+z^2w^6x^{10}y^3+z^3w^7x^{12}y^3$\\
$\mathcal{M}_{3,3}(35,18)=z^3w^7x^{14}y^3+z^2w^6x^{12}y^3$\\
$\mathcal{M}_{3,3}(35,19)=zw^8x^{10}y^3+2zw^7x^{10}y^3+2zw^6x^{10}y^3$\\
$\mathcal{M}_{3,3}(35,20)=0$\\
$\mathcal{M}_{3,3}(35,21)=z^2w^5x^{10}y^3+z^3w^7x^{14}y^3+z^3w^6x^{12}y^3$\\
$\mathcal{M}_{3,3}(35,22)=z^2w^7x^{14}y^3$\\
$\mathcal{M}_{3,3}(35,23)=zw^8x^{14}y^3$\\
$\mathcal{M}_{3,3}(35,24)=z^4w^7x^{14}y^3+z^3w^6x^{12}y^3$\\
$\mathcal{M}_{3,3}(35,25)=z^2w^7x^{10}y^3+z^2w^6x^{10}y^3+z^3w^8x^{12}y^3$\\
$\mathcal{M}_{3,3}(35,26)=z^2w^7x^{12}y^3+z^2w^8x^{12}y^3$\\
$\mathcal{M}_{3,3}(35,27)=zw^8x^{10}y^3+zw^7x^{10}y^3+zw^9x^{10}y^3$\\
$\mathcal{M}_{3,3}(35,28)=0$\\
$\mathcal{M}_{3,3}(35,29)=z^2w^4x^{10}y^3+5z^3w^5x^{10}y^3+z^3w^6x^{12}y^3+zw^5x^{10}y^3+2z^2w^4x^8y^3+3z^3w^6x^{10}y^3+2z^2w^7x^{10}y^3+z^4w^4x^{10}y^3+z^2w^5x^8y^3$\\
$\mathcal{M}_{3,3}(35,30)=0$\\
$\mathcal{M}_{3,3}(35,31)=z^2w^5x^{10}y^3+z^2w^4x^{10}y^3+z^3w^5x^{10}y^3+z^3w^6x^{12}y^3+zw^5x^{10}y^3+2z^2w^7x^{10}y^3+z^4w^4x^{10}y^3+z^2w^6x^{10}y^3$\\
$\mathcal{M}_{3,3}(35,32)=z^4w^6x^{14}y^3+z^5w^5x^{12}y^3+z^5w^6x^{14}y^3+z^4w^4x^{10}y^3$\\
$\mathcal{M}_{3,3}(35,33)=2z^2w^5x^{10}y^3+z^2w^5x^{12}y^3+z^3w^7x^{14}y^3+2z^3w^6x^{12}y^3+z^2w^6x^{10}y^3+z^3w^7x^{12}y^3$\\
$\mathcal{M}_{3,3}(35,34)=z^2w^5x^{12}y^3+z^3w^7x^{14}y^3+z^3w^6x^{12}y^3+z^2w^7x^{12}y^3$\\
$\mathcal{M}_{3,3}(35,35)=z^2w^5x^{10}y^3+2zw^8x^{10}y^3+w^6x^{10}y^3+zw^6x^{10}y^3$\\
$\mathcal{M}_{3,3}(35,36)=0$\\
$\mathcal{M}_{3,3}(35,37)=2z^2w^4x^{10}y^3+2z^3w^6x^{12}y^3+2zw^5x^{10}y^3+3z^2w^7x^{10}y^3+2z^4w^4x^{10}y^3$\\
$\mathcal{M}_{3,3}(36,1)=0$\\
$\mathcal{M}_{3,3}(36,2)=2z^3w^5x^{10}y^3+z^2w^3x^6y^3+z^3w^6x^{12}y^3+2z^3w^6x^{10}y^3+z^2w^7x^{10}y^3+2z^3w^4x^8y^3+z^2w^5x^8y^3$\\
$\mathcal{M}_{3,3}(36,3)=0$\\
$\mathcal{M}_{3,3}(36,4)=3z^3w^5x^{10}y^3+2z^2w^4x^8y^3+2z^3w^6x^{10}y^3+z^2w^7x^{10}y^3+z^2w^6x^{10}y^3+z^2w^5x^8y^3$\\
$\mathcal{M}_{3,3}(36,5)=0$\\
$\mathcal{M}_{3,3}(36,6)=3z^2w^5x^{10}y^3+z^3w^6x^{12}y^3+z^2w^7x^{10}y^3+2z^2w^6x^{10}y^3$\\
$\mathcal{M}_{3,3}(36,7)=0$\\
$\mathcal{M}_{3,3}(36,8)=z^3w^5x^{10}y^3+z^3w^7x^{14}y^3+z^3w^6x^{12}y^3+z^2w^4x^8y^3$\\
$\mathcal{M}_{3,3}(36,9)=z^2w^5x^{10}y^3+2z^3w^6x^{12}y^3+z^2w^7x^{12}y^3$\\
$\mathcal{M}_{3,3}(36,10)=zw^8x^{10}y^3+zw^5x^8y^3+zw^6x^8y^3+2z^2w^7x^{10}y^3+z^2w^6x^{10}y^3$\\
$\mathcal{M}_{3,3}(36,11)=0$\\
$\mathcal{M}_{3,3}(36,12)=z^4w^6x^{14}y^3+z^3w^7x^{14}y^3+z^3w^6x^{12}y^3+z^5w^6x^{14}y^3+z^4w^5x^{12}y^3+z^3w^8x^{12}y^3+z^3w^7x^{12}y^3$\\
$\mathcal{M}_{3,3}(36,13)=2z^3w^5x^{12}y^3+z^2w^4x^{10}y^3+z^3w^7x^{14}y^3+z^2w^6x^{12}y^3+z^3w^6x^{14}y^3$\\
$\mathcal{M}_{3,3}(36,14)=2z^2w^6x^{14}y^3+z^2w^7x^{14}y^3$\\
$\mathcal{M}_{3,3}(36,15)=z^3w^7x^{14}y^3+z^3w^6x^{12}y^3+z^3w^8x^{12}y^3+z^3w^7x^{12}y^3$\\
$\mathcal{M}_{3,3}(36,16)=0$\\
$\mathcal{M}_{3,3}(36,17)=2z^2w^5x^{10}y^3+z^3w^6x^{12}y^3+z^2w^6x^{10}y^3+z^3w^7x^{12}y^3$\\
$\mathcal{M}_{3,3}(36,18)=z^3w^7x^{14}y^3+z^2w^6x^{12}y^3$\\
$\mathcal{M}_{3,3}(36,19)=zw^8x^{10}y^3+2zw^7x^{10}y^3+2zw^6x^{10}y^3$\\
$\mathcal{M}_{3,3}(36,20)=0$\\
$\mathcal{M}_{3,3}(36,21)=z^2w^5x^{10}y^3+z^3w^7x^{14}y^3+z^3w^6x^{12}y^3$\\
$\mathcal{M}_{3,3}(36,22)=z^2w^7x^{14}y^3$\\
$\mathcal{M}_{3,3}(36,23)=zw^8x^{14}y^3$\\
$\mathcal{M}_{3,3}(36,24)=z^4w^7x^{14}y^3+z^3w^6x^{12}y^3$\\
$\mathcal{M}_{3,3}(36,25)=z^2w^7x^{10}y^3+z^2w^6x^{10}y^3+z^3w^8x^{12}y^3$\\
$\mathcal{M}_{3,3}(36,26)=z^2w^7x^{12}y^3+z^2w^8x^{12}y^3$\\
$\mathcal{M}_{3,3}(36,27)=zw^8x^{10}y^3+zw^7x^{10}y^3+zw^9x^{10}y^3$\\
$\mathcal{M}_{3,3}(36,28)=0$\\
$\mathcal{M}_{3,3}(36,29)=z^2w^4x^{10}y^3+5z^3w^5x^{10}y^3+z^3w^6x^{12}y^3+zw^5x^{10}y^3+2z^2w^4x^8y^3+3z^3w^6x^{10}y^3+2z^2w^7x^{10}y^3+z^4w^4x^{10}y^3+z^2w^5x^8y^3$\\
$\mathcal{M}_{3,3}(36,30)=0$\\
$\mathcal{M}_{3,3}(36,31)=z^2w^5x^{10}y^3+z^2w^4x^{10}y^3+z^3w^5x^{10}y^3+z^3w^6x^{12}y^3+zw^5x^{10}y^3+2z^2w^7x^{10}y^3+z^4w^4x^{10}y^3+z^2w^6x^{10}y^3$\\
$\mathcal{M}_{3,3}(36,32)=z^4w^6x^{14}y^3+z^5w^5x^{12}y^3+z^5w^6x^{14}y^3+z^4w^4x^{10}y^3$\\
$\mathcal{M}_{3,3}(36,33)=2z^2w^5x^{10}y^3+z^2w^5x^{12}y^3+z^3w^7x^{14}y^3+2z^3w^6x^{12}y^3+z^2w^6x^{10}y^3+z^3w^7x^{12}y^3$\\
$\mathcal{M}_{3,3}(36,34)=z^2w^5x^{12}y^3+z^3w^7x^{14}y^3+z^3w^6x^{12}y^3+z^2w^7x^{12}y^3$\\
$\mathcal{M}_{3,3}(36,35)=z^2w^5x^{10}y^3+2zw^8x^{10}y^3+w^6x^{10}y^3+zw^6x^{10}y^3$\\
$\mathcal{M}_{3,3}(36,36)=0$\\
$\mathcal{M}_{3,3}(36,37)=2z^2w^4x^{10}y^3+2z^3w^6x^{12}y^3+2zw^5x^{10}y^3+3z^2w^7x^{10}y^3+2z^4w^4x^{10}y^3$\\
$\mathcal{M}_{3,3}(37,1)=0$\\
$\mathcal{M}_{3,3}(37,2)=2z^3w^5x^{10}y^3+z^2w^3x^6y^3+z^3w^6x^{12}y^3+2z^3w^6x^{10}y^3+z^2w^7x^{10}y^3+2z^3w^4x^8y^3+z^2w^5x^8y^3$\\
$\mathcal{M}_{3,3}(37,3)=0$\\
$\mathcal{M}_{3,3}(37,4)=3z^3w^5x^{10}y^3+2z^2w^4x^8y^3+2z^3w^6x^{10}y^3+z^2w^7x^{10}y^3+z^2w^6x^{10}y^3+z^2w^5x^8y^3$\\
$\mathcal{M}_{3,3}(37,5)=0$\\
$\mathcal{M}_{3,3}(37,6)=3z^2w^5x^{10}y^3+z^3w^6x^{12}y^3+z^2w^7x^{10}y^3+2z^2w^6x^{10}y^3$\\
$\mathcal{M}_{3,3}(37,7)=0$\\
$\mathcal{M}_{3,3}(37,8)=z^3w^5x^{10}y^3+z^3w^7x^{14}y^3+z^3w^6x^{12}y^3+z^2w^4x^8y^3$\\
$\mathcal{M}_{3,3}(37,9)=z^2w^5x^{10}y^3+2z^3w^6x^{12}y^3+z^2w^7x^{12}y^3$\\
$\mathcal{M}_{3,3}(37,10)=zw^8x^{10}y^3+zw^5x^8y^3+zw^6x^8y^3+2z^2w^7x^{10}y^3+z^2w^6x^{10}y^3$\\
$\mathcal{M}_{3,3}(37,11)=0$\\
$\mathcal{M}_{3,3}(37,12)=z^4w^6x^{14}y^3+z^3w^7x^{14}y^3+z^3w^6x^{12}y^3+z^5w^6x^{14}y^3+z^4w^5x^{12}y^3+z^3w^8x^{12}y^3+z^3w^7x^{12}y^3$\\
$\mathcal{M}_{3,3}(37,13)=2z^3w^5x^{12}y^3+z^2w^4x^{10}y^3+z^3w^7x^{14}y^3+z^2w^6x^{12}y^3+z^3w^6x^{14}y^3$\\
$\mathcal{M}_{3,3}(37,14)=2z^2w^6x^{14}y^3+z^2w^7x^{14}y^3$\\
$\mathcal{M}_{3,3}(37,15)=z^3w^7x^{14}y^3+z^3w^6x^{12}y^3+z^3w^8x^{12}y^3+z^3w^7x^{12}y^3$\\
$\mathcal{M}_{3,3}(37,16)=0$\\
$\mathcal{M}_{3,3}(37,17)=2z^2w^5x^{10}y^3+z^3w^6x^{12}y^3+z^2w^6x^{10}y^3+z^3w^7x^{12}y^3$\\
$\mathcal{M}_{3,3}(37,18)=z^3w^7x^{14}y^3+z^2w^6x^{12}y^3$\\
$\mathcal{M}_{3,3}(37,19)=zw^8x^{10}y^3+2zw^7x^{10}y^3+2zw^6x^{10}y^3$\\
$\mathcal{M}_{3,3}(37,20)=0$\\
$\mathcal{M}_{3,3}(37,21)=z^2w^5x^{10}y^3+z^3w^7x^{14}y^3+z^3w^6x^{12}y^3$\\
$\mathcal{M}_{3,3}(37,22)=z^2w^7x^{14}y^3$\\
$\mathcal{M}_{3,3}(37,23)=zw^8x^{14}y^3$\\
$\mathcal{M}_{3,3}(37,24)=z^4w^7x^{14}y^3+z^3w^6x^{12}y^3$\\
$\mathcal{M}_{3,3}(37,25)=z^2w^7x^{10}y^3+z^2w^6x^{10}y^3+z^3w^8x^{12}y^3$\\
$\mathcal{M}_{3,3}(37,26)=z^2w^7x^{12}y^3+z^2w^8x^{12}y^3$\\
$\mathcal{M}_{3,3}(37,27)=zw^8x^{10}y^3+zw^7x^{10}y^3+zw^9x^{10}y^3$\\
$\mathcal{M}_{3,3}(37,28)=0$\\
$\mathcal{M}_{3,3}(37,29)=z^2w^4x^{10}y^3+5z^3w^5x^{10}y^3+z^3w^6x^{12}y^3+zw^5x^{10}y^3+2z^2w^4x^8y^3+3z^3w^6x^{10}y^3+2z^2w^7x^{10}y^3+z^4w^4x^{10}y^3+z^2w^5x^8y^3$\\
$\mathcal{M}_{3,3}(37,30)=0$\\
$\mathcal{M}_{3,3}(37,31)=z^2w^5x^{10}y^3+z^2w^4x^{10}y^3+z^3w^5x^{10}y^3+z^3w^6x^{12}y^3+zw^5x^{10}y^3+2z^2w^7x^{10}y^3+z^4w^4x^{10}y^3+z^2w^6x^{10}y^3$\\
$\mathcal{M}_{3,3}(37,32)=z^4w^6x^{14}y^3+z^5w^5x^{12}y^3+z^5w^6x^{14}y^3+z^4w^4x^{10}y^3$\\
$\mathcal{M}_{3,3}(37,33)=2z^2w^5x^{10}y^3+z^2w^5x^{12}y^3+z^3w^7x^{14}y^3+2z^3w^6x^{12}y^3+z^2w^6x^{10}y^3+z^3w^7x^{12}y^3$\\
$\mathcal{M}_{3,3}(37,34)=z^2w^5x^{12}y^3+z^3w^7x^{14}y^3+z^3w^6x^{12}y^3+z^2w^7x^{12}y^3$\\
$\mathcal{M}_{3,3}(37,35)=z^2w^5x^{10}y^3+2zw^8x^{10}y^3+w^6x^{10}y^3+zw^6x^{10}y^3$\\
$\mathcal{M}_{3,3}(37,36)=0$\\
$\mathcal{M}_{3,3}(37,37)=2z^2w^4x^{10}y^3+2z^3w^6x^{12}y^3+2zw^5x^{10}y^3+3z^2w^7x^{10}y^3+2z^4w^4x^{10}y^3$\\
\normalsize
% \begin{spacing}{0.30}
%\section*{Code C$^{++}$ ayant permis la génération de la matrice $\mathcal{A}_{F3}$}
% \end{spacing}
%\addcontentsline{toc}{section}{Code C$^{++}$ ayant permis la génération de la matrice $\mathcal{A}_{F3}$}
%\lstset {language=C++,language=c++,
%basicstyle=\ttfamily\small, %
%identifierstyle=\color{cyan}, %
%keywordstyle=\color{blue}, %
%stringstyle=\color{black!60}, %
%commentstyle=\it\color{green!95!yellow!1}, %
%columns=flexible, %
%tabsize=2, %
%extendedchars=true, %
%showspaces=false, %
%showstringspaces=false, %
%breaklines=true, %
%breakautoindent=true, %
%captionpos=b}
%\begin{lstlisting}

%\end{lstlisting}
\bibliographystyle{apalike-uqam}
\bibliography{un_fichier_bib}
\end{document}
