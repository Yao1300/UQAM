\chapter*{Conclusion et perspectives}
\addcontentsline{toc}{chapter}{CONCLUSION ET PERSPECTIVES}
À travers ce travail, nous avons mis en place des notions fondamentales  suffisantes pour énumérer les polyominos inscrits dans un rectangle  de base $B$ et de hauteur $H$ quelconques.  Au cours de nos travaux nous avons non seulement porté nos investigations au-delà de ce qui a été déjà fait dans la littérature en ce qui concerne l’énumération des polyominos inscrits, notamment l’énumération des polyominos sur la base de l’aire, le périmètre, le nombre de feuilles, la hauteur et de nombre de composantes connexes (le cas de forêts de polyominos inscrites), nous n’avons pas tenu compte d’une classe particulière de polyominos comme cela s’est fait jusqu’à présent en littérature. L’objectif ainsi visé par nos recherches est donc plus général.


En nous inspirant de la notion de la théorie de langages et automates, nous avons construit les   automates $\mathcal{A}_{B}$ et $\mathcal{A}_{FB}$ qui permettent respectivement d’énumérer les polyominos et les forêts de polyominos inscrits dans un rectangle du type $B$ en tenant compte des paramètres énumérés ci-dessus. Pour y arriver, étant donné un polyomino inscrit, nous avons su remarquer que la succession des lignes constituant ce polyomino, du haut vers le bas, se présente comme étant des éléments  dont l’agencement dans cet ordre de succession nous donne ledit polyomino. Cette conception nous a donc permis de concevoir les états de nos deux automates. C’est ainsi que nous avons défini un état comme étant une suite de cellules  ou de cases non occupées couplée d’un mot de l’alphabet $\{0,1,2,3\}$ le tout affecté d’une partition non croisée de composantes connexes, une notion qui a été beaucoup développée surtout dans le second chapitre.
   Pour des raisons de compatibilité entre les états, relevant de la façon dont les lignes d'un polyomino ou forêt sont agencées entre elles  et particulièrement  pour des raisons de connexité dans un polyomino, nous avons défini des règles de transition entre états.  Que  ce soit un polyomino ou une forêt de polyominos, dans la stratégie utilisée dans ce travail, l’un ou l’autre est caractérisé par les états le constituant et l’ordre de succession de ces derniers. Aussi, une transition entre deux états est vue comme un ensemble constitué d’un gain d’aire, de périmètre et de hauteur et un gain ou une perte de nombre de feuilles et de nombre de composantes connexes (cas des forêts de polyominos). Cet ensemble est ici représenté par un polynôme  de Laurent dont le terme général est fonction des paramètres que nous avons considérés. Des formules exactes, permettant le calcul des variations de chacun de ces paramètres ont été établies, notamment les variations d’aires, de périmètres, de nombre de feuilles et de composantes connexes.
   
    Grâce aux transitions entre états, nous avons établi les matrices de transitions relativement aux automates ainsi construits dont les entrées représentent chacune la transition d'un état à l'autre. Cette représentation matricielle des automates nous fournit assez d'informations pour la suite de nos travaux de recherches. Notamment des suites de transitions, on a pu déduire que lorsqu'on élève la matrice de transitions à une puissance $H-1$, $H$ quelconque, nous obtenons une matrice dont chaque entrée $a_{ij}$ représente tous les chemins de l'état $e_{i}$ à l'état $e_{j}$ et fournit le nombre de polyominos de hauteur $H$ ayant un certain nombre d'aire, de périmètre, de feuilles et de composantes connexes. Grâce aux propositions \ref{prop2606221} et \ref{inscr1} nous distinguons parmi tous les polyominos générés ceux qui sont inscrits dans le rectangle $R_{B,H}$.
    
     Toutes les formules établies dans le cadre de ce mémoire ont été programmées et implémentées dans le langage $C++$. À partir de ces formules ainsi implémentées, nous avons construit les matrices de transitions $\mathcal{M}_{F3}$  et $\mathcal{M}_{3}$ respectivement de l'automate des forêts de polyominos et celui des polyominos contenus dans les rectangles du type $B$, dont on peut retrouver les résultats dans l'annexe attaché à ce document. Toujours dans l'annexe, nous avons présenté quelques résultats sur les polyominos inscrits dans le rectangle $R_{3,4}$.
     
     Nous avons validé les matrices $\mathcal{M}_{2}$, $\mathcal{M}_{3}$ et $\mathcal{M}_{F3}$ en déterminant leur série génératrice. Les coefficients des séries génératrices de $\mathcal{M}_{2}$ et $\mathcal{M}_{3}$  sont  les termes des suites  respectivement numéros $A034182$ et $A034184$ de l'\emph{OEIS}.
     
     
     Nous comptons dans la suite de nos travaux 
     \begin{itemize}
     \item établir des méthodes de récurrence permettant, pour tout entier naturel strictement positif $B$ donné,  de générer tous les états des automates $\mathcal{A}_{B}$ et $\mathcal{A}_{FB}$;
     \item établir à partir de la matrice de tansitions $\mathcal{A}_{B}$ ou $\mathcal{A}_{FB}$  une formule qui qui tenant compte de l'aire, du périmètre, de nombre de feuilles et de composantes connexes, nous donne le nombre de polyominos inscrits  dans un rectangle $R_{B,H}$ caractérisés par ces données;
     \item et étudier les cas particulier de polyominos notamment les polyominos arbres et serpents. 
     \end{itemize}
   
  