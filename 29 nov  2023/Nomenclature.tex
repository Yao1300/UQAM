\chapter*{Nomenclature}
Dans cette partie, nous allons lister quelques notations utiles pour la suite.
\begin{eqnarray*}
a & & \textit{ représente l'aire d'une cellule d'un polyomino}\\
p & & \textit{ est le périmètre d'une cellule d'un polyomino}\\
H & & \textit{ représente la hauteur du rectangle}\\
f & & \textit{ désigne le nombre de feuilles d'un polyomino}\\
c & & \textit{  le nombre de composantes connexe d'une forêt de polyominos}\\
R_{B} & & \textit{ désigne la famille des rectangles de largeur $B$}\\
R_{B,H} & & \textit{ désigne un rectangle de largeur $B$ et de hauteur $H$, un élément de $R_{B}$}\\
w^{a}x^{p}yz^{f} & &\textit{ correspond à une transition élémentaire de } a \textit{ cellules, }  \textit{de périmètre  }p, \textit{ de hauteur $1$ }\\ 
& & \textit{avec une variation de }  \vert f\vert  \textit{ feuilles}.\\
P_{B,H} & & \textit{désigne un polyomino inscrit dans un rectangle de largeur $B$ et de hauteur $H$} \\
FP_{B,H} & & \textit{une forêt de polyominos inscrite dans un rectangle de largeur } B \textit{ et de hauteur } H\\
\mathcal{M}_{B} & & \textit{ la matrice de transfert des polyominos inscrits dans  rectangle de}\\
& & \textit{ largeur } B\\
\mathcal{M}_{FB} & & \textit{ la matrice de transfert des forêts de polyominos inscrites dans  rectangle}\\
& & \textit{de largeur } B\\
%\mathcal{M}_{T_{B}} & & \textit{ la matrice des   de transitions  des polyominos}\\
%& & \textit{inscrits dans un rectangle de largeur } B\\
\mathcal{A}_{P_{B}} & & \textit{ l'automate générateur des polyominos inscrits dans un rectangle de}\\
 & &  \textit{largeur } B\\
\mathcal{A}_{FP_{B}} & & \textit{ l'automate générateur des forêts de polyominos inscrites dans un}\\
& & \textit{rectangle de largeur } B
\end{eqnarray*}
\begin{tikzpicture}
%\draw (2, 2) circle (3cm);
%\draw[red, thick, dashed] (2, 2) circle (4cm);
%\draw (2, 2) ellipse (3cm and 1cm);
%\end{tikzpicture}
%\begin{tikzpicture}
%\draw [] (n1) --(n2) node[midway, above]{L};

\end{tikzpicture}
